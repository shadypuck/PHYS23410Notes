\documentclass[../psets.tex]{subfiles}

\pagestyle{main}
\renewcommand{\leftmark}{Problem Set \thesection}
\setcounter{section}{6}

\begin{document}




\section{Spin}
\begin{enumerate}
    \item \marginnote{3/2:}In class, we showed that one can find a matrix representation for the components of the spin operator given by
    \begin{align}
        \hat{S}_x &= \frac{\hbar}{2}
        \begin{pmatrix}
            0 & 1\\
            1 & 0\\
        \end{pmatrix}&
        \hat{S}_y &= \frac{\hbar}{2}
        \begin{pmatrix}
            0 & -i\\
            i & 0\\
        \end{pmatrix}&
        \hat{S}_z &= \frac{\hbar}{2}
        \begin{pmatrix}
            1 & 0\\
            0 & -1\\
        \end{pmatrix}
    \end{align}
    \begin{enumerate}
        \item Use matrix multiplication to show that they fulfill the proper commutator algebra associated with angular momentum components.
        \begin{proof}
            We will proceed one relation at a time through all three relations. Let's begin.\par
            \underline{$[\hat{S}_x,\hat{S}_y]=i\hbar\hat{S}_z$}:
            \begin{align*}
                [\hat{S}_x,\hat{S}_y] &= \hat{S}_x\hat{S}_y-\hat{S}_y\hat{S}_x\\
                &= \frac{\hbar}{2}
                \begin{pmatrix}
                    0 & 1\\
                    1 & 0\\
                \end{pmatrix}
                \cdot\frac{\hbar}{2}
                \begin{pmatrix}
                    0 & -i\\
                    i & 0\\
                \end{pmatrix}
                -\frac{\hbar}{2}
                \begin{pmatrix}
                    0 & -i\\
                    i & 0\\
                \end{pmatrix}
                \cdot\frac{\hbar}{2}
                \begin{pmatrix}
                    0 & 1\\
                    1 & 0\\
                \end{pmatrix}\\
                &= \frac{\hbar^2}{4}\left[
                    \begin{pmatrix}
                        i & 0\\
                        0 & -i\\
                    \end{pmatrix}
                    -
                    \begin{pmatrix}
                        -i & 0\\
                        0 & i\\
                    \end{pmatrix}
                \right]\\
                &= \frac{\hbar^2}{4}
                \begin{pmatrix}
                    2i & 0\\
                    0 & -2i\\
                \end{pmatrix}\\
                &= i\hbar\cdot\frac{\hbar}{2}
                \begin{pmatrix}
                    1 & 0\\
                    0 & -1\\
                \end{pmatrix}\\
                &= i\hbar\hat{S}_z
            \end{align*}
            \underline{$[\hat{S}_y,\hat{S}_z]=i\hbar\hat{S}_x$}:
            \begin{align*}
                [\hat{S}_y,\hat{S}_z] &= \hat{S}_y\hat{S}_z-\hat{S}_z\hat{S}_y\\
                &= \frac{\hbar}{2}
                \begin{pmatrix}
                    0 & -i\\
                    i & 0\\
                \end{pmatrix}
                \cdot\frac{\hbar}{2}
                \begin{pmatrix}
                    1 & 0\\
                    0 & -1\\
                \end{pmatrix}
                -\frac{\hbar}{2}
                \begin{pmatrix}
                    1 & 0\\
                    0 & -1\\
                \end{pmatrix}
                \cdot\frac{\hbar}{2}
                \begin{pmatrix}
                    0 & -i\\
                    i & 0\\
                \end{pmatrix}\\
                &= \frac{\hbar^2}{4}\left[
                    \begin{pmatrix}
                        0 & i\\
                        i & 0\\
                    \end{pmatrix}
                    -
                    \begin{pmatrix}
                        0 & -i\\
                        -i & 0\\
                    \end{pmatrix}
                \right]\\
                &= \frac{\hbar^2}{4}
                \begin{pmatrix}
                    0 & 2i\\
                    2i & 0\\
                \end{pmatrix}\\
                &= i\hbar\cdot\frac{\hbar}{2}
                \begin{pmatrix}
                    0 & 1\\
                    1 & 0\\
                \end{pmatrix}\\
                &= i\hbar\hat{S}_x
            \end{align*}
            \underline{$[\hat{S}_z,\hat{S}_x]=i\hbar\hat{S}_y$}:
            \begin{align*}
                [\hat{S}_z,\hat{S}_x] &= \hat{S}_z\hat{S}_x-\hat{S}_x\hat{S}_z\\
                &= \frac{\hbar}{2}
                \begin{pmatrix}
                    1 & 0\\
                    0 & -1\\
                \end{pmatrix}
                \cdot\frac{\hbar}{2}
                \begin{pmatrix}
                    0 & 1\\
                    1 & 0\\
                \end{pmatrix}
                -\frac{\hbar}{2}
                \begin{pmatrix}
                    0 & 1\\
                    1 & 0\\
                \end{pmatrix}
                \cdot\frac{\hbar}{2}
                \begin{pmatrix}
                    1 & 0\\
                    0 & -1\\
                \end{pmatrix}\\
                &= \frac{\hbar^2}{4}\left[
                    \begin{pmatrix}
                        0 & 1\\
                        -1 & 0\\
                    \end{pmatrix}
                    -
                    \begin{pmatrix}
                        0 & -1\\
                        1 & 0\\
                    \end{pmatrix}
                \right]\\
                &= \frac{\hbar^2}{4}
                \begin{pmatrix}
                    0 & 2\\
                    -2 & 0\\
                \end{pmatrix}\\
                &= i\hbar\cdot\frac{\hbar}{2}
                \begin{pmatrix}
                    0 & -i\\
                    i & 0\\
                \end{pmatrix}\\
                &= i\hbar\hat{S}_y
            \end{align*}
        \end{proof}
        \item Compute $\hat{S}_i^2$ ($i=x,y,z$). If you perform a measurement, what possible values of the components of angular momentum can you get? \emph{Hint}: There are 2 possible values.
        \begin{proof}
            We have that
            \begin{align*}
                \hat{S}_x^2 &= \frac{\hbar^2}{4}
                \begin{pmatrix}
                    0 & 1\\
                    1 & 0\\
                \end{pmatrix}
                \begin{pmatrix}
                    0 & 1\\
                    1 & 0\\
                \end{pmatrix}&
                    \hat{S}_y^2 &= \frac{\hbar^2}{4}
                    \begin{pmatrix}
                        0 & -i\\
                        i & 0\\
                    \end{pmatrix}
                    \begin{pmatrix}
                        0 & -i\\
                        i & 0\\
                    \end{pmatrix}&
                        \hat{S}_z^2 &= \frac{\hbar^2}{4}
                        \begin{pmatrix}
                            1 & 0\\
                            0 & -1\\
                        \end{pmatrix}
                        \begin{pmatrix}
                            1 & 0\\
                            0 & -1\\
                        \end{pmatrix}\\
                \Aboxed{\hat{S}_x^2 &= \frac{\hbar^2}{4}I}&
                    \Aboxed{\hat{S}_y^2 &= \frac{\hbar^2}{4}I}&
                        \Aboxed{\hat{S}_z^2 &= \frac{\hbar^2}{4}I}
            \end{align*}
            As to the second part of the question, note that the possible values of the components of angular momentum correspond to the possible eigenvalues of $\hat{S}_i$. Observe that the matrices for $\hat{S}_i$ are\dots
            \begin{enumerate}
                \item Hermitian;
                \item Traceless;
                \item Have determinant $-\hbar^2/4$.
            \end{enumerate}
            These three properties give us everything we need to find the eigenvalues. To set a notation, let $\lambda_1,\lambda_2$ denote the eigenvalues of $\hat{S}_i$ ($i=x,y$). Now, it is a theorem of linear algebra that the sum of the eigenvalues equals the trace. Hence, property (ii) tells us that
            \begin{equation*}
                \lambda_1+\lambda_2 = {\tr}(\hat{S}_x)
                = {\tr}(\hat{S}_y)
                = 0
            \end{equation*}
            Similarly, it is a theorem of linear algebra that the product of the eigenvalues equals the determinant. Hence, property (iii) tells us that
            \begin{equation*}
                \lambda_1\lambda_2 = {\det}(\hat{S}_x)
                = {\det}(\hat{S}_y)
                = -\frac{\hbar^2}{4}
            \end{equation*}
            Lastly, it is a theorem of linear algebra that Hermitian matrices have real eigenvalues. Thus, property (iii) tells us that we can solve the two-equation, two-variable system
            \begin{equation*}
                \begin{cases}
                    \lambda_1+\lambda_2 = 0\\
                    \lambda_1\lambda_2 = -\frac{\hbar^2}{4}
                \end{cases}
            \end{equation*}
            over the real numbers $\R$ to obtain, WLOG, that
            \begin{align*}
                \lambda_1 &= \frac{\hbar}{2}&
                \lambda_2 &= -\frac{\hbar}{2}
            \end{align*}
            This provides the desired verification.\par
            \underline{Alternate, simpler method of solving the second half of the question}: Since each $\hat{S}_i^2$ has eigenvalue $\hbar^2/4$, it follows that every $\hat{S}_i$ has eigenvalue\footnote{Is this implication well-supported mathematically?? What constraints do we need? Is it important that the $\hat{S}_i$ are operators on a \emph{complex} vector space? Do we still need the traceless and/or Hermitian conditions somewhere, or are we already using them implicitly? Which theorem allows us to do this? I'm thinking the answer might lie somewhere in Chapter 7 of \emph{Linear Algebra Done Right}...}
            \begin{equation*}
                \sqrt{\frac{\hbar^2}{4}} = \pm\frac{\hbar}{2}
            \end{equation*}
        \end{proof}
        \pagebreak
        \item Take a generic, well-normalized spin state
        \begin{equation}
            \chi =
            \begin{pmatrix}
                c_+\\
                c_-\\
            \end{pmatrix}
        \end{equation}
        with $|c_+|^2+|c_-|^2=1$. What is the probability of measuring a value of $\hat{S}_z=\hbar/2$? \emph{Hint}: Express $\chi$ as a linear combination of eigenstates of $\hat{S}_z$ with eigenvalues $\pm 1/2$.
        \begin{proof}
            Taking the hint, let
            \begin{equation*}
                \ket{\chi} = c_+\ket{\tfrac{1}{2},\tfrac{1}{2}}+c_-\ket{\tfrac{1}{2},-\tfrac{1}{2}}
            \end{equation*}
            Then, as in other quantum systems, the probability of measuring a certain eigenvalue of $\hat{S}_z$ when it is in the well-normalized spin state $\chi$ can be determined from the expression for the expected value of $\hat{S}_z$ in $\chi$. In particular, we have that
            \begin{align*}
                \ev{\hat{S}_z}{\chi} &= (c_+^*\bra{\tfrac{1}{2},\tfrac{1}{2}}+c_-^*\bra{\tfrac{1}{2},-\tfrac{1}{2}})\hat{S}_z(c_+\ket{\tfrac{1}{2},\tfrac{1}{2}}+c_-\ket{\tfrac{1}{2},-\tfrac{1}{2}})\\
                &= (c_+^*\bra{\tfrac{1}{2},\tfrac{1}{2}}+c_-^*\bra{\tfrac{1}{2},-\tfrac{1}{2}})\frac{\hbar}{2}(c_+\ket{\tfrac{1}{2},\tfrac{1}{2}}-c_-\ket{\tfrac{1}{2},-\tfrac{1}{2}})\\
                &= \left( \frac{\hbar}{2} \right)|c_+|^2+\left( -\frac{\hbar}{2} \right)|c_-|^2
            \end{align*}
            Thus, the expected value of $\hat{S}_z$ is a weighted average of $\pm\hbar/2$. More specifically, we can expect to measure a value of $\hbar/2$ (for instance) every $|c_+|^2/1$ times. In other words, the probability of measuring a value of $\hat{S}_z=\hbar/2$ is
            \begin{equation*}
                \boxed{|c_+|^2}
            \end{equation*}
        \end{proof}
        \item What are the mean values of $\hat{S}_x,\hat{S}_y,\hat{S}_z$ in the state $\chi$? \emph{Hint}: Use the vector notation to compute the mean values.
        \begin{proof}
            We just computed the mean value of $\hat{S}_z$ in part (c). To reiterate, though,
            \begin{equation*}
                \boxed{\ev{\hat{S}_z}{\chi} = \left( \frac{\hbar}{2} \right)|c_+|^2+\left( -\frac{\hbar}{2} \right)|c_-|^2}
            \end{equation*}
            For $\hat{S}_x,\hat{S}_y$, we could follow a similar approach to part (c). Alternatively, we can take the hint and use vector notation as follows.\par
            For $\hat{S}_x$, we have
            \begin{align*}
                \ev{\hat{S}_x}{\chi} &= \frac{\hbar}{2}
                \begin{pmatrix}
                    c_+^* & c_-^*\\
                \end{pmatrix}
                \begin{pmatrix}
                    0 & 1\\
                    1 & 0\\
                \end{pmatrix}
                \begin{pmatrix}
                    c_+\\
                    c_-\\
                \end{pmatrix}\\
                &= \frac{\hbar}{2}(c_+^*c_-+c_-^*c_+)\\
                \Aboxed{\ev{\hat{S}_x}{\chi} &= \hbar\re(c_+^*c_-)}
            \end{align*}
            For $\hat{S}_y$, we have
            \begin{align*}
                \ev{\hat{S}_y}{\chi} &= \frac{\hbar}{2}
                \begin{pmatrix}
                    c_+^* & c_-^*\\
                \end{pmatrix}
                \begin{pmatrix}
                    0 & -i\\
                    i & 0\\
                \end{pmatrix}
                \begin{pmatrix}
                    c_+\\
                    c_-\\
                \end{pmatrix}\\
                &= \frac{\hbar}{2}\cdot\frac{c_+^*c_--c_-^*c_+}{2i}\cdot 2\\
                \Aboxed{\ev{\hat{S}_y}{\chi} &= \hbar\im(c_+^*c_-)}
            \end{align*}
        \end{proof}
        \item Use the result of part (d), together with the values of $\hat{S}_i^2$, to show that the uncertainty principle is fulfilled, i.e., that
        \begin{equation}
            \sigma_{\hat{S}_x}\sigma_{\hat{S}_y} \geq \frac{1}{2}|\ev{[\hat{S}_x,\hat{S}_y]}{\chi}|
        \end{equation}
        \emph{Hint}: WLOG, let $c_+=\cos(\theta_s/2)\e[i\alpha]$ and $c_-=\sin(\theta_s/2)\e[i\beta]$. Hence, $c_+c_-^*+c_-c_+^*=\sin(\theta_s)\cos(\alpha-\beta)$, $c_+c_-^*-c_-c_+^*=i\sin(\theta_s)\sin(\alpha-\beta)$, and $|c_+|^2-|c_-|^2=\cos(\theta_s)$.
        \begin{proof}
            As we computed in part (b),
            \begin{equation*}
                \hat{S}_x^2 = \hat{S}_y^2
                = \hat{S}_z^2
                = \frac{\hbar^2}{4}I
            \end{equation*}
            Thus, we have that
            \begin{align*}
                \ev{\hat{S}_x^2}{\chi} &= \frac{\hbar^2}{4}\braket{\chi} = \frac{\hbar^2}{4}&
                \ev{\hat{S}_y^2}{\chi} &= \frac{\hbar^2}{4}\braket{\chi} = \frac{\hbar^2}{4}
            \end{align*}
            Additionally, recall from part (d) that
            \begin{align*}
                \ev{\hat{S}_x}{\chi} &= \hbar\re(c_+^*c_-)&
                \ev{\hat{S}_y}{\chi} &= \hbar\im(c_+^*c_-)
            \end{align*}
            Now taking the hint, let
            \begin{align*}
                c_+ &= \cos(\frac{\theta_s}{2})\e[i\alpha]&
                c_- &= \sin(\frac{\theta_s}{2})\e[i\beta]
            \end{align*}
            Then taking the hint and going back a step in the part (d) derivation, we obtain
            \begin{align*}
                \ev{\hat{S}_x}{\chi} &= \hbar\re[\cos(\frac{\theta_s}{2})\e[-i\alpha]\sin(\frac{\theta_s}{2})\e[i\beta]]\\
                &= \frac{\hbar}{2}\cdot 2\sin(\frac{\theta_s}{2})\cos(\frac{\theta_s}{2})\cdot\re[\e[i(\beta-\alpha)]]\\
                &= \frac{\hbar}{2}\sin(\theta_s)\cos(\beta-\alpha)\\
                &= \frac{\hbar}{2}\sin(\theta_s)\cos(\alpha-\beta)
            \end{align*}
            and
            \begin{align*}
                \ev{\hat{S}_y}{\chi} &= \hbar\im(c_+^*c_-)\\
                &= \frac{\hbar}{2}\cdot\frac{c_+^*c_--c_-^*c_+}{2i}\cdot 2\\
                &= \frac{\hbar}{2}\cdot -\frac{i\sin(\theta_s)\sin(\alpha-\beta)}{2i}\cdot 2\\
                &= -\frac{\hbar}{2}\sin(\theta_s)\sin(\alpha-\beta)
            \end{align*}
            It follows that
            \begin{align*}
                \sigma_{\hat{S}_x}^2 &= \ev{\hat{S}_x^2}{\chi}-(\ev{\hat{S}_x}{\chi})^2\\
                &= \frac{\hbar^2}{4}-\frac{\hbar^2}{4}\sin^2(\theta_s)\cos^2(\alpha-\beta)\\
                &= \frac{\hbar^2}{4}\left[ 1-\sin^2(\theta_s)\cos^2(\alpha-\beta) \right]
            \end{align*}
            and
            \begin{align*}
                \sigma_{\hat{S}_y}^2 &= \ev{\hat{S}_y^2}{\chi}-(\ev{\hat{S}_y}{\chi})^2\\
                &= \frac{\hbar^2}{4}-\frac{\hbar^2}{4}\sin^2(\theta_s)\sin^2(\alpha-\beta)\\
                &= \frac{\hbar^2}{4}\left[ 1-\sin^2(\theta_s)\sin^2(\alpha-\beta) \right]
            \end{align*}
            On the other side of the equality, we have that
            \begin{align*}
                \frac{1}{2}|\ev{[\hat{S}_x,\hat{S}_y]}{\chi}| &= \frac{1}{2}|i\hbar\ev{\hat{S}_z}{\chi}|\\
                &= \frac{\hbar}{2}\left| \left( \frac{\hbar}{2} \right)|c_+|^2+\left( -\frac{\hbar}{2} \right)|c_-|^2 \right|\\
                &= \frac{\hbar^2}{4}(|c_+|^2-|c_-|^2)\\
                &= \frac{\hbar^2}{4}\cos(\theta_s)
            \end{align*}
            Thus, we have that
            \begin{align*}
                \sigma_{\hat{S}_x}^2\cdot\sigma_{\hat{S}_y}^2 &\stackrel{?}{\geq} \frac{1}{4}|\ev{[\hat{S}_x,\hat{S}_y]}{\chi}|^2\\
                \frac{\hbar^2}{4}\left[ 1-\sin^2(\theta_s)\cos^2(\alpha-\beta) \right]\cdot\frac{\hbar^2}{4}\left[ 1-\sin^2(\theta_s)\sin^2(\alpha-\beta) \right] &\stackrel{?}{\geq} \frac{\hbar^4}{16}\cos^2(\theta_s)\\
                \left[ 1-\sin^2(\theta_s)\cos^2(\alpha-\beta) \right]\left[ 1-\sin^2(\theta_s)\sin^2(\alpha-\beta) \right] &\stackrel{?}{\geq} \cos^2(\theta_s)\\
                1-\sin^2(\theta_s)\cos^2(\alpha-\beta)-\sin^2(\theta_s)\sin^2(\alpha-\beta)+\sin^4(\theta_s)\cos^2(\alpha-\beta)\sin^2(\alpha-\beta) &\stackrel{?}{\geq} \cos^2(\theta_s)\\
                1-\sin^2(\theta_s)[\cos^2(\alpha-\beta)+\sin^2(\alpha-\beta)]+\sin^4(\theta_s)\cos^2(\alpha-\beta)\sin^2(\alpha-\beta) &\stackrel{?}{\geq} \cos^2(\theta_s)\\
                1-\sin^2(\theta_s)\cdot 1+\sin^4(\theta_s)\cos^2(\alpha-\beta)\sin^2(\alpha-\beta) &\stackrel{?}{\geq} \cos^2(\theta_s)\\
                [1-\sin^2(\theta_s)]+\sin^4(\theta_s)\cos^2(\alpha-\beta)\sin^2(\alpha-\beta) &\stackrel{?}{\geq} \cos^2(\theta_s)\\
                \cos^2(\theta_s)+\sin^4(\theta_s)\cos^2(\alpha-\beta)\sin^2(\alpha-\beta) &\stackrel{?}{\geq} \cos^2(\theta_s)\\
                \sin^4(\theta_s)\cos^2(\alpha-\beta)\sin^2(\alpha-\beta) &\stackrel{?}{\geq} 0\\
                [\sin^2(\theta_s)\cos(\alpha-\beta)\sin(\alpha-\beta)]^2 &\stackrel{\checkmark}{\geq} 0
            \end{align*}
        \end{proof}
        \item What are the results of part (d) if you take an eigenstate of $\hat{S}_z$ with eigenvalue $\hbar/2$ ($\theta_s=\alpha=0$)?
        \begin{proof}
            Using the coordinate changes in the hint for part (e), we know that $\theta_s=\alpha=0$ implies that
            \begin{align*}
                c_+ &= \cos(\frac{0}{2})\e[i\cdot 0] = 1&
                c_- &= \sin(\frac{0}{2})\e[i\cdot\beta] = 0
            \end{align*}
            Thus, substituting into the results from part (d) and algebraically simplifying, we obtain
            \begin{align*}
                \Aboxed{\ev{\hat{S}_z}{\chi} &= \frac{\hbar}{2}}&
                \Aboxed{\ev{\hat{S}_x}{\chi} &= 0}&
                \Aboxed{\ev{\hat{S}_y}{\chi} &= 0}
            \end{align*}
        \end{proof}
    \end{enumerate}
    \item Consider the interaction of the magnetic moment induced by the spin of a particle with a magnetic field. The Hamiltonian is given by
    \begin{equation}
        \hat{H} = -\gamma\hat{\vec{S}}\hat{\vec{B}}
    \end{equation}
    with corresponding Schr\"{o}dinger equation
    \begin{equation}
        \hat{H}\chi = i\hbar\pdv{\chi}{t}
    \end{equation}
    \begin{enumerate}
        \item Re-derive the solution for $\chi(t)$ we presented in class.
        \begin{proof}
            Combining the various parts of the question, we see that we are seeking to solve
            \begin{equation*}
                -\gamma\vec{B}\vec{S}
                \begin{pmatrix}
                    \chi_+\\
                    \chi_-\\
                \end{pmatrix}
                = i\hbar\pdv{t}
                \begin{pmatrix}
                    \chi_+\\
                    \chi_-\\
                \end{pmatrix}
            \end{equation*}
            Choose
            \begin{equation*}
                \vec{B} = B\hat{z}
            \end{equation*}
            Observe that under this choice
            \begin{equation*}
                \vec{B}\vec{S} = B\hat{z}\cdot\vec{S}
                = B\hat{S}_z
                = \frac{B\hbar}{2}
                \begin{pmatrix}
                    1 & 0\\
                    0 & -1\\
                \end{pmatrix}
            \end{equation*}
            \item Thus, the problem becomes
            \begin{equation*}
                -\frac{\gamma B\hbar}{2}
                \begin{pmatrix}
                    1 & 0\\
                    0 & -1\\
                \end{pmatrix}
                \begin{pmatrix}
                    \chi_+\\
                    \chi_-\\
                \end{pmatrix}
                = i\hbar\pdv{t}
                \begin{pmatrix}
                    \chi_+\\
                    \chi_-\\
                \end{pmatrix}
            \end{equation*}
            Fortunately, this problem is not that hard to solve. To begin, the above equation splits into the two following ones (technically as components in equal vectors) after a matrix multiplication.
            \begin{align*}
                -\frac{\gamma B\hbar}{2}\chi_+ &= i\hbar\pdv{\chi_+}{t}&
                \frac{\gamma B\hbar}{2}\chi_- &= i\hbar\pdv{\chi_-}{t}
            \end{align*}
            The solutions are then
            \begin{align*}
                \chi_+ &= \chi_+(0)\e[i\gamma Bt/2]&
                \chi_- &= \chi_-(0)\e[-i\gamma Bt/2]
            \end{align*}
            Combining components, the overall solution is
            \begin{equation*}
                \boxed{
                    \chi(t) =
                    \begin{pmatrix}
                        \chi_+(0)\e[i\gamma Bt/2]\\
                        \chi_-(0)\e[-i\gamma Bt/2]\\
                    \end{pmatrix}
                }
            \end{equation*}
        \end{proof}
        \item Compute the probabilities of finding the particle with spin up and down in the $x$- and $y$-directions. \emph{Hint}: The probability can be computed as the modulus square of the component of $\chi(t)$ on eigenstates of spin up and down in the $x$- and $y$-directions. These components may be determined by computing the inner product of $\chi(t)$ with these particular eigenstates.
        \begin{proof}
            Taking the hint, we have in the $x$-direction that
            \begingroup
            \allowdisplaybreaks
            \begin{align*}
                |d_+|^2 ={}& \left|
                    \frac{1}{\sqrt{2}}
                    \begin{pmatrix}
                        1 & 1\\
                    \end{pmatrix}
                    \begin{pmatrix}
                        \chi_+\\
                        \chi_-\\
                    \end{pmatrix}
                \right|^2\\
                ={}& \frac{1}{2}(|\chi_++\chi_-|^2)\\
                ={}& \frac{1}{2}\left| \chi_+(0)\e[i\gamma Bt/2]+\chi_-(0)\e[-i\gamma Bt/2] \right|^2\\
                ={}& \frac{1}{2}\left| |\chi_+(0)|\e[i(\gamma Bt/2+\phi_+)]+|\chi_-(0)|\e[-i(\gamma Bt/2-\phi_-)] \right|^2\\
                \begin{split}
                    ={}& \frac{1}{2}\left[ |\chi_+(0)|\e[-i(\gamma Bt/2+\phi_+)]+|\chi_-(0)|\e[i(\gamma Bt/2-\phi_-)] \right]\\
                    & \cdot\left[ |\chi_+(0)|\e[i(\gamma Bt/2+\phi_+)]+|\chi_-(0)|\e[-i(\gamma Bt/2-\phi_-)] \right]
                \end{split}\\
                ={}& \frac{1}{2}\left[ |\chi_+|^2+|\chi_-|^2+|\chi_+(0)||\chi_-(0)|\cdot 2\cos(\gamma Bt+\phi_+-\phi_-) \right]\\
                \Aboxed{|d_+|^2 ={}& \frac{1}{2}[1+\sin(\theta_s)\cos(\gamma Bt+\phi_+-\phi_-)]}
            \end{align*}
            \endgroup
            and
            \begin{align*}
                |d_-|^2 ={}& \left|
                    \frac{1}{\sqrt{2}}
                    \begin{pmatrix}
                        1 & -1\\
                    \end{pmatrix}
                    \begin{pmatrix}
                        \chi_+\\
                        \chi_-\\
                    \end{pmatrix}
                \right|^2\\
                ={}& \frac{1}{2}(|\chi_+-\chi_-|^2)\\
                ={}& \frac{1}{2}\left| \chi_+(0)\e[i\gamma Bt/2]-\chi_-(0)\e[-i\gamma Bt/2] \right|^2\\
                ={}& \frac{1}{2}\left| |\chi_+(0)|\e[i(\gamma Bt/2+\phi_+)]-|\chi_-(0)|\e[-i(\gamma Bt/2-\phi_-)] \right|^2\\
                \begin{split}
                    ={}& \frac{1}{2}\left[ |\chi_+(0)|\e[-i(\gamma Bt/2+\phi_+)]-|\chi_-(0)|\e[i(\gamma Bt/2-\phi_-)] \right]\\
                    & \cdot\left[ |\chi_+(0)|\e[i(\gamma Bt/2+\phi_+)]-|\chi_-(0)|\e[-i(\gamma Bt/2-\phi_-)] \right]
                \end{split}\\
                ={}& \frac{1}{2}\left[ |\chi_+|^2+|\chi_-|^2-|\chi_+(0)||\chi_-(0)|\cdot 2\cos(\gamma Bt+\phi_+-\phi_-) \right]\\
                \Aboxed{|d_-|^2 ={}& \frac{1}{2}[1-\sin(\theta_s)\cos(\gamma Bt+\phi_+-\phi_-)]}
            \end{align*}
            Analogously, we have in the $y$-direction that\footnote{The key has the first plus sign below flipped to a minus sign. Is it right, or am I?? I do believe I conjugated correctly...}
            \begin{align*}
                |e_+|^2 ={}& \left|
                    \frac{1}{\sqrt{2}}
                    \begin{pmatrix}
                        1 & i\\
                    \end{pmatrix}
                    \begin{pmatrix}
                        \chi_+\\
                        \chi_-\\
                    \end{pmatrix}
                \right|^2\\
                ={}& \frac{1}{2}(|\chi_++i\chi_-|^2)\\
                ={}& \frac{1}{2}\left| \chi_+(0)\e[i\gamma Bt/2]+i\chi_-(0)\e[-i\gamma Bt/2] \right|^2\\
                ={}& \frac{1}{2}\left| |\chi_+(0)|\e[i(\gamma Bt/2+\phi_+)]+i|\chi_-(0)|\e[-i(\gamma Bt/2-\phi_-)] \right|^2\\
                \begin{split}
                    ={}& \frac{1}{2}\left[ |\chi_+(0)|\e[-i(\gamma Bt/2+\phi_+)]-i|\chi_-(0)|\e[i(\gamma Bt/2-\phi_-)] \right]\\
                    & \cdot\left[ |\chi_+(0)|\e[i(\gamma Bt/2+\phi_+)]+i|\chi_-(0)|\e[-i(\gamma Bt/2-\phi_-)] \right]
                \end{split}\\
                ={}& \frac{1}{2}\left[ |\chi_+|^2+|\chi_-|^2-|\chi_+(0)||\chi_-(0)|\cdot 2\sin(\gamma Bt+\phi_+-\phi_-) \right]\\
                \Aboxed{|e_+|^2 ={}& \frac{1}{2}[1-\sin(\theta_s)\sin(\gamma Bt+\phi_+-\phi_-)]}
            \end{align*}
            and
            \begingroup
            \allowdisplaybreaks
            \begin{align*}
                |e_-|^2 ={}& \left|
                    \frac{1}{\sqrt{2}}
                    \begin{pmatrix}
                        1 & -i\\
                    \end{pmatrix}
                    \begin{pmatrix}
                        \chi_+\\
                        \chi_-\\
                    \end{pmatrix}
                \right|^2\\
                ={}& \frac{1}{2}(|\chi_+-i\chi_-|^2)\\
                ={}& \frac{1}{2}\left| \chi_+(0)\e[i\gamma Bt/2]-i\chi_-(0)\e[-i\gamma Bt/2] \right|^2\\
                ={}& \frac{1}{2}\left| |\chi_+(0)|\e[i(\gamma Bt/2+\phi_+)]-i|\chi_-(0)|\e[-i(\gamma Bt/2-\phi_-)] \right|^2\\
                \begin{split}
                    ={}& \frac{1}{2}\left[ |\chi_+(0)|\e[-i(\gamma Bt/2+\phi_+)]+i|\chi_-(0)|\e[i(\gamma Bt/2-\phi_-)] \right]\\
                    & \cdot\left[ |\chi_+(0)|\e[i(\gamma Bt/2+\phi_+)]-i|\chi_-(0)|\e[-i(\gamma Bt/2-\phi_-)] \right]
                \end{split}\\
                ={}& \frac{1}{2}\left[ |\chi_+|^2+|\chi_-|^2+|\chi_+(0)||\chi_-(0)|\cdot 2\sin(\gamma Bt+\phi_+-\phi_-) \right]\\
                \Aboxed{|e_-|^2 ={}& \frac{1}{2}[1+\sin(\theta_s)\sin(\gamma Bt+\phi_+-\phi_-)]}
            \end{align*}
            \endgroup
        \end{proof}
        \item Based on these probabilities, compute the mean values of the spin in the $x$- and $y$-directions and discuss their behavior in time.
        \begin{proof}
            % It would just be
            % \begin{equation*}
            %     \frac{\hbar}{2}(|d_+|^2-|d_-|^2)
            % \end{equation*}
            % See Lectures 9.1-9.2.

            By the definition of the mean value in terms of eigenvalues and their probabilities, we have that
            \begin{align*}
                \ev{\hat{S}_x}{\chi} ={}& \left( \frac{\hbar}{2} \right)|d_+|^2+\left( -\frac{\hbar}{2} \right)|d_-|^2\\
                \begin{split}
                    ={}& \left( \frac{\hbar}{2} \right)\cdot\frac{1}{2}[1+\sin(\theta_s)\cos(\gamma Bt+\phi_+-\phi_-)]\\
                    & +\left( -\frac{\hbar}{2} \right)\cdot\frac{1}{2}[1-\sin(\theta_s)\cos(\gamma Bt+\phi_+-\phi_-)]
                \end{split}\\
                \Aboxed{\ev{\hat{S}_x}{\chi} ={}& \frac{\hbar}{2}\sin(\theta_s)\cos(\gamma Bt+\phi_+-\phi_-)}
            \end{align*}
            and
            \begin{align*}
                \ev{\hat{S}_y}{\chi} ={}& \left( \frac{\hbar}{2} \right)|e_+|^2+\left( -\frac{\hbar}{2} \right)|e_-|^2\\
                \begin{split}
                    ={}& \left( \frac{\hbar}{2} \right)\cdot\frac{1}{2}[1+\sin(\theta_s)\sin(\gamma Bt+\phi_+-\phi_-)]\\
                    & +\left( -\frac{\hbar}{2} \right)\cdot\frac{1}{2}[1-\sin(\theta_s)\sin(\gamma Bt+\phi_+-\phi_-)]
                \end{split}\\
                \Aboxed{\ev{\hat{S}_y}{\chi} ={}& -\frac{\hbar}{2}\sin(\theta_s)\sin(\gamma Bt+\phi_+-\phi_-)}
            \end{align*}
        \end{proof}
    \end{enumerate}
\end{enumerate}




\end{document}