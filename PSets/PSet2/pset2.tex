\documentclass[../psets.tex]{subfiles}

\pagestyle{main}
\renewcommand{\leftmark}{Problem Set \thesection}
\stepcounter{section}

\begin{document}




\section{Infinite Well Motion and Quantum Tunneling}
\begin{enumerate}
    \item \marginnote{1/19:}In class, we demonstrated that given a certain time-independent potential, one can find solutions to the Schr\"{o}dinger equation such that
    \begin{equation}
        -\frac{\hbar^2}{2m}\vec{\nabla}^2\psi_n(\vec{r})+V(\vec{r})\psi_n(\vec{r}) = E_n\psi_n(\vec{r})
    \end{equation}
    Assume now that we are in one dimension, with the potential being a square well:
    \begin{equation}
        \begin{aligned}
            V(x) &\to \infty && \text{for }x\leq 0\text{ and }x\geq a\\
            V(x) &\to 0      && \text{for }0<x<a
        \end{aligned}
    \end{equation}
    Show that in such a case, the solutions are given by
    \begin{equation}
        \psi_n(x) = \sqrt{\frac{2}{a}}\sin(\frac{n\pi x}{a})
    \end{equation}
    due to the fact that in order to get a finite mean energy value, the wave function must vanish at $x=0,a$. The energy eigenstates are given by
    \begin{equation}
        E_n = \frac{n^2\pi^2\hbar^2}{2ma^2}
    \end{equation}
    The factor $\sqrt{2/a}$ comes from the requirement of a good normalized solution, i.e., one with $\braket{\psi_n}=1$.\par
    Now, imagine that at $t=0$, the particle is in the state
    \begin{equation}\label{eqn:PS2E5}
        \psi(x,0) = \frac{A}{\sqrt{a}}\sin(\frac{\pi x}{a})+\sqrt{\frac{3}{5a}}\sin(\frac{3\pi x}{a})+\frac{1}{\sqrt{5a}}\sin(\frac{5\pi x}{a})
    \end{equation}
    where $A$ is a real constant.
    \begin{enumerate}
        \item Find the value of $A$ such that $\psi(x,0)$ is normalized. (Hint: Use $\braket{\psi_n}{\psi_m}=\delta_{nm}$.)
        \item If measurements of the energy are carried out, what are the values that will be found and what are the probabilities of measuring such energies? Calculate the average energy.
        \item Find the expression of the wave function at a later time $t$. (Hint: What is $\psi_n(x,t)$?)
        \item Is the mean value of the position operator independent of time? What about the mean value of the momentum? (Hint: Use symmetry properties with respect to the central point of the well.)
        \item Would the result of part (d) be different if we replaced $\psi_3$ by $\psi_2$ in Eq. \ref{eqn:PS2E5}?
    \end{enumerate}
    \item 
    \begin{enumerate}
        \item Consider now the wave function $\Psi(x,t)$ of a particle moving in one dimension in a potential $V(x)$ such that
        \begin{equation}
            \begin{aligned}
                V(x) &\to\infty && \text{for }|x|\geq a/2\\
                V(x) &= 0       && \text{for }-a/2<x<0\\
                V(x) &= V_0     && \text{for }0\leq x<a/2
            \end{aligned}
        \end{equation}
        Considering that the wave function and its derivative are continuous at $x=0$, and that the wave function vanishes at $x=\pm a/2$, try to find the equation that gives the possible energy states assuming $E_n>V_0$.\par
        \emph{Hint}: There are different combinations of sine and cosine functions for positive and negative values of $x$.
        \item Consider now an energy $E_n<V_0$. What is the form of the solution at positive values of $x$?\par
        \emph{Hint}: The solutions at $x>0$ are no longer given in terms of sine and cosine functions. They can now be written in terms of exponential functions. You can obtain the new solutions by using the same solutions as for $E_n>V_0$, and the known relations between trigonometric and hyperbolic sine and cosine functions, namely $\sin(ix)=i\sinh(x)$ and $\cos(ix)=\cosh(x)$.\par
        In classical mechanics, a particle cannot enter a region where the potential energy is larger than the energy of the particle, since it would imply that the kinetic energy is negative. Is this still true in quantum mechanics? In other words, does the probability of finding the particle at positive values of $x$ vanish?
        \item What happens when $V_0\to\infty$? In principle, one obtains in this limit an infinite square well of length $a/2$. Hence, the energy eigenstates should change such that
        \begin{equation}
            \sin(\frac{k_na}{2}) = 0
        \end{equation}
        with
        \begin{equation}
            \frac{\hbar^2k_n^2}{2m} = E_n
        \end{equation}
        Is this true? What happens to the wave function at positive values of $x$?
    \end{enumerate}
    \item Although planar waves are not well normalized, one can use them to demonstrate the tunneling phenomenon. Note that a well-normalized solution would be a wave packet of these planar waves, namely
    \begin{equation}
        \psi(x,0) = \frac{1}{\sqrt{2\pi}}\int\dd{k}\Phi(k)\e[ikx]
    \end{equation}
    Consider a stationary solution of a free particle with finite energy $E$ and forget for the time being the question of normalization. The momentum of the particle will be given by $|\hbar k|=\sqrt{E/2m}$ and
    \begin{equation}\label{eqn:PS2E10}
        \psi(x,t) = A\e[i(kx-\omega t)]+B\e[i(-kx-\omega t)]
    \end{equation}
    with $\hbar\omega=E$.\par
    Observe that there are two solutions, one with positive (incoming wave) and the other with negative (reflected wave) values of the momentum in the $x$-direction. In the previous problems, they combined to give sine and cosine functions due to the boundary conditions. Now, imagine that we consider that this particle goes through a potential
    \begin{equation}
        \begin{aligned}
            V(x) &= 0   && \text{for }|x|>a/2\\
            V(x) &= V_0 && \text{for }|x|\leq a/2
        \end{aligned}
    \end{equation}
    Since we want to consider the transmission of the particle at the right of the potential barrier, consider the case in which at the right of this finite potential barrier, the wave function is described by a particle moving freely in the positive direction of $x$ via
    \begin{equation}
        \psi(x,t) = C\e[i(kx-\omega t)]
    \end{equation}
    for $x>a/2$ while for $x<-a/2$, we have the function given in Eq. \ref{eqn:PS2E10}. The ratio $|B/A|^2$ may be considered as the reflection probability, while $|C/A|^2$ is the transmission probability.\par
    The solutions for $-a/2\leq x\leq a/2$, instead, are given by
    \begin{equation}
        \psi(x,t) = F\e[i(k_2x-\omega t)]+G\e[i(-k_2x-\omega t)]
    \end{equation}
    with $k_2=\sqrt{2m(E-V_0)}/\hbar$. Observe that $\omega$ is unchanged.\par
    Using continuity of the wave function and its derivatives at $x=\pm a/2$, calculate the reflection and transmission probabilities. Consider both the case of $E>V_0$ and $E<V_0$, and try to interpret the results. What happens when $V_0\to\infty$?
\end{enumerate}




\end{document}