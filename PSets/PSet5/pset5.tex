\documentclass[../psets.tex]{subfiles}

\pagestyle{main}
\renewcommand{\leftmark}{Problem Set \thesection}
\setcounter{section}{4}

\begin{document}




\section{Three Dimensional Mathematical Tools}
\begin{enumerate}
    \item \marginnote{2/17:}Consider a three-dimensional box, such that the potential is given by
    \begin{equation}
        V(x,y,z) = V_1\theta(|x|-a)+V_2\theta(|y|-b)+V_3\theta(|z|-c)
    \end{equation}
    where the function $\theta$ is such that
    \begin{equation}
        \theta(u) =
        \begin{cases}
            1 & u\geq 0\\
            0 & u<0
        \end{cases}
    \end{equation}
    This means that the potential is zero inside the box that extends from $x=-a$ to $x=a$ in the $x$-direction, $y=-b$ to $y=b$ in the $y$-direction, and $z=-c$ to $z=c$ in the $z$-direction and increases in steps otherwise.
    \begin{enumerate}
        \item Analyze first the case wherein all of $V_1,V_2,V_3\to\infty$. This is the infinite square well in three dimensions. What is the general solution for $\psi(x,y,z)$? \emph{Hint}: Use the method of separation of variables and write $\psi(x,y,z)=X(x)Y(y)Z(z)$.
        \item Give an expression for the total energy of the system in terms of the energies associated with the propagation in the three space directions.
        \item What happens when one of the three $V_i$'s becomes finite? What would be the possible solutions in such a case?
        \item What happens when the three $V_i$'s become finite?
    \end{enumerate}
    \item In the presence of a central force, the potential depends only on the radial distance and not on the direction, i.e.,
    \begin{equation}
        V(\vec{r}) = V(\sqrt{x^2+y^2+z^2})
    \end{equation}
    \begin{enumerate}
        \item Show that in such a potential, the momentum is not conserved (that is, its mean value is not independent of time), but the angular momentum $\vec{L}=\vec{r}\times\vec{p}$ is conserved. \emph{Hint}: Use commutation relations with $\hat{H}$. Test this for each component of $\vec{L}$ separately.
        \item What happens if the system has a translational invariance in the $z$-direction and
        \begin{equation}
            V(\vec{r}) = V(\sqrt{x^2+y^2})
        \end{equation}
        for all $z$? Is any component of the momentum or angular momentum preserved?
        \item For part (b), what would be the mean value $\Exp{\hat{z}}(t)$ of the $z$-coordinate if the mean value $\Exp{\hat{p}_z}$ of the momentum in the $z$-direction is constant at $p_0$ while $\Exp{\hat{z}}=0$ at time $t=0$?
    \end{enumerate}
    \item Imagine I have a potential such that I can find simultaneous eigenstates of $\hat{\vec{L}}{\,}^2$, $\hat{L}_z$, and $\hat{H}$ with respective eigenvalues $\hbar^2\ell(\ell+1)$, $\hbar m$, and $E_{n\ell}$. Suppose that $[\hat{L}_i,\hat{L}_j]=i\hbar\epsilon_{ijk}\hat{L}_k$, $[\hat{\vec{L}}{\,}^2,\hat{L}_i]=[\hat{H},\hat{L}_i]=0$, and $\hat{L}_\pm=\hat{L}_x\pm i\hat{L}_y$.
    \begin{enumerate}
        \item Show that $[\hat{L}_\pm,\hat{L}_z]=\mp\hbar\hat{L}_\pm$.
        \item Show also that this implies that, given an eigenfunction $Y_{\ell m}$ of $\hat{\vec{L}}{\,}^2$ and $\hat{L}_z$,
        \begin{equation}
            \hat{L}_\pm Y_{\ell m} \propto Y_{\ell(m\pm 1)}
        \end{equation}
        \item Show that $\hat{\vec{L}}{\,}^2=\hat{L}_-\hat{L}_++\hbar\hat{L}_z+\hat{L}_z^2$.
        \item With the above information, use the ladder operators $\hat{L}_\pm$ to compute the mean value of $\hat{L}_z$, $\hat{L}_y$, $\hat{L}_x^2$, and $\hat{L}_y^2$ in an eigenstate of $\hat{\vec{L}}{\,}^2$ and $\hat{L}_z$.
        \item Discuss the uncertainty principle for the incompatible variables $\hat{L}_x$ and $\hat{L}_y$. In which way does this differ from the case of $\hat{x}$ and $\hat{p}_x$?
    \end{enumerate}
\end{enumerate}




\end{document}