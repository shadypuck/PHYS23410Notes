\documentclass[../notes.tex]{subfiles}

\pagestyle{main}
\renewcommand{\chaptermark}[1]{\markboth{\chaptername\ \thechapter\ (#1)}{}}
\setcounter{chapter}{9}

\begin{document}




\chapter{Finals Week}
\section{Final Exam Review}
\begin{itemize}
    \item \marginnote{3/4:}Final.
    \begin{itemize}
        \item Thursday, 5:30pm-7:30pm.
        \begin{itemize}
            \item May release before 5:30, but it should be returned at 7:30 \emph{sharp} so they can grade in the evening.
        \end{itemize}
        \item Doable in 2 hours (he believes).
        \item Open book.
        \item Largely conceptual; minimal calculation required.
        \begin{itemize}
            \item No tricky questions.
        \end{itemize}
        \item Heavily based on problem sets and the midterm.
        \begin{itemize}
            \item Simply a test of what we've learned in the course.
        \end{itemize}
        \item For those who have poor PSet grades, the final will acquire more relevance.
        \begin{itemize}
            \item He will try to have a rule along the lines of "If you do better in the final than the problem sets, then we'll boost you're grade by some amount."
        \end{itemize}
        \item For the multiple choice questions, be sure to read \emph{all} of the answer choices before selecting one because they're looking for the \emph{best} answer even when multiple may be partially correct.
        \item Will the final focus more on the topics covered since the midterm?
    \end{itemize}
    \item There are notes for a review class posted on Canvas.
    \begin{itemize}
        \item We'll go for 40-45 minutes today.
    \end{itemize}
    \item We now begin the review.
    \item In this course, we've looked at a particle that satisfies the equation
    \begin{equation*}
        \left[ -\frac{\hbar^2}{2M}\vec{\nabla}^2+V(\vec{r},t) \right]\psi = i\hbar\pdv{\psi}{t}
    \end{equation*}
    \begin{itemize}
        \item Specifically, we've focused on time-independent potentials
        \begin{equation*}
            V(\vec{r},t) = V(\vec{r})
        \end{equation*}
    \end{itemize}
    \item For time-independent potentials, we do not have the possibility that the expected value of the energy is zero.
    \begin{itemize}
        \item Additionally, energy is conserved.
        \begin{equation*}
            \dv{t}(\ev{\hat{H}}{\psi}) = \frac{i}{\hbar}\ev{[\hat{H},\hat{H}]}{\psi}+0 = 0
        \end{equation*}
    \end{itemize}
    \item For generic Hermitian operators, if we have the following, then we know that the corresponding quantity is conserved.
    \begin{equation*}
        \dv{t}(\ev{\hat{O}}{\psi}) = \frac{i}{\hbar}\underbrace{\ev{[\hat{H},\hat{O}]}{\psi}}_0+\underbrace{\ev{\pdv{\hat{O}}{t}}{\psi}}_0
        = 0
    \end{equation*}
    \item There are certain operators to be aware of.
    \begin{itemize}
        \item The Hamiltonian operator gives energy.
        \begin{equation*}
            \hat{H}\ket{\psi_n} = E_n\ket{\psi_n}
        \end{equation*}
        \item Other operators: Momentum ($\hat{\vec{p}}$), position ($\hat{\vec{r}}$), and potential ($\hat{V}(\vec{r})$).
        \item Every operator can be expressed as a function $F(\hat{\vec{p}},\hat{\vec{r}})$ of the momentum and position operators.
        \begin{itemize}
            \item Thus, you can also determine if an operator is conserved using its decomposition in terms of position and momentum operators:
            \begin{equation*}
                \dv{t}(\ev{F(\hat{\vec{p}},\hat{\vec{r}})}{\psi_n}) = 0
            \end{equation*}
        \end{itemize}
    \end{itemize}
    \item A generic wave function can be expressed as a linear combination of a basis of eigenfunctions.
    \begin{equation*}
        \ket{\psi}(t) = \sum_nc_n\ket{\psi_n}\e[-iE_nt/\hbar]
    \end{equation*}
    \item Using such a decomposition, we can calculate the expected energy as follows.
    \begin{equation*}
        \ev{\hat{H}}{\psi} = \sum_n|c_n|^2E_n
    \end{equation*}
    \begin{itemize}
        \item Note that each $|c_n|^2$ is the probability of finding $E_n$ when you take a measurement of the particle.
    \end{itemize}
    \item Another important decomposition is that of the expected position.
    \begin{equation*}
        \ev{\hat{\vec{r}}}{\psi} = \sum_{m,n}c_m^*c_n\mel{\psi_m}{\hat{\vec{r}}}{\psi_n}\e[i(E_m-E_n)t/\hbar]
    \end{equation*}
    \item The quantity $|\psi(\vec{r},t)|^2$ is the probability density of the particle.
    \begin{itemize}
        \item The probability that the particle will be \emph{somewhere} in space is certain.
        \begin{equation*}
            \int\dd^3\vec{r}\ |\psi|^2 = 1
        \end{equation*}
    \end{itemize}
    \item If we have a potential, the energy of the particle should always be greater than the minimum of the potential.
    \begin{itemize}
        \item Additionally, quantum particles can penetrate somewhat into regions where the potential is greater than their energy.
        \item Quantum particles can also \textbf{tunnel} through finitely long regions of high potential.
    \end{itemize}
    \item In this course, we spent a lot of time trying to find the bound states of energy.
    \item The only normalizable states are those for which the energy is quantized.
    \item A very important case into which we looked is the harmonic oscillator.
    \begin{equation*}
        V(x) = \frac{m\omega^2x^2}{2}
    \end{equation*}
    \begin{itemize}
        \item The energy eigenvalues are
        \begin{equation*}
            E_n = \hbar\omega\left( n+\frac{1}{2} \right)
        \end{equation*}
        \item The uncertainty principle is what requires that $1/2$ term, since it implies we must have
        \begin{equation*}
            \sigma_x\sigma_p \geq \frac{\hbar}{2}
        \end{equation*}
        because
        \begin{equation*}
            \sigma_A\cdot\sigma_B \geq \frac{1}{2}|\ev{[\hat{A},\hat{B}]}{\psi}|
        \end{equation*}
        and
        \begin{equation*}
            |[\hat{x},\hat{p}]| = \hbar
        \end{equation*}
        \item Diving deeper into the harmonic oscillator, we defined the raising and lowering operators
        \begin{equation*}
            a_\pm = \frac{1}{\sqrt{2\hbar M\omega}}[\mp i\hat{p}+M\omega\hat{x}]
        \end{equation*}
        \begin{itemize}
            \item These operators have the properties that
            \begin{align*}
                a_+\ket{n} &\propto \ket{n+1}&
                a_-\ket{n} &\propto \ket{n-1}
            \end{align*}
            \item We can also use these to prove that
            \begin{align*}
                \ev{\hat{x}}{n} &= 0&
                \ev{\hat{p}}{n} &= 0
            \end{align*}
        \end{itemize}
        \item The eigenfunctions of the harmonic oscillator form an orthonormal basis of the function space.
        \begin{equation*}
            \braket{n}{m} = \int\dd{x}\psi_n^*\psi_m
            = \delta_{nm}
        \end{equation*}
        \item Energy is shared between the kinetic and potential.
        \begin{align*}
            \ev**{\frac{M\omega\hat{x}^2}{2}}{n} &= \frac{\hbar\omega}{2}\left( n+\frac{1}{2} \right)&
            \ev**{\frac{\hat{p}^2}{2M}}{n} &= \frac{\hbar\omega}{2}\left( n+\frac{1}{2} \right)
        \end{align*}
    \end{itemize}
    \item This is what we needed to know for the midterm.
    \item After that, we went into three dimensions.
    \item In particular, we looked at central potentials, which have the form
    \begin{equation*}
        V(\vec{r}) = V(r)
    \end{equation*}
    \item We introduced the angular momentum operators $\hat{L}_z,\hat{\vec{L}}{\,}^2$.
    \item We look at these operators in spherical coordinates $(r,\theta,\phi)$.
    \item For the component operators, we have the commutativity relation
    \begin{equation*}
        [\hat{L}_i,\hat{L}_j] = i\hbar\epsilon_{ijk}\hat{L}_k
    \end{equation*}
    \item The $z$-direction angular momentum operator has the special property that
    \begin{equation*}
        \hat{L}_z = -i\hbar\pdv{\phi}
    \end{equation*}
    \item The two main operators introduced above satisfy the eigenvalue equations
    \begin{align*}
        \hat{L}_zY_{\ell m} &= \hbar mY_{\ell m}&
        \hat{\vec{L}}{\,}^2Y_{\ell m} = \hbar^2\ell(\ell+1)Y_{\ell m}
    \end{align*}
    \begin{itemize}
        \item The solutions are of the form
        \begin{equation*}
            Y_{\ell m}(\theta,\phi) = P_{\ell,m}(\theta)\e[im\phi]
        \end{equation*}
        where $m$ takes on the $2\ell+1$ values from $-\ell\leq m\leq\ell$.
    \end{itemize}
    \item The overall wave function has the angular components from above and also a radial component so that
    \begin{equation*}
        \psi_{n\ell m}(\vec{r}) = R_{n\ell}(r)Y_{\ell m}(\theta,\phi)
    \end{equation*}
    \item If we define
    \begin{equation*}
        U_{n\ell}(r) = rR_{n\ell}(r)
    \end{equation*}
    then we obtain the effective 1D Schr\"{o}dinger equation
    \begin{equation*}
        \bigg[ -\frac{\hbar^2}{2M}\dv[2]{r}+\underbrace{\frac{\hbar^2\ell(\ell+1)}{2Mr^2}+V(r)}_{V_\text{eff}(r)} \bigg]U_{n\ell} = E_{n\ell}U_{n\ell}
    \end{equation*}
    \item We then used these techniques to address the harmonic oscillator in three dimensions.
    \begin{itemize}
        \item We found the energies to be
        \begin{equation*}
            E_{n\ell} = \hbar\omega\bigg( \underbrace{N+\ell}_n+\frac{3}{2} \bigg)
            = \hbar\omega\bigg( \underbrace{n_1+n_2+n_3}_n+\frac{3}{2} \bigg)
        \end{equation*}
        \begin{itemize}
            \item $N$ is the degree of the polynomial.
            \item $N$ is even.
        \end{itemize}
        \item The bound states are a battle between the two terms in the effective potential energy, as summarized in Figure \ref{fig:sphrSymHOV}.
    \end{itemize}
    \item We also used 3D central potential techniques to tackle the hydrogen atom.
    \begin{itemize}
        \item We found the energies to be
        \begin{equation*}
            E_{n\ell} = -\frac{\Ry}{\underbrace{(N+\ell+1)}_n{}^2}
        \end{equation*}
        \item There are $n^2$ solutions for each each $n$, which we determine by summing the $2\ell+1$ degeneracy over $\ell=0,\dots,n-1$.
        \item Note that
        \begin{equation*}
            \Ry = \SI{13.6}{\electronvolt}
        \end{equation*}
        is a very important number.
    \end{itemize}
    \item Brief review of emitting electromagnetic radiation.
    \item Last thing: Spin.
    \begin{itemize}
        \item Every particle has it; we don't know why.
    \end{itemize}
    \item The quantity
    \begin{equation*}
        \hat{\vec{S}} = \hat{S}_x\hat{x}+\hat{S}_y\hat{y}+\hat{S}_z\hat{z}
    \end{equation*}
    is an angular momentum that has nothing to do with angular direction. It has nothing to do with any direction in spacetime; rather, it is an \emph{intrinsic} property of the particle.
    \item We have the commutation relation
    \begin{equation*}
        [\hat{S}_x,\hat{S}_y] = i\hbar\hat{S}_z
    \end{equation*}
    \item We can introduce ladder operators
    \begin{equation*}
        \hat{S}_\pm = \hat{S}_x\pm i\hat{S}_y
    \end{equation*}
    \item As in real angular momentum, we have analogous eigenvalue expressions
    \begin{align*}
        \hat{\vec{S}}{\,}^2\ket{s,m_s} &= \hbar^2s(s+1)\ket{s,m_s}\\
        \hat{S}_z\ket{s,m_s} &= \hbar m_s\ket{s,m_s}
    \end{align*}
    \item $2s+1$ states implies that $s=0,1/2,1,3/2,2,\dots$ can take on half-integer values.
    \begin{itemize}
        \item Almost all elementary particles have spin $1/2$ (protons, neutrons, the quarks that make them up, electrons, leptons).
        \item Force carriers (photon, gluon) have spin 1.
        \item The graviton (if it exists; "it exists") has spin 2.
        \item Higgs has spin 0.
    \end{itemize}
    \item In the hydrogen atom, there are $2n^2$ states for each $n$ given by
    \begin{equation*}
        \ket{n,\ell,m,\tfrac{1}{2},\pm\tfrac{1}{2}}
    \end{equation*}
    \begin{itemize}
        \item This is the 2 $s$-orbital positions, 8 $s+p$-orbital positions, and on and on.
    \end{itemize}
    \item We can modify the total degeneracy of the spin with a magnetic field. Use the Hamiltonian
    \begin{equation*}
        -\hat{\vec{\mu}}\cdot\vec{B} = -\gamma\cdot\hat{\vec{S}}\cdot\vec{B}
    \end{equation*}
    \begin{itemize}
        \item If $\vec{B}=B\hat{z}$, then the above also equals $-\gamma B\hat{S}_z$.
    \end{itemize}
    \item In a generic state, we can introduce spinors
    \begin{equation*}
        \chi =
        \begin{pmatrix}
            \chi_+\\
            \chi_-\\
        \end{pmatrix}
    \end{equation*}
    \item The probability of finding the particle with spin up or spin down in the $z$-direction must be certain.
    \begin{equation*}
        |\chi_+|^2+|\chi_-|^2 = 1
    \end{equation*}
    \begin{itemize}
        \item $|\chi_+|^2$ is the probability of spin up.
        \item $|\chi_-|^2$ is the probability of spin down.
    \end{itemize}
    \item In the magnetic field case, we have to solve the Schr\"{o}dinger equation
    \begin{equation*}
        -\gamma B\hat{S}_z
        \begin{pmatrix}
            \chi_+\\
            \chi_-\\
        \end{pmatrix}
        = i\hbar\pdv{t}
        \begin{pmatrix}
            \chi_+\\
            \chi_-\\
        \end{pmatrix}
    \end{equation*}
    \item The three spin components can be represented by the matrices
    \begin{align*}
        \hat{S}_x &= \frac{\hbar}{2}
        \begin{pmatrix}
            0 & 1\\
            1 & 0\\
        \end{pmatrix}&
        \hat{S}_y &= \frac{\hbar}{2}
        \begin{pmatrix}
            0 & -i\\
            i & 0\\
        \end{pmatrix}&
        \hat{S}_z &= \frac{\hbar}{2}
        \begin{pmatrix}
            1 & 0\\
            0 & -1\\
        \end{pmatrix}
    \end{align*}
    \begin{itemize}
        \item The matrices are called the Pauli matrices $\sigma_i$.
        \item These matrices have the properties that
        \begin{align*}
            \sigma_i^2 &= I&
            \hat{S}_i &= \frac{\hbar}{2}\sigma_i&
            \hat{S}_i^2 &= \frac{\hbar^2}{4}I
        \end{align*}
    \end{itemize}
    \item Now suppose we let
    \begin{align*}
        |\chi_+|(0) &= \cos(\frac{\theta_s}{2})&
        |\chi_-|(0) &= \sin(\frac{\theta_s}{2})
    \end{align*}
    \begin{itemize}
        \item It then follows that
        \begin{align*}
            \ev{\hat{S}_z}{\chi} &= \frac{\hbar}{2}\cos(\theta_s)\\
            \ev{\hat{S}_x}{\chi} &= \frac{\hbar}{2}\sin(\theta_s)\cos(\gamma Bt+\phi_+-\phi_-)
        \end{align*}
        \item Then the probability of spin up or spin down is
        \begin{equation*}
            P_\pm = \frac{1}{2}[1\pm\sin(\theta_s)\cos(\gamma Bt+\phi_+-\phi_-)]
        \end{equation*}
    \end{itemize}
\end{itemize}




\end{document}