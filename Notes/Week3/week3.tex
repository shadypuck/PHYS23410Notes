\documentclass[../notes.tex]{subfiles}

\pagestyle{main}
\renewcommand{\chaptermark}[1]{\markboth{\chaptername\ \thechapter\ (#1)}{}}
\setcounter{chapter}{2}

\begin{document}




\chapter{Time-Independent Problems in One-Dimensional Systems}
\section{Infinite Well Motion}
\begin{itemize}
    \item \marginnote{1/17:}We begin today by building up to the uncertainty principle another, more general way.
    \item Recall that what we are aiming for is
    \begin{equation*}
        \Delta p_x\Delta x \geq \frac{\hbar}{2}
    \end{equation*}
    where $\Delta p_x,\Delta x$ are the uncertainties in the determination of the momentum and position, respectively:
    \begin{align*}
        (\Delta p_x)^2 &= \Exp{(\hat{\vec{p}}_x-\Exp{\hat{\vec{p}}_x})^2} = \Exp{\hat{\vec{p}}_x{}^2}-\Exp{\hat{\vec{p}}_x}^2&
        (\Delta x)^2 &= \Exp{\hat{\vec{x}}{\,}^2}-\Exp{\hat{\vec{x}}}^2
    \end{align*}
    \item Example of the uncertainty principle: For a plane wave, we know the momentum but not the position. That is, $\Delta x\to\infty$ and $\Delta p_x\to 0$.
    \item More generally, for a \textbf{wave packet}, we know only approximately the position and momentum.
    \item \textbf{Wave packet}: A continuous sum of waves of different frequencies. \emph{Given by}
    \begin{equation*}
        \psi(x,t) = \frac{1}{\sqrt{2\pi}}\int_{-\infty}^\infty\phi(k)\e[i(kx-\omega(k)t)]\dd{k}
    \end{equation*}
    \begin{figure}[h!]
        \centering
        \begin{subfigure}[b]{0.4\linewidth}
            \centering
            \begin{tikzpicture}
                \path (0,0) -- (0,-1);
                \draw (-2,0) -- (2,0);
                \draw [blx,thick] (-0.8,0.1)
                    to[out=0,in=180,in looseness=0.7] (-0.3,0.5)
                    to[out=0,in=180,out looseness=0.7] (0.2,0.1)
                ;
    
                \draw [-latex] (-0.3,0.35) -- ++(0.6,0);
            \end{tikzpicture}
            \caption{A wave packet.}
            \label{fig:wavePacketa}
        \end{subfigure}
        \begin{subfigure}[b]{0.4\linewidth}
            \centering
            \begin{tikzpicture}
                \footnotesize
                \draw (-2.5,0) -- (2.5,0);
                \draw [rex,thick] plot[domain=-2:2,samples=100,smooth] (\x,{e^(-\x*\x)*sin(3*pi*\x r)});
                \draw [blx,thick]
                    plot[domain=-2:2,smooth] (\x,{e^(-\x*\x)})
                    plot[domain=-2:2,smooth] (\x,{-e^(-\x*\x)})
                ;
    
                \draw [-latex] (0.5,{e^(-0.25)}) -- ++(0.6,0) node[right]{$v_g$};
                \draw [-latex] (0.27,{e^(-0.27*0.27)*sin(3*pi*0.27 r)}) -- ++(0.3,0) node[below,xshift=-2pt]{$v_p$};
            \end{tikzpicture}
            \caption{Group and phase velocity.}
            \label{fig:wavePacketb}
        \end{subfigure}
        \caption{Wave packets.}
        \label{fig:wavePacket}
    \end{figure}
    \begin{itemize}
        \item Note that the above formula only applies to the one dimensional case.
    \end{itemize}
    \item Let's investigate the case of a wave packet of free particles.
    \begin{itemize}
        \item In this case,
        \begin{equation*}
            \omega(k) = \frac{\hbar k^2}{2m}
        \end{equation*}
        \begin{itemize}
            \item This is derived from
            \begin{equation*}
                \hbar\omega = E = \frac{p^2}{2m} = \frac{(\hbar k)^2}{2m}
            \end{equation*}
            by cancelling an $\hbar$ from both sides.
        \end{itemize}
        \item Let's assume that $\phi(k)$ is a narrowly peaked function around a certain value $k_0$.
        \item Then we can expand
        \begin{align*}
            \omega(k) &= \omega(k_0)+\eval{\dv{\omega}{k}}_{k=k_0}(k-k_0)+\cdots\\
            &= \omega(k_0)+\eval{\frac{\hbar k}{m}}_{k=k_0}(k-k_0)+\cdots\\
            &= \omega(k_0)+\underbrace{\frac{\hbar k_0}{m}}_{\omega_0'}(k-k_0)+\cdots
        \end{align*}
        \item Define $s:=k-k_0$.
        \item Then $k=k_0+s$ and $\dd{k}=\dd{s}$, so
        \begin{align*}
            \psi(x,t) &= \frac{1}{\sqrt{2\pi}}\int_{-\infty}^\infty\phi(k_0+s)\e[i((k_0+s)x-(\omega(k_0)+\omega_0's)t)]\dd{s}\\
            &= \frac{1}{\sqrt{2\pi}}\e[i(k_0x-\omega(k_0)t)]\int_{-\infty}^\infty\phi(k_0+s)\e[is(x-\omega_0't)]\dd{s}
        \end{align*}
        \item It follows that
        \begin{equation*}
            |\psi(x,t)|^2 = \frac{1}{2\pi}\left| \int_{-\infty}^\infty\phi(k_0+s)\e[is(x-\omega_0't)]\dd{s} \right|^2 = f(x-\omega_0't)
        \end{equation*}
        \begin{itemize}
            \item In words, the probability density is a function of $x-\omega_0't$, so the packet moves with \textbf{group velocity} $\omega_0'=\hbar k_0/m=p_0/m$.
        \end{itemize}
        \item Implication: The wave packet moves with a velocity that is equal ot the classical velocity
        \begin{equation*}
            \eval{\dv{\omega}{k}}_{k=k_0} = \frac{p_0}{m}
        \end{equation*}
    \end{itemize}
    \item \textbf{Group velocity}: A measure of the velocity of a wave packet. \emph{Denoted by} $\bm{v_g}$, $\bm{v_\textbf{group}}$. \emph{Given by}
    \begin{equation*}
        v_\text{group} = \eval{\dv{\omega}{k}}_{k=k_0}
    \end{equation*}
    \item \textbf{Phase velocity}: A measure of the velocity of the ripples. \emph{Denoted by} $\bm{v_p}$, $\bm{v_\textbf{phase}}$. \emph{Given by}
    \begin{equation*}
        v_\text{phase} = \frac{\omega(k_0)}{k_0} = \frac{\hbar k_0}{2m} = \frac{v_\text{group}}{2}
    \end{equation*}
    \item Explicit example of a wave packet: A \textbf{Gaussian wave packet}.
    \item \textbf{Gaussian wave packet}: A one-dimensional wave packet of the following form. \emph{Given by}
    \begin{equation*}
        \psi_0(x,t) = \left( \frac{2}{\pi\sigma^2} \right)^{1/4}\exp[-\frac{(x-v_gt)^2}{\sigma^2}]\e[i(k_0x-v_pt)]
    \end{equation*}
    \begin{itemize}
        \item This means that we must have used the following definition of $\phi(k)$ in the original definition.
        \begin{equation*}
            \phi(k) = \left( \frac{\sigma^2}{2\pi} \right)^{1/4}\exp[-\frac{\sigma^2(k-k_0)^2}{4}]
        \end{equation*}
    \end{itemize}
    \item Uncertainty analysis of a Gaussian wave packet.
    \begin{itemize}
        \item The uncertainties $\Delta x$ and $\Delta k$ are associated with the widths of the Gaussians, as one can determine by computing. Indeed, at $t=0$,
        \begin{align*}
            \Exp{\hat{\vec{x}}} &= 0&
            \Exp{\hat{\vec{x}}{\,}^2} &= (\Delta x)^2&
            \Exp{(k-k_0)^2} &= (\Delta k)^2
        \end{align*}
        \begin{itemize}
            \item Indeed, since $\Exp{k}=k_0$, we know that $\Exp{(k-k_0)^2}=\Exp{k^2}-k_0^2$.
        \end{itemize}
        \item For Gaussians, normalized as $\int|\psi|^2=1$, we obtain
        \begin{equation*}
            \left( \frac{1}{\pi\sigma^2} \right)^{1/2}\int_{-\infty}^\infty u^2\exp(-\frac{u^2}{\sigma^2})\dd{u}
            = (\Delta u)^2
            = \frac{\sigma^2}{2}
        \end{equation*}
        \begin{itemize}
            \item How do we get this??
        \end{itemize}
        \item It follows that the value of $\Delta u$ coincides well with the departure from the central value for which the exponential in $|\psi|^2$ or $|\phi|^2$ is $\e[-1/2]$.
        \item Altogether, we get
        \begin{align*}
            \Delta x &= \frac{\sigma}{2}&
            \Delta k &= \frac{1}{\sigma}
        \end{align*}
        so
        \begin{align*}
            \Delta x\Delta k &= \frac{1}{2}\\
            \Delta x\Delta p_x &= \frac{\hbar}{2}
        \end{align*}
        for a Gaussian wave packet.
        \item Implication: The Gaussian function minimizes the product of the position and momentum uncertainties!
    \end{itemize}
    \item We now move onto discussing the \textbf{infinite square well} potential, a one-dimensional time-independent potential for which we can solve the Schr\"{o}dinger equation exactly.
    \begin{figure}[h!]
        \centering
        \begin{tikzpicture}[
            every node/.style={black,text height=1.5ex,text depth=0.25ex}
        ]
            \footnotesize
            \shade [left color=white,right color=grx!30] (-0.3,0) rectangle (0,2);
            \shade [left color=grx!30,right color=white] (2,0) rectangle (2.3,2);
            \draw (-2,0) -- (4,0);
            \draw [grx,thick] (0,2) node[above left]{$V(x)$}
                -- (0,0) node[below]{$0$}
                -- (2,0) node[below]{$a$}
                -- (2,2)
            ;
            \node at (-1,1) {$V\to\infty$};
            \node at (3,1) {$V\to\infty$};
        \end{tikzpicture}
        \caption{Infinite square well.}
        \label{fig:infiniteSquareWell}
    \end{figure}
    \item \textbf{Infinite square well}: The potential energy function that vanishes for $0<x<a$ and tends to infinity for $x\leq 0$ and $x\geq a$. \emph{Given by}
    \begin{equation*}
        V(x) =
        \begin{cases}
            0 & 0<x<a\\
            \infty & \text{otherwise}
        \end{cases}
    \end{equation*}
    \item We would like to obtain energy eigenstates for this potential. That is, we seek eigenvalues and eigenfunctions for
    \begin{equation*}
        \left[ -\frac{\hbar^2}{2m}\dv[2]{x}+V(x) \right]\psi(x) = E\psi(x)
    \end{equation*}
    \begin{itemize}
        \item Any such eigenstate $\psi$ will have $\psi(x)=0$ in the region of space where $V\to\infty$.
        \item Hence, the Schr\"{o}dinger equation reduces to the boundary-value problem
        \begin{align*}
            -\frac{\hbar^2}{2m}\dv[2]{\psi}{x} &= E\psi&
            \psi(0) &= \psi(a) = 0
        \end{align*}
        \item The above ODE may be expressed in the following equivalent form
        \begin{equation*}
            \dv[2]{\psi}{x} = -\left( \frac{2mE}{\hbar^2} \right)\psi
        \end{equation*}
        \item Observe that this ODE is of the same form as the classical harmonic oscillator equation $\dv*[2]{x}{t}=-(k/m)x$. Thus, it admits a similar set of solutions:
        \begin{align*}
            \psi_n(x) &= C\sin(\sqrt{\frac{2mE_n}{\hbar^2}}x)&
            \sqrt{\frac{2mE_n}{\hbar^2}}a &= n\pi,\ n=1,2,\dots
        \end{align*}
        \item It follows that
        \begin{equation*}
            E_n = \frac{\hbar^2n^2\pi^2}{2ma^2}
        \end{equation*}
        \item The coefficient $C$ can be fixed via the normalization requirement, as follows.
        \begin{align*}
            1 &= \int_0^a|\psi_n(x)|^2\dd{x}\\
            &= C^2\int_0^a\sin^2\left( \frac{\pi nx}{a} \right)\dd{x}\\
            &= C^2\int_0^a\frac{1-\cos(\frac{2\pi nx}{a})}{2}\dd{x}\\
            &= \frac{C^2}{2}\left[ \int_0^a\dd{x}-\int_0^a\cos(\frac{2\pi nx}{a})\dd{x} \right]\\
            &= \frac{C^2}{2}\bigg[ a-\underbrace{\eval{\frac{a}{2n\pi}\sin(\frac{2\pi nx}{a})}_0^a}_0 \bigg]\\
            &= \frac{aC^2}{2}\\
            C &= \sqrt{\frac{2}{a}}
        \end{align*}
        \item Therefore, the complete eigenfunctions and eigenvalues are
        \begin{align*}
            \psi_n(x) &= \sqrt{\frac{2}{a}}\sin(\frac{\pi nx}{a})&
            E_n &= \frac{\hbar^2n^2\pi^2}{2ma^2}
        \end{align*}
    \end{itemize}
    \item A general solution is therefore given by the following, where $\psi_n,E_n$ are defined as above.
    \begin{equation*}
        \psi(x,t) = \sum_nc_n\psi_n(x)\e[-iE_nt/\hbar]
    \end{equation*}
    \item The probability density of the infinite square well potential is time-independent.
    \begin{proof}
        Observe that given any individual eigenstate of energy
        \begin{equation*}
            \psi_n(x,t) = \psi_n(x)\e[-iE_nt/\hbar]
        \end{equation*}
        we have that
        \begin{equation*}
            |\psi_n(x,t)|^2 = |\psi_n(x)|^2
            = \frac{2}{a}\sin^2\left( \frac{\pi nx}{a} \right)
        \end{equation*}
    \end{proof}
    \item Let's investigate the form of the probability density for a few $n$.
    \begin{figure}[H]
        \centering
        \begin{subfigure}[b]{0.19\linewidth}
            \centering
            \begin{tikzpicture}[
                every node/.style={black,text height=1.5ex,text depth=0.25ex}
            ]
                \footnotesize
                \draw [grx,thick] (0,2)
                    -- (0,0) node[below]{$0$}
                    -- (2,0) node[below]{$a$}
                    -- (2,2) node[above]{$|\psi_n|^2$}
                ;
    
                \draw (0.1,1) -- ++(-0.2,0) node[left]{1};
    
                \draw [rex,thick] plot[domain=0:2,smooth] (\x,{sin(pi*\x/2 r)^2});
            \end{tikzpicture}
            \caption{$n=1$.}
            \label{fig:infWellProba}
        \end{subfigure}
        \begin{subfigure}[b]{0.19\linewidth}
            \centering
            \begin{tikzpicture}[
                every node/.style={black,text height=1.5ex,text depth=0.25ex}
            ]
                \footnotesize
                \draw [grx,thick] (0,2)
                    -- (0,0) node[below]{$0$}
                    -- (2,0) node[below]{$a$}
                    -- (2,2) node[above]{$|\psi_n|^2$}
                ;
    
                \draw (0.1,1) -- ++(-0.2,0) node[left]{1};
    
                \draw [rex,thick] plot[domain=0:2,samples=50,smooth] (\x,{sin(2*pi*\x/2 r)^2});
            \end{tikzpicture}
            \caption{$n=2$.}
            \label{fig:infWellProbb}
        \end{subfigure}
        \begin{subfigure}[b]{0.19\linewidth}
            \centering
            \begin{tikzpicture}[
                every node/.style={black,text height=1.5ex,text depth=0.25ex}
            ]
                \footnotesize
                \draw [grx,thick] (0,2)
                    -- (0,0) node[below]{$0$}
                    -- (2,0) node[below]{$a$}
                    -- (2,2) node[above]{$|\psi_n|^2$}
                ;
    
                \draw (0.1,1) -- ++(-0.2,0) node[left]{1};
    
                \draw [rex,thick] plot[domain=0:2,samples=75,smooth] (\x,{sin(3*pi*\x/2 r)^2});
            \end{tikzpicture}
            \caption{$n=3$.}
            \label{fig:infWellProbc}
        \end{subfigure}
        \begin{subfigure}[b]{0.19\linewidth}
            \centering
            \begin{tikzpicture}[
                every node/.style={black,text height=1.5ex,text depth=0.25ex}
            ]
                \footnotesize
                \draw [grx,thick] (0,2)
                    -- (0,0) node[below]{$0$}
                    -- (2,0) node[below]{$a$}
                    -- (2,2) node[above]{$|\psi_n|^2$}
                ;
    
                \draw (0.1,1) -- ++(-0.2,0) node[left]{1};
    
                \draw [rex,thick] plot[domain=0:2,samples=150,smooth] (\x,{sin(6*pi*\x/2 r)^2});
            \end{tikzpicture}
            \caption{$n=6$.}
            \label{fig:infWellProbd}
        \end{subfigure}
        \begin{subfigure}[b]{0.19\linewidth}
            \centering
            \begin{tikzpicture}[
                every node/.style={black,text height=1.5ex,text depth=0.25ex}
            ]
                \footnotesize
                \draw [grx,thick] (0,2)
                    -- (0,0) node[below]{$0$}
                    -- (2,0) node[below]{$a$}
                    -- (2,2) node[above]{$|\psi_n|^2$}
                ;
    
                \draw (0.1,1) -- ++(-0.2,0) node[left]{1};
    
                \draw [rex,thick] plot[domain=0:2,samples=500,smooth] (\x,{sin(20*pi*\x/2 r)^2});
            \end{tikzpicture}
            \caption{$n\gg1$.}
            \label{fig:infWellProbe}
        \end{subfigure}
        \caption{Infinite square well probability density.}
        \label{fig:infWellProb}
    \end{figure}
    \begin{itemize}
        \item Recall that the average height of a sine wave is half its amplitude. Thus the average probability density is
        \begin{equation*}
            \frac{1}{2}\cdot\frac{2}{a} = \frac{1}{a}
        \end{equation*}
    \end{itemize}
    \item Recovering "motion," in the sense that ${\dv*{t}}(\ev{\hat{\vec{x}}}{\psi})\neq 0$
    \begin{figure}[h!]
        \centering
        \begin{subfigure}[b]{0.25\linewidth}
            \centering
            \begin{tikzpicture}
                \footnotesize
                \draw [grx,thick] (0,2)
                    -- (0,0) node[below,black,text height=1.5ex,text depth=0.25ex]{$0$}
                    -- (2,0) node[below,black,text height=1.5ex,text depth=0.25ex]{$a$}
                    -- (2,2)
                ;
    
                \draw [-latex] (0.2,0.4) node[circle,fill=red,inner sep=1pt]{} -- node[above]{$|v_x|$} ++(0.6,0);
                \draw [-latex] (1.8,0.4) node[circle,fill=red,inner sep=1pt]{} -- node[above]{$-|v_x|$} ++(-0.6,0);
            \end{tikzpicture}
            \caption{Classical.}
            \label{fig:infWellMotiona}
        \end{subfigure}
        \begin{subfigure}[b]{0.25\linewidth}
            \centering
            \begin{tikzpicture}
                \footnotesize
                \draw [grx,thick] (0,2)
                    -- (0,0) node[below,black,text height=1.5ex,text depth=0.25ex]{$0$}
                    -- (2,0) node[below,black,text height=1.5ex,text depth=0.25ex]{$a$}
                    -- (2,2)
                ;
    
                \draw [blx,thick] (0.1,0.1)
                    to[out=0,in=180,in looseness=0.7] (0.4,0.4)
                    to[out=0,in=180,out looseness=0.7] (0.7,0.1)
                ;
                \draw [blx,thick] (1.3,0.1)
                    to[out=0,in=180,in looseness=0.7] (1.6,0.4)
                    to[out=0,in=180,out looseness=0.7] (1.9,0.1)
                ;
    
                \draw [-latex] (0.4,0.25) -- ++(0.4,0) node[above]{$v_g$};
                \draw [-latex] (1.6,0.25) -- ++(-0.4,0) node[above]{$v_g$};
            \end{tikzpicture}
            \caption{Quantum.}
            \label{fig:infWellMotionb}
        \end{subfigure}
        \caption{Infinite square well motion.}
        \label{fig:infWellMotion}
    \end{figure}
    \begin{itemize}
        \item We obtain motion upon superimposing different eigenstate wave functions.
        \item Guiding question: What would happen in the classical case of a particle in such a potential?
        \begin{itemize}
            \item The particle would move first to the right with momentum $|p_x|$, then bounce against the wall at $x=a$ and change its momentum to $-|p_x|$, then bounce against the wall at $x=0$ and change its momentum back to $|p_x|$, and so continue indefinitely.
        \end{itemize}
        \item In quantum mechanics, we can mimic the same behavior by forming a wave packet!
        \item Since the particle moves free of forces between $0<x<a$, one can try to build a Gaussian wave packet, similar to the one we discussed in the free particle case. The difference is that any wave function must vanish at $x=0,a$, so it must be represented not by combinations of free waves $\e[ikx]$ at $t=0$ but by
        \begin{equation*}
            \sin(\frac{\pi nx}{a}) = \frac{1}{2i}(\e[i\pi nx/a]-\e[-i\pi nx/a])
        \end{equation*}
        \item Define
        \begin{equation*}
            k_n = \frac{\sqrt{2mE_n}}{\hbar}
            = \frac{\pi n}{a}
        \end{equation*}
        \item Now, what we want is a Gaussian with width $\Delta x$ for $\Delta x\ll a$.
        \item Recalling the free case $|\psi_0(x,t)|^2=(1/\pi\sigma^2)^{1/2}\e[-(x-v_gt)^2/2\sigma^2]$ with $\phi(k)=k\e[-\sigma^2(k-k_0)^2]$, we would like to try
        \begin{equation*}
            \phi(k_n) \propto \e[-\sigma^2(k_n-k_0)^2] =: c_n
        \end{equation*}
        where $\sigma=\Delta x\ll a$, and hence $1/\sigma\gg 1/a$.
        \item Since
        \begin{equation*}
            k_m-k_n = (m-n)\frac{\pi}{a}
        \end{equation*}
        we will obtain a "continuous" distribution of states with $|k_n-k_0|<1/\sigma$ as well as a suppression of other modes.
        \item Left as an exercise to the student to derive further results about this system.
    \end{itemize}
\end{itemize}



\section{Harmonic Oscillator}
\begin{itemize}
    \item \marginnote{1/19:}The harmonic oscillator is one of the most important problems in physics because we can solve it exactly.
    \item It used to approximate solutions near the bottom of smooth potential wells. It does so via
    \begin{equation*}
        V(x) \approx V(x_0)+\eval{\dv{V}{x}}_{x_0}(x-x_0)+\frac{1}{2}\eval{\dv[2]{V}{x}}_{x_0}(x-x_0)^2+\cdots
    \end{equation*}
    \begin{itemize}
        \item Mathematically, this represents small departures from $x_0$.
        \item Recall that the first derivative goes to zero (because we are at a minimum) and the second one is a constant we can call $k$, yielding
        \begin{equation*}
            V(x) = V(x_0)+\frac{1}{2}k(x-x_0)^2
        \end{equation*}
        \item This is now a potential with which we are familiar from classical mechanics.
    \end{itemize}
    \item Recall what happens in classical mechanics.
    \begin{itemize}
        \item We get an equation with a second derivative of $u=x-x_0$:
        \begin{equation*}
            m\dv[2]{u}{t} = -ku
        \end{equation*}
        \item This problem is solved in classical mechanics by defining $\omega^2:=k/m$ and solving the differential equation for
        \begin{equation*}
            u = A\sin(\omega t)+B\cos(\omega t)
        \end{equation*}
        \item From this general solution, we can get to particular solutions using initial conditions.
        \item For example, if $u(0)=0$, then $B=0$ and
        \begin{equation*}
            u = A\sin(\omega t)
        \end{equation*}
        \item What happens if we multiply the original equation of motion by $v=\dv*{u}{t}$? We get the conservation of energy!
        \begin{align*}
            mv\dv{v}{t} &= -ku\dv{u}{t}\\
            \dv{t}(\frac{mv^2}{2}+\frac{ku^2}{2}) = 0
        \end{align*}
        \item Note that we associate the left term above with $KE=p^2/2m$ and the right term above with $V(u)$.
        \item This gives us
        \begin{align*}
            V(u) &= \frac{ku^2}{2}
                = \frac{k[A\sin(\omega t)]^2}{2}
                = \frac{kA^2}{2}\sin^2(\omega t)\\
            K(u) &= \frac{mv^2}{2}
                = \frac{m}{2}\left( \dv{t}[A\sin(\omega t)] \right)^2
                = \frac{A^2m\omega^2}{2}\cos^2(\omega t)
                = \frac{kA^2}{2}\cos^2(\omega t)
        \end{align*}
        so that
        \begin{equation*}
            V(u)+K(u) = \frac{kA^2}{2}
        \end{equation*}
        for all $u$!
    \end{itemize}
    \item The situation is different in quantum mechanics.
    \begin{itemize}
        \item Here, we must begin from
        \begin{equation*}
            -\frac{\hbar^2}{2m}\dv[2]{x}\psi_n(x)+\frac{kx^2}{2}\psi_n(x) = E_n\psi_n(x)
        \end{equation*}
    \end{itemize}
    \item What do we know, qualitatively, about a solution to this ODE?
    \begin{itemize}
        \item Since $V(x)$ is time-independent, the eigenfunctions will be of the form
        \begin{equation*}
            \psi_n(x,t) = \psi_n(x)\e[-iE_nt/\hbar]
        \end{equation*}
        \item We will be able to normalize these solutions via
        \begin{equation*}
            \int\dd{x}\psi_m^*(x)\psi_n(x) = \delta_{nm}
        \end{equation*}
        \item The general solution will then be a sum of the normalized solutions, like the following.
        \begin{equation*}
            \psi(x,t) = \sum_nc_n\psi_n(x,t)
        \end{equation*}
        \item The normalization condition \emph{here} will then yield
        \begin{equation*}
            \sum_n|c_n|^2 = 1
        \end{equation*}
        \item Lastly, we will be able to calculate expected values, such as
        \begin{equation*}
            \ev{\hat{H}}{\psi} = \sum_m|c_m|^2E_m
        \end{equation*}
    \end{itemize}
    \item We now work toward quantitative solutions $\psi_n(x,t)$, based on insight from the following picture.
    \begin{figure}[h!]
        \centering
        \begin{tikzpicture}[
            every node/.style=black
        ]
            \footnotesize
            \draw (-2,0) -- (2,0) node[right]{$x$};
            \draw [-stealth] (0,-0.5) -- (0,1.5);
    
            \draw [grx,thick] (-1,1.4) parabola bend (0,0) (1,1.4) node[below right]{$V(x)=\frac{kx^2}{2}$};
            \draw [rex,thick] (-2,0)
                to[out=0,in=180,out looseness=1.5,in looseness=0.7] (0,1)
                to[out=0,in=180,out looseness=0.7,in looseness=1.5] (2,0) node[above]{$\psi_n(x)$}
            ;
        \end{tikzpicture}
        \caption{Solving the quantum harmonic oscillator with an asymptotic Schr\"{o}dinger equation.}
        \label{fig:harmOscSol}
    \end{figure}
    \begin{itemize}
        \item Although it may not be immediately obvious how to solve the Schr\"{o}dinger equation in this case, we can see from Figure \ref{fig:harmOscSol} that at large values of $x$, $\psi_n(x)=0$. Thus, for large $x$, we will have
        \begin{equation*}
            -\frac{\hbar^2}{2m}\dv[2]{x}\psi_n(x)+\frac{kx^2}{2}\psi_n(x) = 0
        \end{equation*}
        \begin{itemize}
            \item This tells us the \textbf{asymptotic} behavior of the equation.
        \end{itemize}
        \item We can then algebraically rearrange this equation into the form
        \begin{equation*}
            \dv[2]{x}\psi_n(x)-\frac{m^2\omega^2x^2}{\hbar^2}\psi_n(x) = 0
        \end{equation*}
        \item To solve it, use an ansatz proportional to the following.
        \begin{equation*}
            \psi_n(x) \propto \exp[-\frac{m\omega}{2\hbar}x^2]
        \end{equation*}
        \begin{itemize}
            \item This works because
            \begin{align*}
                \dv{\psi_n}{x} &= -\frac{m\omega x}{\hbar}\exp[-\frac{m\omega}{2\hbar}x^2]\\
                \dv[2]{\psi_n}{x} &= \left( -\frac{m\omega}{\hbar}+\frac{m^2\omega^2x^2}{\hbar^2} \right)\exp[-\frac{m\omega}{2\hbar}x^2]
            \end{align*}
        \end{itemize}
        \item In particular, use the ansatz
        \begin{equation*}
            \psi_n(x) = f_n(x)\exp[-\frac{m\omega}{2\hbar}x^2]
        \end{equation*}
        \item Now insert this ansatz into the original equation and solve for values of $f_n(x)$ that give an exact solution.
        \item Start by calculating that
        \begin{equation*}
            \dv{\psi_n}{x} = \dv{f_n(x)}{x}\exp[-\frac{m\omega}{2\hbar}x^2]-f_n(x)\frac{m\omega x}{\hbar}\exp[-\frac{m\omega}{2\hbar}x^2]
        \end{equation*}
        and thus
        \begin{align*}
            \begin{split}
                \dv[2]{\psi_n}{x} ={}& \dv[2]{f_n(x)}{x}\exp[-\frac{m\omega}{2\hbar}x^2]-2\dv{f_n(x)}{x}\frac{m\omega x}{\hbar}\exp[-\frac{m\omega}{2\hbar}x^2]\\
                & -f_n(x)\frac{m\omega}{\hbar}\exp[-\frac{m\omega}{2\hbar}x^2]+f_n(x)\frac{m^2\omega^2x^2}{\hbar^2}\exp[-\frac{m\omega}{2\hbar}x^2]
            \end{split}\\
            ={}& \left[ f_n''(x)-\frac{2m\omega x}{\hbar}f_n'(x)-f_n(x)\frac{m\omega}{\hbar}+f_n(x)\frac{m^2\omega^2x^2}{\hbar^2} \right]\exp[-\frac{m\omega}{2\hbar}x^2]\\
            ={}& \left[ f_n''(x)-\frac{2m\omega x}{\hbar}f_n'(x)+\frac{m\omega}{\hbar}\left( \frac{m\omega x^2}{\hbar}-1 \right)f_n(x) \right]\exp[-\frac{m\omega}{2\hbar}x^2]
        \end{align*}
        \item Now we insert the above into the full original Schr\"{o}dinger equation, cancelling the exponential term immediately to save space.
        \begin{align*}
            -\frac{\hbar^2}{2m}\left[ f_n''(x)-\frac{2m\omega x}{\hbar}f_n'(x)+\frac{m\omega}{\hbar}\left( \frac{m\omega x^2}{\hbar}-1 \right)f_n(x) \right]+\frac{m\omega^2x^2}{2}f_n(x) &= E_nf_n(x)\\
            -\frac{\hbar^2}{2m}\left( f_n''-\frac{2m\omega x}{\hbar}f_n'-\frac{m\omega}{\hbar}f_n \right) &= E_nf_n\\
            -\frac{\hbar^2}{2m}f_n''+\hbar\omega xf_n'+\left( \frac{\hbar\omega}{2}-E_n \right)f_n &= 0
        \end{align*}
    \end{itemize}
    \item Thus, we have obtained an ODE that we can solve to find particular solutions.
    \item One obvious solution: the \textbf{minimal energy solution}.
    \item \textbf{Minimal energy solution} (to the quantum harmonic oscillator): Take $f_n$ to be a constant $C$. \emph{Given by}
    \begin{align*}
        \psi_0(x) &= C\exp[-\frac{m\omega}{2\hbar}x^2]&
        E_0 &= \frac{\hbar\omega}{2}
    \end{align*}
    \begin{itemize}
        \item Note that it is the above ODE that necessitates $E_0=\hbar\omega/2$ if $f_n$ is to be a constant.
        \item The minimal energy solution classically is zero, but in quantum mechanics, there will always be some energy!
        \begin{itemize}
            \item Zero energy is impossible because it would imply that the position and momentum are both zero. But there needs to be some uncertainty, in both, so the position and momentum \emph{cannot} both be zero.
            \item Essentially, the uncertainty principle \emph{necessitates} a finite nonzero minimal energy. Stated another way, zero energy is \emph{inconsistent} with the uncertainty principle.
        \end{itemize}
    \end{itemize}
    \item What if we postulate that $f_1=b_1x$ for some constant $b_1$?
    \begin{itemize}
        \item Then the ODE simplifies to
        \begin{align*}
            \hbar\omega b_1x+\left( \frac{\hbar\omega}{2}-E_1 \right)b_1x &= 0\\
            E_1 &= \frac{3\hbar\omega}{2}
        \end{align*}
        \item Note that
        \begin{equation*}
            E_1-E_0 = \hbar\omega
        \end{equation*}
        \begin{itemize}
            \item This observation is important because we can actually prove that
            \begin{equation*}
                E_{n+1}-E_n = \hbar\omega
            \end{equation*}
            for all $n=0,1,2,\dots$.
            \item We will not prove this in this class, though; we will just postulate it.
            \item Essentially, what we do is assume that
            \begin{equation*}
                f_N(x) = \sum_{n=1}^Nb_nx^n
            \end{equation*}
            and solve.
            \item All the solutions are either even or odd solutions based on whether $N$ is even or odd. These "even" and "odd" solutions correspond to even and odd polynomial functions.
        \end{itemize}
        \item This means that
        \begin{equation*}
            E_n = \hbar\omega\left( n+\frac{1}{2} \right)
        \end{equation*}
    \end{itemize}
    \item In particular, if we let $\xi=x\sqrt{m\omega/\hbar}$, then the solutions $f_N$ are the \textbf{Hermite polynomials}.
    \item \textbf{Hermite polynomial}: A polynomial of the following form. \emph{Denoted by} $\bm{H_n(\xi)}$. \emph{Given by}
    \begin{equation*}
        H_n(\xi) = (-1)^n\exp(\xi^2)\dv[n]{\xi}[\exp(-\xi^2)]
    \end{equation*}
    \item Thus, the general solutions to the quantum harmonic oscillator
    \begin{equation*}
        \psi_n(x) = \left( \frac{m\omega}{\hbar\pi} \right)^{1/4}\frac{H_n(\xi)}{\sqrt{2^nn!}}\exp[-\frac{\xi^2}{2}]
    \end{equation*}
    \item On Monday, we will derive this result using \textbf{rasing} and \textbf{lowering} operators.
\end{itemize}



\section{Office Hours (Wagner)}
\begin{itemize}
    \item PSet 2, Q1: Do you want us to rederive the wavefunction, or just answer the questions in the parts?
    \item Do we have to show our integration steps, or are integral calculators fine?
    \begin{itemize}
        \item Show whatever we feel comfortable with.
        \item Sounds like Wagner thinks we should be able to do all these calculations like we were born doing them, but integral calculators and skipping steps shouldn't lose us any points.
    \end{itemize}
    \item Including Problem 3 was probably a mistake, but now that it's included, we have to do it.
    \item PSet 2, Q1d:
    \begin{itemize}
        \item One way to do this problem is to remember that $\dv*{\Exp{\hat{r}}}{t}=\Exp{p}/m$ and $\dv*{\Exp{p}}{t}=0$.
        \item Now the $\psi=\sum_nc_n\sin(k_nx)$.
        \item The mean value of the momentum, once computed explicitly, is
        \begin{equation*}
            \Exp{p} \propto \int\dd{x}\left[ \sum_nc_n\sin(k_nx)\cdot\pdv{x}(\sum_nc_n\sin(k_nx)) \right]
        \end{equation*}
        \item Then we integrate using the trick that
        \begin{equation*}
            \sin x\cos y = \frac{1}{2}\sin(x+y)+\frac{1}{2}\sin(x-y)
        \end{equation*}
        \begin{itemize}
            \item Wagner briefly proves this trig identity.
        \end{itemize}
        \item Recall tricks like given an \emph{even} function $f$,
        \begin{equation*}
            \int_{-L}^L\dd{x}x\cdot f(x) = 0
        \end{equation*}
    \end{itemize}
    \item PSet 2, Q2:
    \begin{itemize}
        \item There is some part where we do not need to find exact solutions.
    \end{itemize}
\end{itemize}




\end{document}