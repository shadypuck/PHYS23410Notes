\documentclass[../notes.tex]{subfiles}

\pagestyle{main}
\renewcommand{\chaptermark}[1]{\markboth{\chaptername\ \thechapter\ (#1)}{}}

\begin{document}




\chapter{Origins of Quantum Mechanics}
\section{Historical Perspective}
\begin{itemize}
    \item \marginnote{1/3:}Intro to Wagner.
    \begin{itemize}
        \item Will be teaching both Quantum I-II.
        \item The structure of the course is explained in the syllabus on Canvas.
        \item Every Friday, we get a new PSet due the next Friday.
        \begin{itemize}
            \item This Friday, we will probably not get a PSet; PSet 1 will be handed out on Friday the 12th.
        \end{itemize}
        \item $50\%$ of our grade is PSets; $50\%$ is midterm and final.
        \begin{itemize}
            \item This may fluctuate a bit.
        \end{itemize}
        \item Starting second week, we'll have 3 regular meetings each week.
        \item If you have any problems, please get in touch with Wagner or the TAs!
        \item Email: \href{mailto:elcwagner@gmail.com}{elcwagner@gmail.com}.
        \item OH will probably be on Wednesdays.
        \item PSets posted on Canvas; solutions posted on Canvas after the deadline, too!
        \item If there is something missing from Canvas, contact Wagner.
    \end{itemize}
    \item Announcement.
    \begin{itemize}
        \item No discussion sections today most likely; write to the TAs to confirm or if we want to discuss anything with the TAs.
        \item Wagner will \emph{hopefully} (not certainly) be back on Fridays.
    \end{itemize}
    \item Outline of the course.
    \begin{enumerate}
        \item Historical perspective.
        \begin{itemize}
            \item Particle wave duality.
            \item Uncertainty principle
        \end{itemize}
        \item Schr\"{o}dinger equation and the wave function.
        \item Formalism --- observables in QM.
        \item Time-independent potentials.
        \begin{itemize}
            \item One-dimensional problems.
        \end{itemize}
        \item Angular momentum.
        \item Three dimensional problems.
        \begin{itemize}
            \item The hydrogen atom.
        \end{itemize}
        \item Spin, fermions, and bosons.
        \item Symmetries and conservation laws.
    \end{enumerate}
    \item We now begin discussing the origins of quantum mechanics.
    \item \textbf{Photoelectric effect}: Electrons ejected from a metal when irradiated with light behave in a strange way.
    \begin{itemize}
        \item In 1887, Hertz discovered this effect.
        \item By 1905, it was clear that\dots
        \begin{enumerate}
            \item No electrons were emitted unless the frequency of light was above a threshold value;
            \item The kinetic energy of the electrons grew linearly with frequency;
            \item The number of electrons depended on the light intensity.
        \end{enumerate}
        \item These were three very strange phenomena.
        \item In 1905, Einstein proposed a radical solution to this problem:
        \begin{enumerate}
            \item Light is composed of \textbf{quanta}, that today we call \textbf{photons}.
            \item Each photon's energy is proportional to the frequency of light.
        \end{enumerate}
        \item Essentially, Einstein said that if we model light this way, our model works.
        \item Thus, the kinetic energy of the electrons is given by
        \begin{equation*}
            K = h\nu-W
        \end{equation*}
        where $h\nu$ is the kinetic energy of the photon and $W$ is the minimum energy necessary to separate the electrons from the metal.
        \item Assuming the intensity to be proportional to the number of photons, we obtain the right behavior.
    \end{itemize}
    \item The constant that relates the energy of the photon to its frequency is
    \begin{equation*}
        h = \SI{6.626e-34}{\joule\second}
        = \SI{4.125e-15}{\electronvolt\second}
    \end{equation*}
    and had been introduced before by Planck in 1900 to solve the so-called black body radiation problem.
    \begin{itemize}
        \item Planck, however, thought of light emitted in quanta as a description of the emission process and not as the nature of light. In Planck's derivation, the average energy of the radiation emitted at a given frequency and temperature was given by a simple average weighted by Boltzmann factors:
        \begin{equation*}
            \langle E \rangle = \frac{\sum nh\nu\e[-hn\nu/kT]}{\sum\e[-hn\nu/kT]}
            = \frac{h\nu}{\e[h\nu/kT]-1}
        \end{equation*}
        \begin{itemize}
            \item $nh\nu$ is photon energy, and the $\e$ term is a Boltzmann factor.
            \item Boltzmann factors won't play a further role in this course (phew!).
        \end{itemize}
        \item This implied a suppression for large frequencies instead of the classical value of $kT$. This value had been obtained for $h\to 0$, and it implied an unobserved infinite emission energy when summed over all frequencies!
    \end{itemize}
    \item We now look into some implications of light quanta. Specifically, we will look at\dots
    \begin{itemize}
        \item \textbf{Compton scattering};
        \item Light spectra;
        \item Wave aspect of particles.
    \end{itemize}
    \item \textbf{Compton scattering}:\footnote{Recall the brief allusion to this in CHEM30200Notes.} The inelastic scattering of light off of a charged particle, resulting in a decrease in energy of the photon.
    \begin{figure}[H]
        \centering
        \begin{tikzpicture}[
            every node/.style={black,text height=1.5ex,text depth=0.25ex},scale=2
        ]
            \footnotesize
            \draw [orx,thick,-latex,decorate,decoration={snake,segment length=3.33mm,amplitude=1pt,post length=1mm}] (0,0) -- node[above]{$h\nu'$} (25:1) coordinate (hv);
            \draw [grx,thick,-latex] (0,0) -- (-35:1) coordinate (e);
            \draw (0,0) coordinate (O) -- (1.5,0) coordinate (l);
            \fill [rex] circle (1pt) node[below]{$e$};
            
            \begin{scope}[on background layer]
                \pic [draw,orx,angle radius=1cm,angle eccentricity=1.16,pic text={$\theta$}] {angle=l--O--hv};
                \pic [draw,grx,angle radius=1cm,angle eccentricity=1.19,pic text={$\theta'$}] {angle=e--O--l};
            \end{scope}
    
            \draw [blx,thick,-latex,decorate,decoration={snake,segment length=2mm,amplitude=1pt,post length=1mm}] (-2,0) -- node[below]{$h\nu$} (-0.5,0);
        \end{tikzpicture}
        \caption{Compton scattering.}
        \label{fig:comptonScattering}
    \end{figure}
    \begin{itemize}
        \item Further confirmation of the existence of quanta of light came from the study of its scattering with electrons.
        \item Two properties were observed:
        \begin{enumerate}
            \item Outgoing light had a different frequency than incoming.
            \item Frequency of the outgoing light depended on the emission angle.
        \end{enumerate}
        \item Let's treat photons as relativistic particles. Conservation of energy and momentum should be imposed. Then we have\dots
        \begin{enumerate}
            \item $h\nu+m_ec^2=h\nu'+E_e$ --- Energy.
            \item $h\nu=h\nu'\cos\theta+c|p_e|\cos\theta'$ --- Momentum.
            \item $h\nu'\sin\theta=c|p_e|\sin\theta'$ --- Momentum.
        \end{enumerate}
        \item Where do these equations come from?
        \begin{itemize}
            \item Fact: $|p|=E/c$.
            \item Equations 2,3 have been multiplied through by $c$!
        \end{itemize}
        \item But we know that $E_e^2=m_e^2c^4+c^2|p_e|^2$.
        \begin{itemize}
            \item "This should have been taught in a previous course." It wasn't for me. What else did I miss, and where can I read about it??
        \end{itemize}
        \item Hence, from (1),
        \begin{equation*}
            (h\nu-h\nu'+m_ec^2)^2 = m_e^2c^4+c^2|p_e|^2
        \end{equation*}
        \item In addition, $(2)^2+(3)^2$ yields
        \begin{equation*}
            c^2|p_e|^2 = (h\nu-h\nu'\cos\theta)^2+(h\nu'\sin\theta)^2
        \end{equation*}
        \item By substituting the second expression into the first, expanding, cancelling the common $m_e^2c^4$, $(h\nu)^2$ and $(h\nu')^2$ factors on left and right, and algebraically rearranging, we get
        \begin{align*}
            2m_ec^2(h\nu-h\nu') &= 2h^2\nu\nu'(1-\cos\theta)\\
            \frac{1}{\nu'}-\frac{1}{\nu} = \frac{h}{m_ec^2}(1-\cos\theta)
        \end{align*}
        \item This last result above is important!
        \item Observe that we get $\nu=\nu'$ for $h=0$; this is the classical result!
        \item An alternate form of the above result may be obtained via the relation $c=\lambda\nu$:
        \begin{equation*}
            \Delta\lambda = \lambda'-\lambda = \frac{h}{m_ec}(1-\cos\theta)
        \end{equation*}
        \item The quantity $h/m_ec$ is called the \textbf{Compton wavelength} and plays an important role in atomic physics.
        \item Compton scattering experimental results are in full agreement with the light quanta predictions, i.e., the derivation just described!
    \end{itemize}
    \item \textbf{Compton wavelength}: The quantity defined as follows. \emph{Denoted by} $\bm{\lambda_c}$. \emph{Given by}
    \begin{equation*}
        \lambda_c = \frac{h}{m_ec} = \SI{2.426e-12}{\meter}
    \end{equation*}
    \item Light spectra.
    \begin{itemize}
        \item From the definition of the Compton wavelength, we have that the energy of an electron is
        \begin{equation*}
            E_e = m_ec^2 = \frac{ch}{\lambda_c} = \SI{511}{\kilo\electronvolt}
        \end{equation*}
        \item Since gamma rays are those with $h\nu>\SI{100}{\kilo\electronvolt}$, $\lambda_c$ corresponds to one of these.
        \item For comparison, visible light has a frequency of around \SI{e15}{\hertz}, implying that $h\nu=\SIrange{3}{6}{\electronvolt}$
    \end{itemize}
    \item Wave aspect of particles.
    \begin{itemize}
        \item De Broglie, in 1923, speculated that since light behaved in a dual way (i.e., as a wave and also as a particle), so should any other particle in nature.
        \item For instance, electrons must have a wave-light behavior.
        \item Only difference between electrons and light is that electrons are massive (have mass), while light has a vanishing mass.
        \item Light has energy $E=c|\vec{p}|=h\nu$ and momentum $\vec{p}=(h/2\pi)\cdot\vec{k}$, where $\vec{k}$ is the \textbf{wavevector} having magnitude $|\vec{k}|=2\pi/\lambda$.
        \item For massive particles,
        \begin{equation*}
            E = \sqrt{c^2\vec{p}{\,}^2+m^2c^4} = mc^2\sqrt{1+\frac{\vec{p}{\,}^2}{m^2c^2}}
        \end{equation*}
        \item For nonrelativistic particles, $|\vec{p}|\ll mc$, so we may expand the square root's Taylor expansion to first order to get
        \begin{equation*}
            E = mc^2+\frac{\vec{p}{\,}^2}{2m}
        \end{equation*}
        \begin{itemize}
            \item Recall also that
            \begin{equation*}
                \vec{p} = \hbar\vec{k}
            \end{equation*}
            \item Scalar-wise, note that $p=E/c=h\nu/c=h/\lambda=(h/2\pi)(2\pi/\lambda)=\hbar k$.
            \item The derivation of angular momentum in the Bohr model yields the correct result (this one), even though the conceptual wave function used in the derivation is wrong.
            \item Review these derivations and the relation to here!
        \end{itemize}
        \item It follows that
        \begin{equation*}
            E-mc^2 = \frac{\vec{p}{\,}^2}{2m}
        \end{equation*}
        where $E-mc^2$ is the kinetic energy
        \begin{equation*}
            E_k = \hbar\omega
        \end{equation*}
        \item These assumptions lead to the form of the wave equation.
    \end{itemize}
    \item \textbf{Wavevector}: The vector with magnitude equal to the wavenumber $2\pi/\lambda$ and direction perpendicular to the wavefront, that is, in the direction of wave propagation.
    \begin{itemize}
        \item Wagner believes that this was covered in PHYS 13300; it wasn't.
    \end{itemize}
    \item \textbf{Reduced Planck constant}: Planck's constant divided by $2\pi$. \emph{Denoted by} $\bm{\hbar}$. \emph{Given by}
    \begin{equation*}
        \hbar = \frac{h}{2\pi}
    \end{equation*}
    \item We now look into some implications of electrons being waves.
    \item The wave function for a free electron of momentum $\vec{p}$ will be
    \begin{equation*}
        \Psi_e \sim \e[i(\vec{k}\cdot\vec{r}-\omega t)]
    \end{equation*}
    where $\omega=2\pi\nu$.
    \begin{itemize}
        \item But if electrons are waves, then they can be represented in states that include the superposition of many wave functions.
        \item Take two such waves $\Psi_1(\vec{r},t)$ and $\Psi_2(\vec{r},t)$.
        \item Then
        \begin{equation*}
            \Psi = \alpha_1\e[i(\vec{k}_1\vec{r}-\omega_1t)]+\alpha_2\e[i(\vec{k}_2\vec{r}-\omega_2t)]
        \end{equation*}
        is an acceptable electron wave function.
        \item If we interpret $\hbar\vec{k}_1$ and $\hbar\vec{k}_2$ as momentum, we see an important difference between classical and quantum mechanics: The electrons may be \emph{simultaneously in two different momentum states}.
        \item Implication: There is in general no real "path" of the electron with a well-defined position and momentum. At most, one can define a "wave packet," with a certain mean value of position and momentum.
    \end{itemize}
    \item How do we justify that electrons also behave like waves? With a double slit experiment, of course!
    \item Double slit experiment.
    \begin{figure}[h!]
        \centering
        \begin{subfigure}[b]{0.47\linewidth}
            \centering
            \begin{tikzpicture}[
                every node/.style=black
            ]
                \footnotesize
                \draw [thick] (0,-1) -- (0,-0.4) (0,-0.2) -- (0,0.2) (0,0.4) -- (0,1);
                \node [right=1mm] at (0,0.3) {$S1$};
                \node [right=1mm] at (0,-0.3) {$S2$};
        
                \draw [rex,thick,-latex] (-2,0.3) -- (-0.5,0.3);
                \draw [rex,thick,-latex] (-2,-0.3) -- node[below]{Beam} (-0.5,-0.3);
        
                \draw [thick] (3,-1) -- node[right]{Screen} (3,1);
                \draw [rey!20!rex,densely dotted] (2.9,-0.4)
                    to[out=90,in=-90,out looseness=1.5,in looseness=0.6] (2.3,0.3) node[right]{$S1$}
                    to[out=90,in=-90,out looseness=0.6,in looseness=1.5] (2.9,1)
                ;
                \draw [rey,dashed] (2.9,-1)
                    to[out=90,in=-90,out looseness=1.5,in looseness=0.6] (2.3,-0.3) node[right]{$S2$}
                    to[out=90,in=-90,out looseness=0.6,in looseness=1.5] (2.9,0.4)
                ;
                \draw [rex,densely dash dot] (2.9,-1)
                    to[out=90,in=-90] node[below left]{Total} (2,0)
                    to[out=90,in=-90] (2.9,1)
                ;
            \end{tikzpicture}
            \caption{Classical.}
            \label{fig:doubleSlita}
        \end{subfigure}
        \begin{subfigure}[b]{0.47\linewidth}
            \centering
            \begin{tikzpicture}[
                every node/.style=black
            ]
                \footnotesize
                \draw [thick] (0,-1) -- (0,-0.4) (0,-0.2) -- (0,0.2) (0,0.4) -- (0,1);
                \draw [rex,decorate,decoration={expanding waves,segment length=2mm,angle=25}] (0,0.3) -- ++(0.6,0);
                \draw [rex,decorate,decoration={expanding waves,segment length=2mm,angle=25}] (0,-0.3) -- ++(0.6,0);
        
                \draw [rex,thick,-latex] (-2,0.3) -- (-0.5,0.3);
                \draw [rex,thick,-latex] (-2,-0.3) -- node[below]{Beam} (-0.5,-0.3);
        
                \draw [thick] (3,-1) -- node[right]{Screen} (3,1);
                \draw [rey!20!rex,densely dotted] (2.9,-0.4)
                    to[out=90,in=-90,out looseness=1.5,in looseness=0.6] (2.3,0.3) node[right]{$S1$}
                    to[out=90,in=-90,out looseness=0.6,in looseness=1.5] (2.9,1)
                ;
                \draw [rey,dashed] (2.9,-1)
                    to[out=90,in=-90,out looseness=1.5,in looseness=0.6] (2.3,-0.3) node[right]{$S2$}
                    to[out=90,in=-90,out looseness=0.6,in looseness=1.5] (2.9,0.4)
                ;
                \draw [rex,densely dash dot,looseness=0.5] (2.9,-1)
                    to[out=90,in=-90] node[below left]{Total} (2.2,-0.5)
                    to[out=90,in=-90] (2.4,-0.3)
                    to[out=90,in=-90] (1.8,0)
                    to[out=90,in=-90] (2.4,0.3)
                    to[out=90,in=-90] (2.2,0.5)
                    to[out=90,in=-90] (2.9,1)
                ;
            \end{tikzpicture}
            \caption{Wave-mechanical.}
            \label{fig:doubleSlitb}
        \end{subfigure}
        \caption{Double slit experiment.}
        \label{fig:doubleSlit}
    \end{figure}
    \begin{itemize}
        \item Let's take a beam of electrons impacting a wall with two slits.
        \begin{itemize}
            \item Note that the focusing sheet with the single slit is not shown in Figure \ref{fig:doubleSlit}.
        \end{itemize}
        \item The electrons going through the slits are measured on a screen.
        \item Classically, one expects the total number of electrons to be the simple sum of those going through slits $S1$ and $S2$.
        \begin{itemize}
            \item If we cover one slit, we'll see one hump; if we cover the other slit, we'll see the other hump.
            \item If both are uncovered, the humps will add to a big hump.
            \item This is what's happening in Figure \ref{fig:doubleSlita}.
        \end{itemize}
        \item If electrons behave as waves, however, the wave function $\Psi=\Psi_{S1}+\Psi_{S2}$ will be the superpositions of the waves coming from $S1$ and $S2$.
        \begin{itemize}
            \item $|\Psi|^2=I$ --- next class, we will justify this, but for now we just accept it.
        \end{itemize}
        \item The intensity, proportional to the number of electrons, will be given by
        \begin{equation*}
            |\Psi|^2 = |\Psi_{S1}+\Psi_{S2}|^2
            = |\Psi_{S1}|^2+|\Psi_{S2}|^2+(\Psi_{S1}^*\Psi_{S2}+\Psi_{S2}^*\Psi_{S1})
        \end{equation*}
        \item Calling $I_i:=|\Psi_{Si}|^2$, the above equation transforms into
        \begin{equation*}
            |\Psi|^2 = I_1+I_2+2\sqrt{I_1I_2}\cos\delta
        \end{equation*}
        where $\delta$ is the phase difference that will depend on the waves' wavelength and the difference of the distances of the screens to the slits.
        \item We will therefore see an interference pattern.
        \item The observations are in full agreement with this prediction.
    \end{itemize}
    \item What is this mysterious $\Psi$?
    \begin{itemize}
        \item We know that $|\Psi|^2$ is the density of probability of finding an electron in a given point.
        \item Essentially, if you take a small volume $\Delta V=\Delta x\Delta y\Delta z$, then $|\Psi|^2\Delta V$ is the probability of finding the electron in $\Delta V$.
        \item Therefore, since the sum of all probabilities should be normalized to 1, we know that
        \begin{equation*}
            \int_{-\infty}^\infty|\Psi|^2\dd{x}\dd{y}\dd{z} = 1
        \end{equation*}
    \end{itemize}
    \item Typical lectures will be 100\% blackboard-based, not this format.
    \item He will deliver notes at least one day before each lecture.
    \item Tell him if the lecture pace isn't good.
    \item Do the lectures align with the textbooks at all?
    \begin{itemize}
        \item \textcite{bib:Griffiths} starts with the Schr\"{o}dinger equation without motivation; Wagner doesn't like that, so he motivates it a bit and then goes with \textcite{bib:Griffiths} from there.
        \item Some historical perspective is good.
        \begin{itemize}
            \item UChicago used to have a course called Modern Physics that covered physics that was no longer modern to do all this stuff, but then they concluded it was useless and this content can be summarized in one lecture (today's!).
            \item Nowadays, the optional QM III covers advanced topics.
        \end{itemize}
        \item Most books (with rare exceptions) cover the same topics, so I can pretty much pick up any book I want to follow along with.
        \begin{itemize}
            \item That being said, \textcite{bib:Landau} is far more advanced and not at all suitable for a first brush with the material, but it is beautiful and Wagner highly recommends it. \textcite{bib:Landau} provides great intuition.
        \end{itemize}
    \end{itemize}
\end{itemize}




\end{document}