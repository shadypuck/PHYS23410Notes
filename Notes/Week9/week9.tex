\documentclass[../notes.tex]{subfiles}

\pagestyle{main}
\renewcommand{\chaptermark}[1]{\markboth{\chaptername\ \thechapter\ (#1)}{}}
\setcounter{chapter}{8}

\begin{document}




\chapter{Particle Physics}
\section{Spin in a Magnetic Field}
\begin{itemize}
    \item \marginnote{2/26:}Today's goal: Spin in a magnetic field.
    \item Review.
    \begin{itemize}
        \item We describe spin as an intrinsic angular momentum.
        \item It has three components $\hat{S}_x,\hat{S}_y,\hat{S}_z$ that don't commute with each other:
        \begin{equation*}
            [\hat{S}_x,\hat{S}_y] = i\hbar\hat{S}_z
        \end{equation*}
        \item The spin operators obey the usual rules of angular momentum, i.e., we can define a state with a definite value of spin squared and direction.
        \begin{align*}
            \hat{\vec{S}}{\,}^2\ket{s,m_s} &= \hbar^2s(s+1)\ket{s,m_s}\\
            \hat{S}_z\ket{s,m_s} &= \hbar m_s\ket{s,m_s}
        \end{align*}
        \item We discovered that the values of $s$ can take half-integer values.
        \begin{itemize}
            \item There are $2s+1$ states for a given $s$, related to the fact that we can have differnt projections of the spin in the $z$-direction indexed by values $-s\leq m_s\leq s$.
        \end{itemize}
        \item A particle moving in the hydrogen atom can only have $\pm 1/2$ states, called "spin up" or "spin down."
        \begin{itemize}
            \item This comes from the fact that in this space, a good representation of the spin operator is in terms of the Pauli matrices:
            \begin{align*}
                \hat{S}_z &= \frac{\hbar}{2}
                \begin{pmatrix}
                    1 & 0\\
                    0 & -1\\
                \end{pmatrix}&
                \hat{S}_x &= \frac{\hbar}{2}
                \begin{pmatrix}
                    0 & 1\\
                    1 & 0\\
                \end{pmatrix}&
                \hat{S}_y &= \frac{\hbar}{2}
                \begin{pmatrix}
                    0 & -i\\
                    i & 0\\
                \end{pmatrix}
            \end{align*}
            \begin{itemize}
                \item Observe that these are Hermitian matrices.
            \end{itemize}
        \end{itemize}
        \item It follows from the matrix definition that
        \begin{equation*}
            \hat{S}_i^2 = \frac{\hbar^2}{4}I
        \end{equation*}
        for $i=x,y,z$ and hence
        \begin{equation*}
            \hat{\vec{S}}{\,}^2 = \hat{S}_x^2+\hat{S}_y^2+\hat{S}_z^2
            = \frac{3\hbar^2}{4}I
        \end{equation*}
        \item If we perform a measurement of the spin in any direction, we always obtain $\pm\hbar/2$.
        \begin{itemize}
            \item This is because these are the eigenvalues of the spin operator (the observables).
        \end{itemize}
        \item An additional layer of formalism: Spinors.
        \item We defined $\chi_\pm$, which have the properties that
        \begin{align*}
            \hat{S}_z\chi_+ &= \frac{\hbar}{2}\chi_+&
            \hat{S}_z\chi_- &= -\frac{\hbar}{2}\chi_-
        \end{align*}
        \begin{itemize}
            \item We sometimes denote these eigenstates as $\chi_\pm^z$.
        \end{itemize}
        \item In the $x$-direction, we have
        \begin{align*}
            \chi_+^x &= \frac{1}{\sqrt{2}}
            \begin{pmatrix}
                1\\
                1\\
            \end{pmatrix}&
                \chi_-^x &= \frac{1}{\sqrt{2}}
                \begin{pmatrix}
                    1\\
                    -1\\
                \end{pmatrix}\\
            \hat{S}_z\chi_+^x &= \frac{\hbar}{2}\chi_+^x&
                \hat{S}_z\chi_-^x &= -\frac{\hbar}{2}\chi_-^x
        \end{align*}
        \item In the $y$-direction, we have
        \begin{align*}
            \chi_+^y &= \frac{1}{\sqrt{2}}
            \begin{pmatrix}
                1\\
                i\\
            \end{pmatrix}&
                \chi_-^y &= \frac{1}{\sqrt{2}}
                \begin{pmatrix}
                    1\\
                    -i\\
                \end{pmatrix}\\
            \hat{S}_z\chi_+^y &= \frac{\hbar}{2}\chi_+^y&
                \hat{S}_z\chi_-^y &= -\frac{\hbar}{2}\chi_-^y
        \end{align*}
        \item It follows from the normalization that
        \begin{equation*}
            |\chi_+|^2+|\chi_-|^2 = 1
        \end{equation*}
        and hence that $|\chi_+|^2$ is the probability of finding the part with spin up in $z$.
        \item We define the state
        \begin{equation*}
            \chi =
            \begin{pmatrix}
                \chi_+\\
                \chi_-\\
            \end{pmatrix}
        \end{equation*}
        and can find that
        \begin{equation*}
            \ev{\hat{S}_z}{\chi} = \frac{\hbar}{2}
            \begin{pmatrix}
                \chi_+^* & \chi_-^*\\
            \end{pmatrix}
            \begin{pmatrix}
                1 & 0\\
                0 & -1\\
            \end{pmatrix}
            \begin{pmatrix}
                \chi_+\\
                \chi_-\\
            \end{pmatrix}
            = \frac{\hbar}{2}\left( |\chi_+|^2-|\chi_-|^2 \right)
        \end{equation*}
        \item We can introduce coefficients such that
        \begin{equation*}
            \chi = c_+\chi_+^z+c_-\chi_-^z
            =
            \begin{pmatrix}
                c_+\\
                c_-\\
            \end{pmatrix}
            =:
            \begin{pmatrix}
                \chi_+\\
                \chi_-\\
            \end{pmatrix}
        \end{equation*}
        and
        \begin{equation*}
            \chi = d_+\chi_+^x+d_-\chi_-^x
            =:
            \begin{pmatrix}
                \chi_+\\
                \chi_-\\
            \end{pmatrix}
        \end{equation*}
        \item Then herein, $|d_\pm|^2$ is the probability of finding the particle with spin up or down in the $x$-direction.
        \item To find one of the two components of the spin eigenstate in a certain direction, take the inner product with the desired eigenstate.
        \begin{itemize}
            \item Examples:
            \begin{align*}
                \begin{pmatrix}
                    1 & 0\\
                \end{pmatrix}
                \begin{pmatrix}
                    \chi_+^z\\
                    \chi_-^z\\
                \end{pmatrix}
                &= c_+&
                \frac{1}{\sqrt{2}}
                \begin{pmatrix}
                    1 & 1\\
                \end{pmatrix}
                \begin{pmatrix}
                    \chi_+^x\\
                    \chi_-^x\\
                \end{pmatrix}
                &= d_+
            \end{align*}
            \item Essentially, we are making use of the following orthogonality relation.
            \begin{equation*}
                \chi_+^\dagger\chi = \chi_+^\dagger(d_+\chi_+^x+d_-\chi_-^x)
                = d_+\chi_+^\dagger\chi_++d_-\chi_+^\dagger\chi_-
                = d_+\cdot 1+d_-\cdot 0
                = d_+
            \end{equation*}
            \item This orthogonality relation is a specific case of the following, more general one.
            \begin{equation*}
                (\chi_+^i)^\dagger\chi_-^i = 0
            \end{equation*}
        \end{itemize}
        \item An explanation of the spinor entries.
        \begin{itemize}
            \item Since
            \begin{equation*}
                \ev{\hat{S}_z}{\tfrac{1}{2},\tfrac{1}{2}} = \frac{\hbar}{2}
            \end{equation*}
            and
            \begin{equation*}
                \ev{\hat{S}_x}{\tfrac{1}{2},\tfrac{1}{2}} = \frac{1}{2}\ev{(\hat{S}_++\hat{S}_-)}{\tfrac{1}{2},\tfrac{1}{2}}
                = 0
            \end{equation*}
            we have that the probability has to be spin up or down; it can't be side to side.
        \end{itemize}
    \end{itemize}
    \item We now begin on new content: A spin in a magnetic field.
    \begin{itemize}
        \item This is related to the interaction between two magnetic fields.
    \end{itemize}
    \item Recall that when a charged particle spins, it acquires a magnetic moment
    \begin{equation*}
        \vec{\mu} = \underbrace{\frac{qe}{2M}\cdot g}_\gamma\vec{S}
    \end{equation*}
    \begin{itemize}
        \item $g$ is called the \textbf{gyromagnetic factor}.
        \item At Fermilab, it was measured/computed to be
        \begin{equation*}
            g = 2+\frac{\alpha}{2\pi}+\cdots
        \end{equation*}
        where $\alpha$ is electromagnetic fine structure constant from the 2/16 lecture.
        \item Compute $g$ to $5^5$ decimal places via experiment, Dirac equation/relativity, quantum corrections.
        \item Kinoshita was a god of computation that made an error in this field and there's some politically incorrect story about that.
    \end{itemize}
    \item From here, we define the Hamiltonian
    \begin{equation*}
        \hat{H} = -\vec{\mu}\cdot\vec{B}-\frac{\hbar^2}{2M}\vec{\nabla}^2+V(\vec{r},t)
    \end{equation*}
    \item Now here, the eigenfunction is a spinor with two components, so we need to solve the following problem.
    \begin{equation*}
        \hat{H}
        \begin{pmatrix}
            \psi_+(x,y,z)\\
            \psi_-(x,y,z)\\
        \end{pmatrix}
        = i\hbar\pdv{t}
        \begin{pmatrix}
            \psi_+(x,y,z)\\
            \psi_-(x,y,z)\\
        \end{pmatrix}
    \end{equation*}
    \item In general, $\hat{H}$ need not be diagonal, and we may have to consider how $\psi_+,\psi_-$ couple.
    \begin{itemize}
        \item However, most commonly, we assume that
        \begin{equation*}
            \frac{\Exp{\hat{\vec{p}}{\,}^2}}{2M},\Exp{V} \ll \Exp{-\vec{\mu}\cdot\vec{B}}
        \end{equation*}
    \end{itemize}
    \item Thus, we will ignore the other terms and solve instead the following problem.
    \begin{equation*}
        -\gamma\vec{B}\vec{S}
        \begin{pmatrix}
            \chi_+\\
            \chi_-\\
        \end{pmatrix}
        = i\hbar\pdv{t}
        \begin{pmatrix}
            \chi_+\\
            \chi_-\\
        \end{pmatrix}
    \end{equation*}
    \item Choose
    \begin{equation*}
        \vec{B} = B\hat{z}
    \end{equation*}
    \item Observe that
    \begin{equation*}
        \vec{B}\vec{S} = B\hat{z}\cdot\vec{S}
        = B\hat{S}_z
        = \frac{B\hbar}{2}
        \begin{pmatrix}
            1 & 0\\
            0 & -1\\
        \end{pmatrix}
    \end{equation*}
    \begin{itemize}
        \item What is an operator and what is not?? Is $\vec{B}$ an operator? Is $\vec{S}$?
    \end{itemize}
    \pagebreak
    \item Thus, the problem becomes
    \begin{equation*}
        -\frac{\gamma B\hbar}{2}
        \begin{pmatrix}
            1 & 0\\
            0 & -1\\
        \end{pmatrix}
        \begin{pmatrix}
            \chi_+\\
            \chi_-\\
        \end{pmatrix}
        = i\hbar\pdv{t}
        \begin{pmatrix}
            \chi_+\\
            \chi_-\\
        \end{pmatrix}
    \end{equation*}
    \item Fortunately, this problem is not that hard to solve. To begin, the above equation splits into the two following ones (technically as components in equal vectors) after a matrix multiplication.
    \begin{align*}
        -\frac{\gamma B\hbar}{2}\chi_+ &= i\hbar\pdv{\chi_+}{t}&
        \frac{\gamma B\hbar}{2}\chi_- &= i\hbar\pdv{\chi_-}{t}
    \end{align*}
    \item The solutions are then
    \begin{align*}
        \chi_+ &= \chi_+(0)\e[i\gamma Bt/2]&
        \chi_- &= \chi_-(0)\e[-i\gamma Bt/2]
    \end{align*}
    \item Therefore,
    \begin{equation*}
        \ev{\hat{S}_z}{\chi}(0) = \frac{\hbar}{2}\left( |\chi_+(0)|^2-|\chi_-(0)|^2 \right)
    \end{equation*}
    \item Additionally, we can solve for the time dependence of the mean value of $\hat{S}_x$.
    \begin{itemize}
        \item To begin, we have that
        \begin{align*}
            \ev{\hat{S}_x}{\chi}(t) &= \frac{\hbar}{4}
            \begin{pmatrix}
                \chi_+^*(0)\e[-i\gamma Bt/2] & \chi_-^*(0)\e[i\gamma Bt/2]\\
            \end{pmatrix}
            \begin{pmatrix}
                0 & 1\\
                1 & 0\\
            \end{pmatrix}
            \begin{pmatrix}
                \chi_+(0)\e[i\gamma Bt/2]\\
                \chi_-(0)\e[-i\gamma Bt/2]\\
            \end{pmatrix}\\
            &= \frac{\hbar}{4}\left[ \chi_+^*(0)\chi_-(0)\e[-i\gamma Bt]+\chi_-^*(0)\chi_+(0)\e[i\gamma Bt] \right]
        \end{align*}
        \item Now observe that $\chi_\pm(0)$ are just complex numbers that may be written in the form
        \begin{equation*}
            \chi_\pm(0) = |\chi_\pm(0)|\e[i\phi_\pm]
        \end{equation*}
        \item Thus, continuing from the above,
        \begin{align*}
            \ev{\hat{S}_x}{\chi}(t) &= \frac{\hbar}{4}|\chi_+(0)||\chi_-(0)|\left[ \e[-i\gamma Bt+i\phi_--i\phi_+]+\e[i\gamma Bt-i\phi_-+i\phi_+] \right]\\
            &= \frac{\hbar}{2}|\chi_+(0)||\chi_-(0)|\cos(-\gamma Bt+\phi_--\phi_+)
        \end{align*}
    \end{itemize}
    \item Analogously, we have that
    \begin{equation*}
        \ev{\hat{S}_y}{\chi}(t) = \frac{\hbar}{2}|\chi_+(0)||\chi_-(0)|\sin(-\gamma Bt+\phi_--\phi_+)
    \end{equation*}
    \item Together, these last two major results lead to \textbf{spin precession}.
    \item \textbf{Spin precession}: The oscillation of the mean values of $\hat{S}_x,\hat{S}_y$ in time.
    \item Thus, the spin keeps its component in the same direction, but rotates.
    \begin{figure}[H]
        \centering
        \begin{tikzpicture}[
            every node/.style=black
        ]
            \small
            \draw [-stealth] (0,0,0) -- (0,0,2) node[below left=-1pt]{$x$};
            \draw [-stealth] (0,0,0) -- (2,0,0) node[right]{$y$};
            \draw [-stealth] (0,0,0) -- (0,2,0) node[above]{$\vec{B}$};
    
            \draw [yex,semithick,densely dashed] (0,1.8,0) -- ({1.5*cos(30)},1.8,{1.5*sin(30)});
            \draw [orx,semithick,->] plot[domain=30:135] ({1.5*cos(\x)},1.8,{1.5*sin(\x)});
            \draw [rex,thick,-latex] (0,0,0) -- ({1.5*cos(30)},1.8,{1.5*sin(30)});
        \end{tikzpicture}
        \caption{Rotating spinor.}
        \label{fig:rotatingSpinor}
    \end{figure}
    \item Calculating the probability of a generic particle being spin up in the $x$-direction.
    \begin{itemize}
        \item Suppose the particle is in the state
        \begin{equation*}
            \chi =
            \begin{pmatrix}
                c_+\\
                c_-\\
            \end{pmatrix}
        \end{equation*}
        \item Then --- as stated earlier --- the probability that the particle is spin up in the $x$-direction is the modulus square of
        \begin{equation*}
            d_+ = (\chi_+^x)^\dagger\chi = \frac{1}{\sqrt{2}}
            \begin{pmatrix}
                1 & 1\\
            \end{pmatrix}
            \begin{pmatrix}
                c_+\\
                c_-\\
            \end{pmatrix}
            = \frac{1}{\sqrt{2}}(c_++c_-)
        \end{equation*}
        \item The modulus square of the above is
        \begin{equation*}
            |d_+|^2 = \frac{1}{2}(c_+^*+c_-^*)(c_++c_-)
        \end{equation*}
        \item Using the polar form of the spin eigenstate derived last lecture, it follows that
        \begin{align*}
            |d_+|^2 &= \frac{1}{2}\left[ \cos(\frac{\theta_s}{2})\e[i\phi_s/2]+\sin(\frac{\theta_s}{2})\e[-i\phi_s/2] \right]\left[ \cos(\frac{\theta_s}{2})\e[-i\phi_s/2]+\sin(\frac{\theta_s}{2})\e[i\phi_s/2] \right]\\
            &= \frac{1}{2}\left[ \cos^2(\frac{\theta_s}{2})+\sin^2(\frac{\theta_s}{2})+\sin(\frac{\theta_s}{2})\cos(\frac{\theta_s}{2})(\e[i\phi_s]+\e[-i\phi_s]) \right]\\
            &= \frac{1}{2}\left[ 1+2\sin(\frac{\theta_s}{2})\cos(\frac{\theta_s}{2})\cos(\phi_s) \right]\\
            &= \frac{1}{2}\left[ 1+\sin(\theta_s)\cos(\phi_s) \right]\\
            &= \frac{1}{2}\left[ 1+\frac{2}{\hbar}\ev{\hat{S}_x}{\chi} \right]\\
            &= \frac{1}{2}\left[ 1+\frac{2}{\hbar}\cdot\frac{\hbar}{2}|\chi_+(0)||\chi_-(0)|\cos(-\gamma Bt+\phi_--\phi_+) \right]\\
            &= \frac{1}{2}+\frac{|\chi_+(0)||\chi_-(0)|}{2}\cos(-\gamma Bt+\phi_--\phi_+)
        \end{align*}
    \end{itemize}
    \item Combining this with the analogous result for the probability of a generic particle being spin down in the $x$-direction, we have that
    \begin{equation*}
        |d_\pm|^2 = \frac{1}{2}\pm\frac{|\chi_+(0)||\chi_-(0)|}{2}\cos(-\gamma Bt+\phi_--\phi_+)
    \end{equation*}
\end{itemize}



\section{Office Hours (Yunjia)}
\begin{itemize}
    \item \marginnote{2/27:}PSet 7, Q1a: Just show the three commutator relations?
    \begin{itemize}
        \item Yes.
    \end{itemize}
    \item PSet 7, Q1b: Are the two parts of this question independent?
    \begin{itemize}
        \item Yes.
        \item Also note that you'll need to use the traceless condition in your answer.
    \end{itemize}
    \item PSet 7: Do you want us to redo the derivations from class?
    \begin{itemize}
        \item Yes.
    \end{itemize}
\end{itemize}



\section{Stern-Gerlach Experiment}
\begin{itemize}
    \item \marginnote{2/28:}Reminder that the final is next Thursday (unless you need it earlier, like me).
    \item Today: Finish discussing spin in a magnetic field and discuss the amazing Stern-Gerlach experiment.
    \item Review.
    \begin{itemize}
        \item We have a Hamiltonian that ignores kinetic and potential energy.
        \begin{equation*}
            \hat{H} = -\vec{\mu}\cdot\vec{B}
        \end{equation*}
        \begin{itemize}
            \item $\vec{\mu}=\gamma\vec{S}$ is the magnetic moment.
        \end{itemize}
        \item Thus, we have to solve the following Schr\"{o}dinger equation.
        \begin{equation*}
            \hat{H}\chi(t) = i\hbar\pdv{t}[\chi(t)]
        \end{equation*}
        \begin{itemize}
            \item Recall that
            \begin{equation*}
                \chi(t) =
                \begin{pmatrix}
                    \chi_+(t)\\
                    \chi_-(t)\\
                \end{pmatrix}
            \end{equation*}
        \end{itemize}
        \item We also have the following representation of the components of the spin operator.
        \begin{align*}
            \hat{S}_x &= \frac{\hbar}{2}
            \begin{pmatrix}
                0 & 1\\
                1 & 0\\
            \end{pmatrix}&
            \hat{S}_y &= \frac{\hbar}{2}
            \begin{pmatrix}
                0 & -i\\
                i & 0\\
            \end{pmatrix}&
            \hat{S}_z &= \frac{\hbar}{2}
            \begin{pmatrix}
                1 & 0\\
                0 & -1\\
            \end{pmatrix}
        \end{align*}
        \item Picking $\vec{B}=B\hat{z}$, the Schr\"{o}dinger equation expands to
        \begin{equation*}
            -\frac{\gamma B\hbar}{2}
            \begin{pmatrix}
                1 & 0\\
                0 & -1\\
            \end{pmatrix}
            \begin{pmatrix}
                \chi_+\\
                \chi_-\\
            \end{pmatrix}
            = i\hbar\pdv{t}
            \begin{pNiceMatrix}
                \pdv{\chi_+}{t}\\
                \pdv{\chi_-}{t}\\
            \end{pNiceMatrix}
        \end{equation*}
        \item This vector differential equation then separates (because the matrix is diagonal) into the following two scalar differential equations.
        \begin{align*}
            -\frac{\gamma B\hbar}{2}\chi_+ &= i\hbar\pdv{\chi_+}{t}&
            \frac{\gamma B\hbar}{2}\chi_- &= i\hbar\pdv{\chi_-}{t}
        \end{align*}
        \item These ODEs can be solved for the following solutions.
        \begin{align*}
            \chi_+ &= \chi_+(0)\e[i\gamma Bt/2]&
            \chi_- &= \chi_-(0)\e[-i\gamma Bt/2]
        \end{align*}
        \item Then we can compute the mean value of the spin in the three different directions in arbitrary configurations
        \begin{equation*}
            \chi =
            \begin{pmatrix}
                \chi_+\\
                \chi_-\\
            \end{pmatrix}
        \end{equation*}
        \item One example of doing this is
        \begin{align*}
            \ev{\hat{S}_z}{\chi} &= \frac{\hbar}{2}
            \begin{pmatrix}
                \chi_+^* & \chi_-^*\\
            \end{pmatrix}
            \begin{pmatrix}
                1 & 0\\
                0 & -1\\
            \end{pmatrix}
            \begin{pmatrix}
                \chi_+\\
                \chi_-\\
            \end{pmatrix}\\
            &= \frac{\hbar}{2}(|\chi_+|^2-|\chi_-|^2)\\
            &= \frac{\hbar}{2}(|\chi_+(0)|^2-|\chi_-(0)|^2)
        \end{align*}
        \begin{itemize}
            \item Then recall that $|\chi_+|^2,|\chi_-|^2$ are the probabilities of finding the particle with spin up or down, so that together,
            \begin{equation*}
                |\chi_+|^2+|\chi_-|^2 = 1
            \end{equation*}
        \end{itemize}
        \item We can also compute the mean value of spin in the $x$-direction.
        \begin{align*}
            \ev{\hat{S}_x}{\chi} &= \frac{\hbar}{2}
            \begin{pmatrix}
                \chi_+^* & \chi_-^*\\
            \end{pmatrix}
            \begin{pmatrix}
                0 & 1\\
                1 & 0\\
            \end{pmatrix}
            \begin{pmatrix}
                \chi_+\\
                \chi_-\\
            \end{pmatrix}\\
            &= \frac{\hbar}{2}
            \begin{pmatrix}
                \chi_+^* & \chi_-^*\\
            \end{pmatrix}
            \begin{pmatrix}
                \chi_-\\
                \chi_+\\
            \end{pmatrix}\\
            &= \frac{\hbar}{2}(\chi_+^*\chi_-+\chi_-^*\chi_+)\\
            &= \frac{\hbar}{2}\cdot 2\re(\chi_+^*\chi_-)\\
            &= \frac{\hbar}{2}\cdot 2\re\left[ |\chi_+|(0)|\chi_-|(0)\e[-i(\gamma Bt+\chi_+-\chi_-)] \right]\\
            &= \frac{\hbar}{2}\cdot 2|\chi_+|(0)|\chi_-|(0)\cos(\gamma Bt+\phi_+-\phi_-)
        \end{align*}
        \item Note that to get the next-to-last line above, we used the substitutions
        \begin{align*}
            \chi_+(0) &= |\chi_+(0)|\e[i\phi_+]&
            \chi_-(0) &= |\chi_-(0)|\e[i\phi_-]
        \end{align*}
        \item With some algebraic manipulation, we can derive that
        \begin{align*}
            |\chi_+|(0) &= \cos(\frac{\theta_s}{2})&
            |\chi_-|(0) &= \sin(\frac{\theta_s}{2})
        \end{align*}
        \item In particular, these equations come from (or imply) the results that
        \begin{align*}
            \ev{\hat{S}_z}{\chi} &= \frac{\hbar}{2}\left[ \cos^2\left( \frac{\theta_s}{2} \right)-\sin^2\left( \frac{\theta_s}{2} \right) \right]
                = \frac{\hbar}{2}\cos(\theta_s)\\
            \ev{\hat{S}_x}{\chi} &= \frac{\hbar}{2}\sin(\theta_s)\cos(\gamma Bt+\phi_+-\phi_-)
        \end{align*}
        \begin{itemize}
            \item Should it be $\hbar/4$ in the second expression above because of the "2" factor in the following trigonometric identity from which the relevant result is derived??
            \begin{equation*}
                2\sin(\frac{\theta_s}{2})\cos(\frac{\theta_s}{2}) = \sin(\theta_s)
            \end{equation*}
            \item We can relate these results to Figure \ref{fig:polarSpinor}.
            \item The quantity $\gamma B$ is known as the \textbf{Larmor frequency} or \textbf{spin precession}.
        \end{itemize}
        \item If we pay attention to the following, the problem set will be much, much easier!
        \item Let's evaluate the mean value of spin in the $y$-direction again.
        \begin{align*}
            \ev{\hat{S}_y}{\chi} &= \frac{\hbar}{2}
            \begin{pmatrix}
                \chi_+^* & \chi_-^*\\
            \end{pmatrix}
            \begin{pmatrix}
                0 & -i\\
                i & 0\\
            \end{pmatrix}
            \begin{pmatrix}
                \chi_+\\
                \chi_-\\
            \end{pmatrix}\\
            &= \frac{\hbar}{2}
            \begin{pmatrix}
                \chi_+^* & \chi_-^*\\
            \end{pmatrix}
            \begin{pmatrix}
                -i\chi_-\\
                i\chi_+\\
            \end{pmatrix}\\
            &= \frac{\hbar}{2}\cdot\frac{1}{i}\cdot(\chi_+^*\chi_--\chi_-^*\chi_+)\\
            &= \frac{\hbar}{2}\sin(\theta_s)\sin(\gamma Bt+\phi_+-\phi_-)
        \end{align*}
        \item Note that the previous results imply that
        \begin{align*}
            [\hat{H},\hat{S}_z] &= 0&
            [\hat{H},\hat{S}_x] &\neq 0&
            [\hat{H},\hat{S}_y] &\neq 0
        \end{align*}
        \item What if we take the eigenstate of the spin in the upwards $x$-direction? The probability of finding the particle with spin up in the $x$-direction is
        \begin{equation*}
            \left|
                \frac{1}{\sqrt{2}}
                \begin{pmatrix}
                    1 & 1\\
                \end{pmatrix}
                \begin{pmatrix}
                    \chi_+\\
                    \chi_-\\
                \end{pmatrix}
            \right|^2
        \end{equation*}
        \begin{itemize}
            \item Thus, this is $|d_+|^2$ where
            \begin{equation*}
                \chi = d_+\chi_+^x+d_-\chi_-^x
            \end{equation*}
        \end{itemize}
        \item Recall that
        \begin{align*}
            \chi_+^x &= \frac{1}{\sqrt{2}}
            \begin{pmatrix}
                1\\
                1\\
            \end{pmatrix}&
            \chi_-^x &= \frac{1}{\sqrt{2}}
            \begin{pmatrix}
                1\\
                -1\\
            \end{pmatrix}
        \end{align*}
        \item Essentially, the computation we have done inside the absolute value bars above is
        \begin{equation*}
            (\chi_+^x)^\dagger\chi = d_+
        \end{equation*}
        \item Thus, we can get all the way to
        \begin{align*}
            |d_+|^2 ={}& \frac{1}{2}(|\chi_++\chi_-|^2)\\
            ={}& \frac{1}{2}\left| \chi_+(0)\e[i\gamma Bt/2]+\chi_-(0)\e[-i\gamma Bt/2] \right|^2\\
            ={}& \frac{1}{2}\left| |\chi_+(0)|\e[i(\gamma Bt/2+\phi_+)]+|\chi_-(0)|\e[-i(\gamma Bt/2-\phi_-)] \right|^2\\
            \begin{split}
                ={}& \frac{1}{2}\left[ |\chi_+(0)|\e[-i(\gamma Bt/2+\phi_+)]+|\chi_-(0)|\e[i(\gamma Bt/2-\phi_-)] \right]\\
                & \cdot\left[ |\chi_+(0)|\e[i(\gamma Bt/2+\phi_+)]+|\chi_-(0)|\e[-i(\gamma Bt/2-\phi_-)] \right]
            \end{split}\\
            ={}& \frac{1}{2}\left[ |\chi_+|^2+|\chi_-|^2+|\chi_+(0)||\chi_-(0)|\cdot 2\cos(\gamma Bt+\phi_+-\phi_-) \right]\\
            ={}& \frac{1}{2}[1+\sin(\theta_s)\cos(\gamma Bt+\phi_+-\phi_-)]
        \end{align*}
        and
        \begin{equation*}
            |d_-|^2 = \frac{1}{2}[1-\sin(\theta_s)\cos(\gamma Bt+\phi_+-\phi_-)]
        \end{equation*}
    \end{itemize}
    \item We will now cover the Stern-Gerlach very fast, omitting certain details in Wagner's notes.
    \item The setup.
    \begin{figure}[h!]
        \centering
        \begin{tikzpicture}
            \footnotesize
            \draw [rex,thick,-latex] (-1.5,0.5) -- node[black,above]{$\Exp{p_x}$} ++(1,0);
            \draw [rex,semithick]
                (-0.6,0.5) -- (0,0.5)
                (0,0.5) to[out=0,in=-150] (2,0.9) -- node[above]{$+$} node[below=2mm,black]{$(p_x,p_z)$} ++(30:1)
                (0,0.5) to[out=0,in=150] (2,0.1) -- node[below]{$-$} ++(-30:1)
            ;
    
            \draw [white,ultra thick] (0.5,0) -- ++(0,1);
            \draw [white,ultra thick] (1.5,0) -- ++(0,1);
            \draw [orx,->,shorten <=2pt,shorten >=2pt] (0.5,0) -- ++(0,1);
            \node [fill=white,inner sep=1pt] at (1,0.5) {${\color{orx}B\hat{z}}$};
            \draw [orx,->,shorten <=2pt,shorten >=2pt] (1.5,0) -- ++(0,1);
    
            \draw [yey,ultra thick]
                (0,1) -- (2,1)
                (0,0) -- (2,0)
            ;
            \draw [|-|] (0,-0.5) -- node[below]{$\dd{x}$} ++(2,0);
    
            \draw (3,-1) -- node[right]{Screen} (3,2);
            \draw [stealth-stealth] (-1.4,1.6) node[above]{$z$} -- ++(0,-0.3) -- ++(0.3,0) node[right]{$x$};
        \end{tikzpicture}
        \caption{Stern-Gerlach experiment.}
        \label{fig:sternGerlach}
    \end{figure}
    \begin{itemize}
        \item A particle enters the setup with mean momentum $\Exp{p_x}$.
        \item If we're trying to keep the particle straight in the magnetic field, it will be difficult because it will experience a Lorentz force that directs it out of the page.
        \item The magnetic field is given by
        \begin{equation*}
            \vec{B} = (B_0+\alpha z)\hat{z}
        \end{equation*}
        \item The change (??) in the magnetic field is zero.
        \begin{equation*}
            \vec{\nabla}\vec{B} = 0
        \end{equation*}
        \item We assume that $B_0\gg\alpha z$.
    \end{itemize}
    \item The Schr\"{o}dinger equation to solve here is
    \begin{equation*}
        -\vec{\mu}\cdot\vec{B}\chi = i\hbar\pdv{\chi}{t}
    \end{equation*}
    \begin{itemize}
        \item It follows that
        \begin{equation*}
            \chi_\pm = \chi_\pm(0)\e[\mp(i\gamma/2)(B_0+\alpha z)t]
        \end{equation*}
        \item Thus,
        \begin{equation*}
            \ev{\hat{p}_z}{\chi} = \int\dd{z}
            \begin{pmatrix}
                \chi_+(z) & \chi_-(z)\\
            \end{pmatrix}
            \left( -i\hbar\pdv{z} \right)
            \begin{pmatrix}
                \chi_+(z)\\
                \chi_-(z)\\
            \end{pmatrix}
        \end{equation*}
        where
        \begin{equation*}
            \chi =
            \begin{pmatrix}
                \chi_+\\
                \chi_-\\
            \end{pmatrix}
        \end{equation*}
        \item The above equation simplifies to
        \begin{equation*}
            \ev{\hat{p}_z}{\chi} = \int\dd{z}(|\chi_+(z)|^2+|\chi_-(z)|^2)
            = 1
        \end{equation*}
        \item Additionally, we have that
        \begin{equation*}
            \ev{p_z}{\chi_\pm} = \pm|\chi_+(0)|^2\frac{\gamma Bt}{2}
        \end{equation*}
    \end{itemize}
    \item If we run three consecutive Stern-Gerlach experiments in series, we can split spins in the $x$, $y$, and $z$ directions.
    \begin{itemize}
        \item See picture from class.
    \end{itemize}
    \item Note that spin is a completely quantum phenomenon; there is \emph{no} classical analogy.
\end{itemize}




\end{document}