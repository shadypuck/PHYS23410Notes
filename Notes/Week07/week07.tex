\documentclass[../notes.tex]{subfiles}

\pagestyle{main}
\renewcommand{\chaptermark}[1]{\markboth{\chaptername\ \thechapter\ (#1)}{}}
\setcounter{chapter}{6}

\begin{document}




\chapter{Time-Independent Problems in 3D}
\section{Three-Dimensional Harmonic Oscillator}
\begin{itemize}
    \item \marginnote{2/12:}Last time.
    \begin{itemize}
        \item We discussed some of the problems we face in 3D.
        \item The Hamiltonian is now
        \begin{equation*}
            \hat{H} = -\frac{\hbar^2}{2m}\left[ \pdv[2]{x}+\pdv[2]{y}+\pdv[2]{z} \right]+V(x,y,z)
        \end{equation*}
        \begin{itemize}
            \item Derivatives in three coordinates.
            \item The potential is time-independent.
            \item If the potential does not depend on anything more specific (e.g., is not central, for instance), then only $\hat{H}$ is conserved.
        \end{itemize}
        \item We solve
        \begin{equation*}
            \hat{H}\psi(x,y,z) = E\psi(x,y,z)
        \end{equation*}
        for $\psi,E$.
        \item There are three compatible operators:
        \begin{equation*}
            \hat{H},\ \hat{\vec{L}}{\,}^2,\ \hat{L}_z
        \end{equation*}
        \begin{itemize}
            \item The $z$-angular momentum operator, in particular, has the form
            \begin{equation*}
                \hat{L}_z = -i\hbar\pdv{\phi}
            \end{equation*}
            which is analogous to the form $\hat{p}_z=-i\hbar(\pdv*{z})$.
        \end{itemize}
        \item The potential is central, i.e.,
        \begin{equation*}
            V(x,y,z) = V(r) = V(\sqrt{x^2+y^2+z^2})
        \end{equation*}
        \begin{itemize}
            \item If the potential is depends on $r$, we solve the ODE in polar coordinates $(r,\theta,\phi)$.
        \end{itemize}
    \end{itemize}
    \item There are also many cases when we only have
    \begin{equation*}
        V(x,y,z) = V(\sqrt{x^2+y^2})
    \end{equation*}
    \begin{itemize}
        \item In this case, $\hat{H},\hat{L}_z,\hat{p}_z$ will all be compatible.
    \end{itemize}
    \item If the potential depends via
    \begin{equation*}
        V(x,y,z) = V(\sqrt{x^2+y^2},z)
    \end{equation*}
    then we will conserve $\hat{H},\hat{L}_z$.
    \begin{itemize}
        \item We will play with this in the problem set.
    \end{itemize}
    \pagebreak
    \item Today, we begin with the \textbf{asymmetric harmonic oscillator}.
    \item \textbf{Asymmetric harmonic oscillator}: A particle subject to the following three-dimensional potential. \emph{Constraint}
    \begin{equation*}
        V(x,y,z) = \frac{M\omega_1^2x^2}{2}+\frac{M\omega_2^2y^2}{2}+\frac{M\omega_3^2z^2}{2}
    \end{equation*}
    \begin{itemize}
        \item This potential is special in the sense that it allows us to solve by separation of variables.
        \item In other words, since we can write the ODE in the form
        \begin{equation*}
            \left[ -\frac{\hbar^2}{2M}\pdv[2]{\psi}{x}+\frac{M\omega_1^2x^2}{2}\psi \right]+\left[ -\frac{\hbar^2}{2M}\pdv[2]{\psi}{y}+\frac{M\omega_2^2y^2}{2}\psi \right]+\left[ -\frac{\hbar^2}{2M}\pdv[2]{\psi}{z}+\frac{M\omega_3^2z^2}{2}\psi \right] = E\psi
        \end{equation*}
        we may write
        \begin{equation*}
            \psi(x,y,z) = X(x)Y(y)Z(z)
        \end{equation*}
        \item This allows us to algebraically manipulate the ODE into the form
        \begin{equation*}
            \frac{1}{X}\left[ -\frac{\hbar^2}{2M}\dv[2]{X}{x}+\frac{M\omega_1^2x^2}{2}X \right]+\frac{1}{Y}\left[ -\frac{\hbar^2}{2M}\dv[2]{Y}{y}+\frac{M\omega_2^2y^2}{2}Y \right]+\frac{1}{Z}\left[ -\frac{\hbar^2}{2M}\dv[2]{Z}{z}+\frac{M\omega_3^2z^2}{2}Z \right] = E
        \end{equation*}
        \begin{itemize}
            \item We switch from partial to total derivatives here because now each function is only a function of one variable (e.g., $X(x)$ depends only on $x$)!
        \end{itemize}
        \item Since the sum of these three independent terms is equal to a constant, each term must equal a constant!
        \item Splitting the above equation into three, we obtain
        \begin{align*}
            -\frac{\hbar^2}{2M}\dv[2]{X}{x}+\frac{M\omega_1^2x^2}{2}X &= E_{n_1}X\\
            -\frac{\hbar^2}{2M}\dv[2]{Y}{y}+\frac{M\omega_2^2y^2}{2}Y &= E_{n_2}Y\\
            -\frac{\hbar^2}{2M}\dv[2]{Z}{z}+\frac{M\omega_3^2z^2}{2}Z &= E_{n_3}Z
        \end{align*}
        \begin{itemize}
            \item It follows that
            \begin{equation*}
                E = E_{n_1}+E_{n_2}+E_{n_3}
            \end{equation*}
        \end{itemize}
        \item We already know the solution to each of these three ODEs! They are just quantum harmonic oscillators. Thus,
        \begin{equation*}
            E_{n_i} = \hbar\omega_i\left( n_i+\frac{1}{2} \right)
        \end{equation*}
        and
        \begin{equation*}
            E = E_{n_1n_2n_3}
            = \hbar\omega_1\left( n_1+\frac{1}{2} \right)+\hbar\omega_2\left( n_2+\frac{1}{2} \right)+\hbar\omega_3\left( n_3+\frac{1}{2} \right)
        \end{equation*}
        \item Additionally, it follows that the wave functions of each direction are of the form (for example)
        \begin{equation*}
            X_{n_1}(x) = \left( \frac{M\omega_1}{\hbar\pi} \right)^{1/4}\frac{H_{n_1}(\xi_1)}{\sqrt{2^{n_1}n_1!}}\exp[-\frac{\xi_1^2}{2}]
        \end{equation*}
        where $\xi_1=x\sqrt{M\omega_1/\hbar}$.
    \end{itemize}
    \item What happens to $X_{n_1}(x),Y_{n_2}(y)$ in the limiting case that $n_1\to n_2$, $x\to y$, and $\omega_1\to\omega_2$?
    \begin{itemize}
        \item We start approaching something interesting.
        \item We need to go a bit further, though.
    \end{itemize}
    \item Now consider the limiting case where
    \begin{equation*}
        \omega_1 = \omega_2
        = \omega_3
        = \omega
    \end{equation*}
    \begin{itemize}
        \item Herein, the Hamiltonian becomes
        \begin{align*}
            \hat{H} &= -\frac{\hbar^2}{2M}\vec{\nabla}^2+\frac{M\omega^2}{2}(x^2+y^2+z^2)\\
            &= -\frac{\hbar^2}{2M}\vec{\nabla}^2+\frac{M\omega^2r^2}{2}
        \end{align*}
        \item In this \emph{central potential}, recall that we have
        \begin{equation*}
            \hat{\vec{L}}{\,}^2Y_{\ell m}(\theta,\phi) = \hbar^2\ell(\ell+1)Y_{\ell m}(\theta,\phi)
        \end{equation*}
        and
        \begin{equation*}
            \hat{L}_zY_{\ell m}(\theta,\phi) = \hbar mY_{\ell m}(\theta,\phi)
        \end{equation*}
        and
        \begin{equation*}
            -\frac{\hbar^2}{2M}\dv[2]{r}[U_{n\ell}(r)]+\underbrace{\left[ V(r)+\frac{\hbar^2\ell(\ell+1)}{2Mr^2} \right]}_{V_\text{eff}(r)}U_{n\ell}(r) = E_{n\ell}U_{n\ell}(r)
        \end{equation*}
    \end{itemize}
    \item This leads directly into our discussion of the \textbf{spherically symmetric harmonic oscillator}.
    \item \textbf{Spherically symmetric harmonic oscillator}: A particle subject to the following one-dimensional effective potential. \emph{Constraint}
    \begin{equation*}
        V_\text{eff}(r) = \frac{M\omega^2r^2}{2}+\frac{\hbar^2\ell(\ell+1)}{2Mr^2}
    \end{equation*}
    \begin{figure}[h!]
        \centering
        \begin{tikzpicture}
            \small
            \draw [-stealth] (-1,0) -- (4,0) node[right]{$r$};
            \draw [-stealth] (0,-0.5) -- (0,2.5);
    
            \footnotesize
            \draw [blx,thick] plot[domain=0.318:3.95,samples=50,smooth] (\x,{1/(4*\x*\x)}) node[above left]{$\dfrac{\hbar^2\ell(\ell+1)}{2Mr^2}$};
            \draw [rex,thick] plot[domain=0:3.162,smooth] (\x,{\x*\x/4}) node[below=5mm]{$\dfrac{M\omega r^2}{2}$};
            
            \draw [grx,thick] plot[domain=0.318:3.146,samples=50,smooth] (\x,{1/(4*\x*\x)+\x*\x/4});
            \draw [grx]
                (0.648,0.7) -- (1.543,0.7)
                (0.551,0.9) -- (1.816,0.9)
                (0.490,1.1) -- (2.040,1.1)
            ;
            \node [grx] at (1,2) {$V_\text{eff}(r)$};
        \end{tikzpicture}
        \caption{Spherically symmetric harmonic oscillator potential.}
        \label{fig:sphrSymHOV}
    \end{figure}
    \begin{itemize}
        \item The problem we have to solve here is
        \begin{equation*}
            -\frac{\hbar^2}{2M}\dv[2]{r}[U_{n\ell}(r)]+\left[ \frac{M\omega r^2}{2}+\frac{\hbar^2\ell(\ell+1)}{2Mr^2} \right]U_{n\ell}(r) = E_{n\ell}U(r)
        \end{equation*}
        \item Recall that
        \begin{align*}
            \psi(r,\theta,\phi) &= R_{n\ell}(r)Y_{\ell m}(\theta,\phi)&
            R_{n\ell}(r)r &= U_{n\ell}(r)
        \end{align*}
        \item In the effective potential, we have the interplay of two peaking potentials as in Figure \ref{fig:sphrSymHOV}.
        \begin{itemize}
            \item The particle will have certain energy states within the well.
        \end{itemize}
        \item In the limiting case that $r$ is small ($r\to 0$), we can approximate the potential as giving us
        \begin{equation*}
            -\frac{\hbar^2}{2M}\dv[2]{U_{n\ell}}{r}+\frac{\hbar^2\ell(\ell+1)}{2Mr^2}U_{n\ell}+\cdots = 0
        \end{equation*}
        \begin{itemize}
            \item In this case, the solution is proportional to
            \begin{equation*}
                U_{n\ell} \propto Cr^{\ell+1}
            \end{equation*}
            \item This is because
            \begin{align*}
                \dv{r}(Cr^{\ell+1}) &= (\ell+1)Cr^\ell\\
                \dv[2]{r}(Cr^{\ell+1}) &= \ell(\ell+1)C\frac{r^{\ell+1}}{r^2}
            \end{align*}
        \end{itemize}
        \item In the limiting case that $r$ is large ($r\to\infty$), we can approximate the potential as giving us
        \begin{equation*}
            -\frac{\hbar^2}{2M}\dv[2]{U_{n\ell}}{r}+\frac{M\omega^2r^2}{2}U_{n\ell}+\cdots = 0
        \end{equation*}
        \begin{itemize}
            \item In this case, the solution is proportional to
            \begin{equation*}
                U_{n\ell} = C\e[-M\omega r^2/2\hbar]
            \end{equation*}
        \end{itemize}
        \item Thus, we combine the two partial solutions to propose the overall ansatz
        \begin{equation*}
            U_{n\ell} = f_{n\ell}r^{\ell+1}\e[-M\omega r^2/2\hbar]
        \end{equation*}
        \item Substituting back into the original ODE, we obtain the differential equation
        \begin{equation*}
            f_{n\ell}''+2\left( \frac{\ell+1}{r}-\frac{M\omega r}{\hbar} \right)f_{n\ell}'+\left[ \frac{2ME_{n\ell}}{\hbar^2}-\frac{(2\ell+3)M\omega}{\hbar} \right]f_{n\ell} = 0
        \end{equation*}
        \item As we have previously, propose that
        \begin{equation*}
            f_{n\ell}(r) = \sum_ja_jr^j
        \end{equation*}
        \begin{itemize}
            \item But there's a problem: $f_{n\ell}'(r=0)=a_1$, and this would allow the $(\ell+1)/r$ term to diverge and make the differential equation blow up.
            \item Thus, we choose $a_1=0$ and proceed.
        \end{itemize}
        \item Substituting this power series into the differential equation, we obtain
        \begin{equation*}
            \sum_jj(j-1)a_jr^{j-2}+2\left( \frac{\ell+1}{r}-\frac{M\omega r}{\hbar} \right)\sum_jja_jr^{j-1}+\left[ \frac{2ME_{n\ell}}{\hbar^2}-\frac{(2\ell+3)M\omega}{\hbar} \right]\sum_ja_jr^j = 0
        \end{equation*}
        \item Make a change of variables $j\to j+2$ so that we can start the sum from zero.
        \begin{multline*}
            \sum_{j=0}^\infty(j+2)(j+1)a_{j+2}r^j+2\left( \frac{\ell+1}{r}-\frac{M\omega r}{\hbar} \right)\sum_{j=0}^\infty(j+2)a_{j+2}r^{j+1}\\
            +\left[ \frac{2ME_{n\ell}}{\hbar^2}-\frac{(2\ell+3)M\omega}{\hbar} \right]\sum_{j=0}^\infty a_jr^j = 0
        \end{multline*}
        \item We will finish this derivation on Wednesday.
    \end{itemize}
\end{itemize}



\section{Office Hours (Yunjia)}
\begin{itemize}
    \item \marginnote{2/13:}PSet 2, Q2c.
    \begin{itemize}
        \item If we can get up to Equation 12 in the answer key, that's full credit.
        \item The thing with $\kappa_{II}^{-1}$ is the idea that if we have a value that's very large (like $\kappa_{II}$ will be as $V_0\to\infty$ since $\kappa_{II}\propto V_0^{1/2}$), then we can Taylor expand in its reciprocal.
        \begin{itemize}
            \item We cannot Taylor expand in the large values; we can only Taylor expand in small values.
            \item This technique is called \textbf{perturbation theory} and will be a major topic of QMechII; Yunjia's use of it here was admittedly a bit extra.
        \end{itemize}
    \end{itemize}
    \item A brief introduction to perturbation theory.
    \begin{itemize}
        \item Suppose we seek to solve an equation
        \begin{equation*}
            f(x,\epsilon) = 0
        \end{equation*}
        where $\epsilon$ is small.
        \item We can approximate the solution in the form
        \begin{equation*}
            f^{(0)}(x)+f^{(1)}(x)\epsilon+f^{(2)}(x)\epsilon^2 = 0
        \end{equation*}
        where the digit superscripts in parentheses just denote different functions, not derivatives or anything like that. For example, we could equally well have used the notation $f,g,h$; it's just that this is less general.
        \item To solve the original equation, we first solve
        \begin{equation*}
            f^{(0)}(x_0) = 0
        \end{equation*}
        for $x_0$.
        \item Then we solve
        \begin{equation*}
            f^{(0)}(x_0+\epsilon x_1)+\epsilon f^{(1)}(x_0) = 0
        \end{equation*}
        for $x_1$.
        \item Continuing in this fashion, our solution takes on the following form and is progressively refined as more terms are calculated.
        \begin{equation*}
            x = x_0+\epsilon x_1+\epsilon^2x_2+\cdots
        \end{equation*}
    \end{itemize}
\end{itemize}



\section{Spherically Symmetric Harmonic Oscillator}
\begin{itemize}
    \item \marginnote{2/14:}Review.
    \begin{itemize}
        \item Recall that the 3D case we're considering corresponds to the Hamiltonian
        \begin{equation*}
            \hat{H} = -\frac{\hbar^2}{2M}\left( \pdv[2]{x}+\pdv[2]{y}+\pdv[2]{z} \right)+\frac{M\omega^2}{2}(x^2+y^2+z^2)
        \end{equation*}
        \begin{itemize}
            \item For this Hamiltonian, we are trying to solve the Eigenvalue equation
            \begin{equation*}
                \hat{H}\psi(x,y,z) = E\psi(x,y,z)
            \end{equation*}
            \item The solution may be obtained in Cartesian coordinates as a limiting case of the asymmetric harmonic oscillator, i.e., via the separation of variables
            \begin{equation*}
                \psi(x,y,z) = X(x)Y(y)Z(z)
            \end{equation*}
            \item This results in the solutions
            \begin{align*}
                \psi(x,y,z) &= \prod_{i=1}^3H_{n_i}(\xi_i)\e[-\xi_i^2/2]c_{n_i}&
                E_{n_1n_2n_3} &= \hbar\omega\left( n_1+n_2+n_3+\frac{3}{2} \right)
            \end{align*}
            where $\xi_i=x_i\sqrt{m\omega/\hbar}$ and $x_1,x_2,x_3=x,y,z$, respectively.
        \end{itemize}
        \item Recall also the polar coordinates $r,\theta,\phi$. The solution may be obtained here as well.
        \begin{itemize}
            \item In polar coordinates, we can see that the potential described above is central.
            \item Thus, we have that
            \begin{equation*}
                \hat{\vec{L}}{\,}^2Y_{\ell m}(\theta,\phi) = \hbar^2\ell(\ell+1)Y_{\ell m}(\theta,\phi)
            \end{equation*}
            and
            \begin{equation*}
                \hat{L}_zY_{\ell m}(\theta,\phi) = \hbar mY_{\ell m}(\theta,\phi)
            \end{equation*}
            and
            \begin{equation*}
                -\frac{\hbar^2}{2M}\dv[2]{r}[U_{n\ell}(r)]+\underbrace{\left[ V(r)+\frac{\hbar^2\ell(\ell+1)}{2Mr^2} \right]}_{V_\text{eff}(r)}U_{n\ell}(r) = E_{n\ell}U_{n\ell}(r)
            \end{equation*}
            where $-\ell\leq m\leq\ell$ and thus there is a $2\ell+1$ degeneracy of $E_{n\ell}$ associated with different $m$.
            \begin{itemize}
                \item Recall that
                \begin{equation*}
                    R_{n\ell}(r) = \frac{U_{n\ell}(r)}{r}
                \end{equation*}
            \end{itemize}
            \item Substituting in
            \begin{equation*}
                V(r) = \frac{M\omega^2r^2}{2}
            \end{equation*}
            we obtain the effective potential described in Figure \ref{fig:sphrSymHOV}.
            \item Limiting cases then lead us to construct the ansatz
            \begin{equation*}
                U_{n\ell} = f_{n\ell}r^{\ell+1}\e[-M\omega r^2/2\hbar]
            \end{equation*}
            \item Now propose that
            \begin{equation*}
                f_{n\ell}(r) = \sum_ja_jr^j
            \end{equation*}
            \item Recall that we may obtain the differential equation
            \begin{equation*}
                f_{n\ell}''+2\left( \frac{\ell+1}{r}-\frac{M\omega r}{\hbar} \right)f_{n\ell}'+\left[ \frac{2ME_{n\ell}}{\hbar^2}-\frac{(2\ell+3)M\omega}{\hbar} \right]f_{n\ell} = 0
            \end{equation*}
            \item We must set $a_1=0$.
            \item Moving on, we obtain
            \begin{equation*}
                \sum_jj(j-1)a_jr^{j-2}+2\left( \frac{\ell+1}{r}-\frac{M\omega r}{\hbar} \right)\sum_jja_jr^{j-1}+\left[ \frac{2ME_{n\ell}}{\hbar^2}-\frac{(2\ell+3)M\omega}{\hbar} \right]\sum_ja_jr^j = 0
            \end{equation*}
        \end{itemize}
    \end{itemize}
    \item We now begin on new content, continuing the same derivation from above.
    \item We can further simplify the above equation by solving for $a_{j+2}$ in terms of $a_j$.
    \begin{itemize}
        \item Begin by bringing all $r$'s into the summations and running all sums from $0$ to $\infty$ with no terms that go to zero so that every term is in $r^j$.
        \begin{multline*}
            \sum_{j=0}^\infty(j+2)(j+1)a_{j+2}r^j+2(\ell+1)\sum_{j=0}^\infty(j+2)a_{j+2}r^j-\frac{2M\omega}{\hbar}\sum_{j=0}^\infty ja_jr^j\\
            % +2\left( \frac{\ell+1}{r}-\frac{M\omega r}{\hbar} \right)\sum_{j=0}^\infty(j+2)a_{j+2}r^{j+1}\\
            +\left[ \frac{2ME_{n\ell}}{\hbar^2}-\frac{(2\ell+3)M\omega}{\hbar} \right]\sum_{j=0}^\infty a_jr^j = 0
        \end{multline*}
        \item Combine the summations.
        \begin{equation*}
            \sum_{j=0}^\infty\left[ (j+1)(j+2)a_{j+2}+2(j+2)(\ell+1)a_{j+2}-\frac{2jM\omega}{\hbar}a_j+\frac{2ME_{n\ell}}{\hbar^2}a_j-\frac{(2\ell+3)M\omega}{\hbar}a_j \right]r^j = 0
        \end{equation*}
        \item Simplify and combine terms.
        \begin{equation*}
            \sum_{j=0}^\infty\left[ (j+2)(j+2\ell+3)a_{j+2}+\left( \frac{2ME_{n\ell}}{\hbar^2}-\frac{M\omega}{\hbar}(2j+2\ell+3) \right)a_j \right] = 0
        \end{equation*}
        \item Because each term in the above summation is affixed to a different power of $r$, meaning that no two terms can cancel, not only is the entire sum above equal to zero, but each individual term in it is equal to zero, too.
        \item Thus, for all $j\in\Z_{\geq 0}$,
        \begin{align*}
            0 &= (j+2)(j+2\ell+3)a_{j+2}+\left( \frac{M\omega}{\hbar}(2j+2\ell+3)-\frac{2ME_{n\ell}}{\hbar^2} \right)a_j\\
            a_{j+2} &= \frac{\frac{2ME_{n\ell}}{\hbar^2}-\frac{M\omega}{\hbar}(2j+2\ell+3)}{(j+2)(j+2\ell+3)}a_j
        \end{align*}
    \end{itemize}
    \item This combined with the fact that $a_1=0$ means that \emph{all odd $a_j$ equal zero.}
    \begin{itemize}
        \item It follows that $f_{n\ell}$ can be viewed as a function of $r^2$, not just $r$, since this fact means that the power series will be of the form
        \begin{equation*}
            f_{n\ell}(r) = a_0+a_2r^2+a_4r^4+a_6r^6+\cdots+a_{2n}r^{2n}+\cdots
        \end{equation*}
    \end{itemize}
    \item Now observe that in the limit of large $j$ (i.e., as $j\to\infty$),
    \begin{equation*}
        a_{j+2} \approx \frac{\frac{M\omega}{\hbar}(2j)}{j^2+2j}
    \end{equation*}
    and thus\footnote{How did we get this transformation to exponential growth??}
    \begin{equation*}
        f_{n\ell}(r) \approx \e[M\omega r^2/\hbar]
    \end{equation*}
    \begin{itemize}
        \item This, in turn, would lead to an exponential growth of $U_{n\ell}$ as $r\to\infty$ and hence a non-renormalizable solution.
        \item Consequently, there must be some maximum value of $j$ which we will denote by $n:=j_\text{max}$.
        \item In particular, $n$ will be the value of $j$ such that the numerator of the expression above giving $a_{j+2}(a_j)$ equals zero. This will guarantee that $a_{n+2}=0$ and hence all $a_j=0$ for $j>n$.
        \item Solving for this $n$, we have that
        \begin{align*}
            \frac{2ME_{n\ell}}{\hbar^2} &= \frac{M\omega}{\hbar}(2n+2\ell+3)\\
            E_{n\ell} &= \hbar\omega\left( n+\ell+\frac{3}{2} \right)
        \end{align*}
        \begin{itemize}
            \item Recall that $n$ is even; $n\geq 0$; $\ell\geq 0$; and for each $\ell$, we have $2\ell+1$ solutions with $-\ell\leq m\leq\ell$ where $\hbar m$ are the eigenvalues of $\hat{L}_z$.
        \end{itemize}
    \end{itemize}
    \item Notice the remarkable similarity between the energy equations for the spherically symmetric harmonic oscillator in Cartesian coordinates (left below) and polar coordinates (right below).
    \begin{align*}
        E_{n_1n_2n_3} &= \hbar\omega\left( \bar{n}+\frac{3}{2} \right)&
        E_{\bar{n}} &= \hbar\omega\left( \bar{n}+\frac{3}{2} \right)
    \end{align*}
    \begin{itemize}
        \item On the left above, $\bar{n}=n_1+n_2+n_3$. On the right above, $\bar{n}=n+\ell$.
    \end{itemize}
    \item Now let's investigate some particular solutions in both cases.
    \item $\bar{n}=0$.
    \begin{itemize}
        \item \underline{Cartesian}: The only possible values are $n_1=n_2=n_3=0$, corresponding to
        \begin{equation*}
            \e[-M\omega(x^2+y^2+z^2)/2\hbar]
        \end{equation*}
        \item \underline{Polar}: The only possible values are $n=\ell=0$, corresponding to
        \begin{equation*}
            \e[-M\omega r^2/2\hbar]
        \end{equation*}
        \item In both cases, there is only one solution, and the solutions are mathematically equivalent.
    \end{itemize}
    \item $\bar{n}=1$.
    \begin{itemize}
        \item \underline{Cartesian}: We could have $n_1=1$, $n_2=n_3=0$; $n_2=1$, $n_1=n_3=0$; or $n_3=1$, $n_2=n_3=0$; corresponding to
        \begin{align*}
            x\e[-M\omega r^2/2\hbar]&&
            y\e[-M\omega r^2/2\hbar]&&
            z\e[-M\omega r^2/2\hbar]
        \end{align*}
        \item \underline{Polar}: We have $n=0$; $\ell=1$; and $m=1$, $m=0$, or $m=-1$; corresponding to
        \begin{align*}
            r\e[-M\omega r^2/2\hbar]\underbrace{\sin\theta\e[i\phi]}_{(x+iy)/r}&&
            r\e[-M\omega r^2/2\hbar]\cos\theta&&
            r\e[-M\omega r^2/2\hbar]\underbrace{\sin\theta\e[-i\phi]}_{(x-iy)/r}
        \end{align*}
        \item In both cases, there are three solutions, and the solutions are mathematically equivalent (up to linear combinations).
    \end{itemize}
    \item A pattern is emerging: Naturally, it makes sense that the coordinate system chosen should not affect the solutions.
    \item $\bar{n}=2$.
    \begin{table}[h!]
        \centering
        \small
        \renewcommand{\arraystretch}{1.2}
        \begin{subtable}{0.3\linewidth}
            \centering
            \begin{tabular}{c|c|c}
                $n_1$ & $n_2$ & $n_3$\\
                \hline
                0 & 0 & 2\\
                0 & 2 & 0\\
                2 & 0 & 0\\
                0 & 1 & 1\\
                1 & 0 & 1\\
                ${\color{white}-}1{\color{white}-}$ & ${\color{white}-}1{\color{white}-}$ & ${\color{white}-}0{\color{white}-}$\\
            \end{tabular}
            \caption{Cartesian coordinates.}
            \label{tab:cartSphrHOa}
        \end{subtable}
        \begin{subtable}{0.3\linewidth}
            \centering
            \begin{tabular}{c|c|c}
                ${\color{white}-}n{\color{white}-}$ & ${\color{white}-}\ell{\color{white}-}$ & $m$\\
                \hline
                0 & 2 & 2\\
                0 & 2 & 1\\
                0 & 2 & 0\\
                0 & 2 & $-1{\color{white}-}$\\
                0 & 2 & $-2{\color{white}-}$\\
                2 & 0 & 0\\
            \end{tabular}
            \caption{Spherical coordinates.}
            \label{tab:cartSphrHOb}
        \end{subtable}
        \caption{Spherically symmetric harmonic oscillator solutions ($\bar{n}=2$).}
        \label{tab:cartSphrHO}
    \end{table}
    \begin{itemize}
        \item In both cases, there are six solutions.
        \item Note that we do not consider the case where $n=\ell=1$ in Table \ref{tab:cartSphrHOb} because this would mean that $j_\text{max}=n=1$ is an odd number, which is not allowed.
    \end{itemize}
\end{itemize}



\section{Hydrogen Atom: Energy Eigenvalues and Eigenstates}
\begin{itemize}
    \item \marginnote{2/16:}Today: The hydrogen atom.
    \item The central potential is
    \begin{equation*}
        V(r) = -\frac{e^2}{4\pi\epsilon_0r}
    \end{equation*}
    \item The problem is an electron revolving around a proton.
    \item The proton and electron have very different masses.
    \begin{align*}
        M_pc^2 &\approx \SI{1}{\giga\electronvolt}&
        m_ec^2 &\approx \SI{511}{\kilo\electronvolt}
    \end{align*}
    \begin{itemize}
        \item The ratio is
        \begin{equation*}
            \frac{M_p}{m_e} \approx 2000
        \end{equation*}
        \item This justifies assuming that the proton is fixed (the Born-Oppenheimer approximation).
    \end{itemize}
    \item The relevant Schr\"{o}dinger equation is
    \begin{equation*}
        -\frac{\hbar^2}{2m_e}\vec{\nabla}^2\psi_{n\ell m}(r,\theta,\phi)-\frac{e^2}{4\pi\epsilon_0r}\psi_{n\ell m}(r,\theta,\phi) = E_{n\ell}\psi_{n\ell m}(\theta,\phi)
    \end{equation*}
    \begin{itemize}
        \item Note that $E$ does not depend on $m$ because $m$ corresponds to the $2\ell+1$ degeneracy in energy.
        \begin{itemize}
            \item Moreover, $m$ only specifies orientation in space, which should intuitively not affect energy because space is isotropic and affine.
            \item This is something we should absolutely know!!
        \end{itemize}
        \item Recall that
        \begin{align*}
            \psi_{n\ell m}(r,\theta,\phi) &= R_{n\ell}(r)Y_{\ell m}(\theta,\phi)\\
            \hat{\vec{L}}{\,}^2Y_{\ell m}(\theta,\phi) &= -\hbar^2\ell(\ell+1)Y_{\ell m}(\theta,\phi)\\
            \hat{L}_zY_{\ell m}(\theta,\phi) &= \hbar mY_{\ell m}(\theta,\phi)
        \end{align*}
        \item Recall also polar coordinates
        \begin{align*}
            z &= r\cos\theta\\
            x &= r\sin\theta\cos\phi\\
            y &= r\sin\theta\sin\phi
        \end{align*}
    \end{itemize}
    \item Making the substitution
    \begin{equation*}
        U_{n\ell}(r) = rR_{n\ell}(r)
    \end{equation*}
    can simplify the Schr\"{o}dinger equation to the following equivalent effective potential and 1D problem.
    \begin{equation*}
        -\frac{\hbar^2}{2m_e}\dv[2]{r}[U_{n\ell}(r)]+\left[ \frac{\hbar^2\ell(\ell+1)}{2m_er^2}-\frac{e^2}{4\pi\epsilon_0r} \right]U_{n\ell}(r) = E_{n\ell}U_{n\ell}(r)
    \end{equation*}
    \begin{itemize}
        \item This is the problem that started the whole game of quantum mechanics; it has enormous consequences in particle physics.
    \end{itemize}
    \item As with the discussion associated with Figure \ref{fig:sphrSymHOV}, we have two competing potentials here (see Figure \ref{fig:1HV}).
    \begin{figure}[h!]
        \centering
        \begin{tikzpicture}
            \small
            \draw [-stealth] (-1,0) -- (4,0) node[right]{$r$};
            \draw [-stealth] (0,-2.5) -- (0,2.5);
    
            \footnotesize
            \draw [blx,thick] plot[domain=0.318:3.95,samples=50,smooth] (\x,{1/(4*\x*\x)}) node[above left]{$\dfrac{\hbar^2\ell(\ell+1)}{2m_er^2}$};
            \draw [rex,thick] plot[domain=0.4:3.95,smooth] (\x,{-1/\x}) node[below left]{$-\dfrac{e^2}{4\pi\epsilon_0r}$};
            
            \draw [grx,thick] plot[domain=0.1746:3.95,samples=100,smooth] (\x,{1/(4*\x*\x)-1/\x});
            \draw [grx]
                (0.293,-0.5) -- (1.707,-0.5)
                (0.323,-0.7) -- (1.106,-0.7)
                (0.380,-0.9) -- (0.731,-0.9)
            ;
            \node [grx,left] at (0,2) {$V_\text{eff}(r)$};
            \node [grx] at (0.8,-0.3) {$E_{n\ell}$};
        \end{tikzpicture}
        \caption{Hydrogen atom potential.}
        \label{fig:1HV}
    \end{figure}
    \begin{itemize}
        \item We are interested in finding the \textbf{bound states}.
        \item In the limiting case that $r$ is small ($r\to 0$), we can approximate the potential as with Figure \ref{fig:sphrSymHOV} and take
        \begin{equation*}
            U_{n\ell} \propto Cr^{\ell+1}
        \end{equation*}
        \item In the limiting case that $r$ is large ($r\to \infty$), we can approximate the potential as going to zero and take
        \begin{equation*}
            U_{n\ell} \propto A\e[\pm k_{n\ell}r]
        \end{equation*}
        where
        \begin{equation*}
            |E_{n\ell}| = \frac{\hbar^2k_{n\ell}^2}{2m_e}
        \end{equation*}
        \item Thus, we combine the two partial solutions to propose the overall ansatz
        \begin{equation*}
            U_{n\ell}(r) = f_{n\ell}(r)r^{\ell+1}\e[-k_{n\ell}r]
        \end{equation*}
        \begin{itemize}
            \item Note that we choose the negative exponent so the solution does not blow up at large $r$.
        \end{itemize}
        \item Following the algebra in the notes, we obtain the following ODE determining $f_{n\ell}$.
        \begin{equation*}
            f_{n\ell}''(r)+f_{n\ell}'(r)\left[ \frac{2(\ell+1)}{r}-2k_{n\ell} \right]+f_{n\ell}(r)\left[ -\frac{2k_{n\ell}(\ell+1)}{r}+\frac{2m_e}{\hbar^2}\frac{e^2}{4\pi\epsilon_0r} \right] = 0
        \end{equation*}
        \begin{itemize}
            \item Aside: The prefactor to the rightmost $1/r$ term above (excepting the 2 coefficient) is typically written as follows.
            \begin{equation*}
                \frac{m_ec}{\hbar}\frac{e^2}{4\pi\epsilon_0\hbar c}
            \end{equation*}
            \item The right fraction is the \textbf{electromagnetic fine structure constant}.
            \item Additionally, the other factor $\hbar/mc$ decomposes into $(h/m_ec)\cdot(1/2\pi)$ where we may recall from the first lecture that $h/m_ec$ is the \textbf{Compton wavelength} $\lambda_c$.
            \item The overall quantity is equal to the inverse of the \textbf{Bohr radius}.
        \end{itemize}
        \item Thus, we can simplify the above equation to
        \begin{equation*}
            f_{n\ell}''(r)+f_{n\ell}'(r)\left[ \frac{2(\ell+1)}{r}-2k_{n\ell} \right]+f_{n\ell}(r)\left[ -\frac{2k_{n\ell}(\ell+1)}{r}+\frac{2}{a_\text{B}r} \right] = 0
        \end{equation*}
        \item As per usual, we propose that
        \begin{equation*}
            f_{n\ell}(r) = \sum_ja_jr^j
        \end{equation*}
        and collapse functions that diverge.
        \item Substituting this power series into the differential equation, we obtain
        \begin{align*}
            0 &= \sum_ja_jr^{j-2}j(j-1)+\sum_ja_jjr^{j-1}\left[ \frac{2(\ell+1)}{r}-2k_{n\ell} \right]+\sum_ja_jr^j\left[ -\frac{2k_{n\ell}(\ell+1)}{r}+\frac{2}{a_\text{B}r} \right]\\
            &= \sum_ja_jr^{j-1}j(j-1)+\sum_ja_jjr^{j-1}[2(\ell+1)-2k_{n\ell}r]+\sum_ja_jr^j\left[ -2k_{n\ell}(\ell+1)+\frac{2}{a_\text{B}} \right]\\
            &= \sum_ja_{j+1}r^jj(j+1)+\sum_ja_{j+1}(j+1)r^j2(\ell+1)-\sum_ja_jjr^jk_{n\ell}2+\sum_ja_jr^j\left[ -2k_{n\ell}(\ell+1)+\frac{2}{a_\text{B}} \right]
        \end{align*}
        \begin{itemize}
            \item From line 1 to line 2, we multiplied through this function equal to zero by $r$.
            \item From line 2 to line 3, we reindex some terms on the left from $j\to j+1$.
        \end{itemize}
        \item It follows just like last class that
        \begin{equation*}
            a_{j+1}(j+1)[j+2(\ell+1)] = a_j\left[ 2k_{n\ell}j+2k_{n\ell}(\ell+1)-\frac{2}{a_\text{B}} \right]
        \end{equation*}
        \item Thus, we get that
        \begin{equation*}
            a_{j+1} = \frac{k_{n\ell}(2j+2\ell+2)-\frac{2}{a_\text{B}}}{(j+1)[j+2(\ell+1)]}a_j
        \end{equation*}
        \item Once again, for similar reasons, there will also be some $j_\text{max}=n$.
        \item Then
        \begin{equation*}
            (n+\ell+1)k_{n\ell} = \frac{1}{a_\text{B}}
        \end{equation*}
        which means that
        \begin{equation*}
            k_{n\ell} = \frac{1}{a_\text{B}(n+\ell+1)}
        \end{equation*}
        \item Then we get that
        \begin{equation*}
            E_{n\ell} = -\frac{\hbar^2}{2m_ea_\text{B}^2(n+\ell+1)^2}
        \end{equation*}
        where everything except the quantum numbers is the \textbf{Rydberg constant}.
        \item Consequently, in this case, we may define $\bar{n}=n+\ell+1$.
    \end{itemize}
    \item \textbf{Bound state}: A state in which the electron can escape to $\infty$.
    \begin{itemize}
        \item We tend to suppress bound states so that the wave function does not have probability at $\infty$.
    \end{itemize}
    \item \textbf{Electromagnetic fine structure constant}: The constant defined as follows. \emph{Denoted by} $\bm{\alpha}$. \emph{Given by}
    \begin{equation*}
        \alpha = \frac{e^2}{4\pi\epsilon_0\hbar c} \approx \frac{1}{137}
    \end{equation*}
    \item \textbf{Bohr radius}: The most probable distance from the nucleus of a hydrogen atom for its electron to exist. \emph{Denoted by} $\bm{a_\textbf{B}}$. \emph{Given by}
    \begin{equation*}
        a_\text{B} = \frac{4\pi\epsilon_0\hbar^2}{m_ee^2}
        \approx \frac{137}{2\pi}\lambda_c
        = \SI{5.3e-11}{\meter}
    \end{equation*}
    \begin{itemize}
        \item Note that we get the approximation from the aside's note that
        \begin{equation*}
            a_\text{B}^{-1} = \lambda_c^{-1}2\pi\alpha
        \end{equation*}
    \end{itemize}
    \item \textbf{Rydberg constant}: The constant defined as follows. \emph{Denoted by} $\bm{\Ry}$. \emph{Given by}
    \begin{equation*}
        \Ry = \frac{\hbar^2}{2m_ea_\text{B}^2} = \SI{13.6}{\electronvolt}
    \end{equation*}
    \item Wagner is Argentenian.
    \item We'll continue on Monday.
\end{itemize}




\end{document}