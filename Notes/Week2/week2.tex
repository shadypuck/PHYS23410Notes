\documentclass[../notes.tex]{subfiles}

\pagestyle{main}
\renewcommand{\chaptermark}[1]{\markboth{\chaptername\ \thechapter\ (#1)}{}}
\stepcounter{chapter}

\begin{document}




\chapter{The Schr\"{o}dinger Equation}
\section{Ehrenfest Theorem and Uncertainty Principle}
\begin{itemize}
    \item \marginnote{1/8:}Announcement: PSet 1 due Friday at midnight.
    \item Recap.
    \begin{itemize}
        \item $\psi(\vec{r},t)$ is a wave function to which we associate a \textbf{probability density}.
        \begin{itemize}
            \item Integrating this probability density over a volume yields the probability that the particle is in $V$.
            \item Moreover, $\psi$ is not arbitrary but must satisfy the Schr\"{o}dinger equation.
        \end{itemize}
        \item $\hat{\vec{p}}$ is the momentum operator, defined as the differential operator $-i\hbar\vec{\nabla}$.
        \item Expressing the Schr\"{o}dinger equation in terms of $\hat{\vec{p}}$, we see that it represents the application of a Hamiltonian operator in the usual form from last quarter (i.e., kinetic plus potential energy) to a certain function.
        \item $\Exp{\hat{\vec{r}}}$ is the mean position, and $\Exp{\hat{\vec{p}}}$ is the mean momentum.
        \begin{itemize}
            \item The mean position and mean momentum satisfy the classical relation, i.e., $\dd{\Exp{\hat{\vec{r}}}}/\dd{t}=\Exp{\hat{\vec{p}}}/m$.
        \end{itemize}
    \end{itemize}
    \item \textbf{Probability density}: The quantity given as follows. \emph{Given by}
    \begin{equation*}
        |\psi(\vec{r},t)|^2
    \end{equation*}
    \item We now prove something even more amazing than the classical relation result: An analogy to the classical Newton's law.
    \item \textbf{Ehrenfest's theorem}: The time derivative of the expectation value of the momentum operator is related to the expectation value of the force $F:=-\vec{\nabla}V$ on a massive particle moving in a scalar potential $V(\vec{r},t)$ as follows.
    \begin{equation*}
        \dv{\Exp{\hat{\vec{p}}}}{t} = \Exp{-\vec{\nabla}V(\vec{r},t)}
    \end{equation*}
    \begin{proof}
        Consider the Schr\"{o}dinger equation:
        \begin{equation*}
            -i\hbar\pdv{\psi}{t} = \frac{\hbar^2}{2m}\vec{\nabla}^2\psi-V(\vec{r},t)\psi
        \end{equation*}
        Take the complex conjugate of it. This means that we're sending $i\mapsto -i$, keeping $V$ fixed (it's real), and sending $\psi\mapsto\psi^*$ (the inclusion of $i$ in the Schr\"{o}dinger equation means that $\psi$ is complex in general and thus has a nontrivial complex conjugate).
        \begin{equation*}
            -i\hbar\pdv{\psi^*}{t} = -\frac{\hbar^2}{2m}\vec{\nabla}^2\psi^*+V(\vec{r},t)\psi^*
        \end{equation*}
        We will use the above two equations to substitute into the following algebraic derivation.
        \begin{align*}
            \dv{\Exp{\hat{\vec{p}}}}{t} ={}& \dv{t}(\int\dd^3\vec{r}\ \psi^*(-i\hbar\vec{\nabla}\psi))\\
            ={}& \int\dd^3\vec{r}\ {\pdv{\psi^*}{t}}(-i\hbar\vec{\nabla}\psi)+\int\dd^3\vec{r}\ \psi^*\left( -i\hbar\vec{\nabla}\pdv{\psi}{t} \right)\\
            ={}& \int\dd^3\vec{r}\,\left[ -i\hbar{\pdv{\psi^*}{t}}(\vec{\nabla}\psi) \right]+\int\dd^3\vec{r}\ \psi^*\vec{\nabla}\left( -i\hbar\pdv{\psi}{t} \right)\\
            \begin{split}
                ={}& \int\dd^3\vec{r}\,\left[ -\frac{\hbar^2}{2m}\vec{\nabla}^2\psi^*(\vec{\nabla}\psi) \right]+\int\dd^3\vec{r}\ \psi^*\vec{\nabla}\left( \frac{\hbar^2}{2m}\vec{\nabla}^2\psi \right)\\
                & +\int\dd^3\vec{r}\,\left[ V(\vec{r},t)\psi^*(\vec{\nabla}\psi)+\psi^*\vec{\nabla}(-V(\vec{r},t)\psi) \right]
            \end{split}\\
            ={}& \int\dd^3\vec{r}\ \psi^*\vec{\nabla}(-V(\vec{r},t)\psi)\\
            ={}& \int\dd^3\vec{r}\ \psi^*(-\vec{\nabla}V(\vec{r},t))\psi\\
            ={}& \Exp{-\vec{\nabla}V(\vec{r},t)}
        \end{align*}
        as desired.
    \end{proof}
    \item How does everything cancel from the long line to the following line in the above proof??
    \item In quantum mechanics, we have \textbf{observables} which are in one-to-one correspondence with operators.
    \begin{table}[h!]
        \centering
        \small
        \renewcommand{\arraystretch}{1.4}
        \begin{tabular}{c|c}
            \textbf{Observables} & \textbf{Operators ($\bm{\hat{O}}$)}\\
            \hline
            $\vec{r}$ & $\hat{\vec{r}}$\\
            $V(\vec{r},t)$ & $\hat{V}(\vec{r},t)$\\
            $\hat{\vec{p}}$ & $-i\hbar\vec{\nabla}$\\
            $\hat{H}$ & $-\dfrac{\hbar^2}{2m}\vec{\nabla}^2+V(\vec{r},t)$\\
        \end{tabular}
        \caption{Observables vs. operators.}
        \label{tab:observablesOperators}
    \end{table}
    \begin{itemize}
        \item Recall that any Hermitian operator has a real observable.
    \end{itemize}
    \item Define
    \begin{equation*}
        \hat{O}_{ij} := \int\dd^3\vec{r}\ \psi_i^*\hat{O}\psi_j
    \end{equation*}
    \begin{itemize}
        \item Then note that
        \begin{equation*}
            \hat{O}_{ij} = (\hat{O}_{ji})^*
        \end{equation*}
        \item Thus, an equivalent definition of a Hermitian operator is one such that the above equation is satisfied for all relevant $i,j$.
    \end{itemize}
    \item Recall that the Schr\"{o}dinger equation is linear.
    \begin{itemize}
        \item Let $\psi=\sum_ic_i\psi_i$.
        \item Then
        \begin{equation*}
            \int\dd^3\vec{r}\ \psi^*\hat{O}\psi = \sum_{i,j}\int\dd^3\vec{r}\ c_i^*\psi_i^*\hat{O}c_j\psi_j
            = \sum_{i,j}c_i^*c_j\hat{O}_{ij}
        \end{equation*}
        is real.
        \item Takeaway: Averages over arbitrary wavefunctions are real.
        \item Similarly, suppose that $\vec{r}$ is Hermitian. Then any function $V(\vec{r})$ of it is also Hermitian.
        \item Once again,
        \begin{equation*}
            \int\dd^3\vec{r}\ \psi_i^*(-i\hbar\vec{\nabla}\psi_j) = \left( \int\dd^3\vec{r}\ \psi_j^*(-i\hbar\vec{\nabla}\psi_i) \right)^*
            = \int\dd^3\vec{r}\ \psi_j(i\hbar\vec{\nabla}\psi_i^*)
            \to -\int\dd^3\vec{r}\ \vec{\nabla}\psi_j(i\hbar\psi_i^*)
        \end{equation*}
        Involves integration by parts?? Perhaps via
        \begin{align*}
            \int\dd^3\vec{r}\ \psi_j(i\hbar\vec{\nabla}\psi_i^*) &= i\hbar\int\dd^3\vec{r}\ \vec{\nabla}(\psi_j\psi_i^*)-\int\dd^3\vec{r}\ \vec{\nabla}\psi_j(i\hbar\psi_i^*)\\
            &= i\hbar\vec{\nabla}\int\dd^3\vec{r}\ (\psi_j\psi_i^*)-\int\dd^3\vec{r}\ \vec{\nabla}\psi_j(i\hbar\psi_i^*)\\
            &= i\hbar\vec{\nabla}0-\int\dd^3\vec{r}\ \vec{\nabla}\psi_j(i\hbar\psi_i^*)\\
            &= -\int\dd^3\vec{r}\ \vec{\nabla}\psi_j(i\hbar\psi_i^*)
        \end{align*}
        What is the takeaway??
    \end{itemize}
    \item Linear algebra analogy.
    \begin{itemize}
        \item Recall that we can write any vector $\vec{v}$ componentwise as $\vec{v}=v_x\vec{x}+v_y\vec{y}+v_z\vec{z}$.
        \item We can apply matrices $A$ to such vectors to generate other vectors via $A\vec{v}=\vec{v}{\,}'$ and the like.
        \item Lastly, we have an inner product $\cdot$ such that $\vec{a}\cdot\vec{b}=\delta_{ab}$, where $a,b=x,y,z$.
        \item On an infinite-dimensional vector space, such as that containing all the $\psi$, we still can decompose $\psi=\sum_nc_n\psi_n$ into an infinite sum of basis components, apply operators $\hat{O}\psi=\psi'$, and have an inner product $\int\dd^3\vec{r}\ \psi^*_m\psi_n=\delta_{mn}$.
        \item Another analogy: Like the inner product of a vector and unit vector is the component of the vector in that direction (e.g., $\vec{v}\cdot\vec{x}=v_x$), we have
        \begin{equation*}
            \int\dd^3\vec{r}\ \psi_m^*\psi = \int\dd^3\vec{r}\psi_m^*\sum_nc_n\psi_n = c_m
        \end{equation*}
        \item One more analogy: $\vec{x}^TA\vec{x}=A_{xx}$ is like $\ev{\hat{O}}{\psi_i}=\hat{O}_{ii}$.
    \end{itemize}
\end{itemize}




\end{document}