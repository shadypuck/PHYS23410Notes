\documentclass[../notes.tex]{subfiles}

\pagestyle{main}
\renewcommand{\chaptermark}[1]{\markboth{\chaptername\ \thechapter\ (#1)}{}}
\setcounter{chapter}{7}

\begin{document}




\chapter{The Hydrogen Atom}
\section{Hydrogen Atom II}
\begin{itemize}
    \item \marginnote{2/19:}Review of the hydrogen atom.
    \begin{itemize}
        \item The potential is given by
        \begin{equation*}
            V(r) = -\frac{e^2}{4\pi\epsilon_0r}
        \end{equation*}
        \begin{itemize}
            \item This is an important case of motion in a central potential in quantum mechanics.
        \end{itemize}
        \item We go to polar coordinates because they are most convenient for motion in a central potential.
        \item We achieve separation of variables via
        \begin{equation*}
            \psi_{n\ell m}(r,\theta,\phi) = R_{n\ell}(r)Y_{\ell m}(\theta,\phi)
        \end{equation*}
        \item This leads into the spherical harmonics
        \begin{align*}
            \hat{\vec{L}}{\,}^2Y_{\ell m}(\theta,\phi) &= \hbar^2\ell(\ell+1)Y_{\ell m}(\theta,\phi)\\
            \hat{L}_zY_{\ell m}(\theta,\phi) &= \hbar mY_{\ell m}(\theta,\phi)
        \end{align*}
        \begin{itemize}
            \item Additionally, the quantum number $m$ satisfies $-\ell\leq m\leq\ell$, giving us $2\ell+1$ solutions for each $\ell$.
        \end{itemize}
        \item With the spherical harmonics, the main question becomes how to find $R_{n\ell}$.
        \begin{itemize}
            \item We do this via the change of variables
            \begin{equation*}
                U_{n\ell}(r) = rR_{n\ell}(r)
            \end{equation*}
            yielding a function that satisfies the analogous one-dimensional effective system
            \begin{equation*}
                -\frac{\hbar^2}{2M}\dv[2]{r}[U_{n\ell}(r)]+\underbrace{\left[ V(r)+\frac{\hbar^2\ell(\ell+1)}{2Mr^2} \right]}_{V_\text{eff}(r)}U_{n\ell}(r) = E_{n\ell}U_{n\ell}(r)
            \end{equation*}
        \end{itemize}
        \item We analyze such systems using their asymptotic behavior as $r\to 0$ and $r\to\infty$.
        \begin{itemize}
            \item See Figure \ref{fig:1HV}. We are looking for bound states $E_{n\ell}$.
            \item When the energy is positive, we have continuous solutions; it's only when the energy is negative that we have quantized bound states.
        \end{itemize}
        \item Performing such analyses, we propose an ansatz
        \begin{equation*}
            U_{n\ell}(r) = f_{n\ell}(r)r^{\ell+1}\e[-k_{n\ell}r]
        \end{equation*}
        where
        \begin{equation*}
            E_{n\ell} = -\frac{\hbar^2k_{n\ell}^2}{2M}
        \end{equation*}
        \item We suppose that $f$ is a polynomial function
        \begin{equation*}
            f_{n\ell}(r) = \sum_ja_jr^j
        \end{equation*}
        and solve for it using the differential equation
        \begin{equation*}
            f_{n\ell}''(r)+f_{n\ell}'(r)\left[ \frac{2(\ell+1)}{r}-2k_{n\ell} \right]+f_{n\ell}(r)\left[ -\frac{2k_{n\ell}(\ell+1)}{r}+\frac{2}{a_\text{B}r} \right] = 0
        \end{equation*}
        \item This analysis leads us to the recursion relation
        \begin{equation*}
            a_{j+1} = \frac{2k_{n\ell}(j+\ell+1)-\frac{2}{a_\text{B}}}{(j+1)(j+2\ell+2)}a_j
        \end{equation*}
        \item We then choose $j_\text{max}=N$, yielding
        \begin{equation*}
            k_{n\ell} = \frac{1}{a_\text{B}\underbrace{(N+\ell+1)}_n}
        \end{equation*}
        where we canonically call $n$ the \textbf{principal quantum number}.
        \begin{itemize}
            \item This is a divergence from last time's notation, but one made for good reason, as we will see shortly.
            \item Note that we did not introduce this notation last time because we didn't want to have to discuss its subtleties then, as we will today.
        \end{itemize}
        \item Thus, the energy depends only on this value $n$ via
        \begin{equation*}
            E_{n\ell} = -\frac{\hbar^2}{2m_ea_\text{B}^2(N+\ell+1)^2} = -\frac{\Ry}{n^2}
        \end{equation*}
        where $\Ry$ is the Rydberg constant defined last time.
        \begin{itemize}
            \item Essentially, the energy depends on this value $n$ which in turn has a hidden dependence on $\ell$.
        \end{itemize}
    \end{itemize}
    \item We now begin on new content, continuing from above however.
    \item The energy spacing versus $n$.
    \begin{figure}[h!]
        \centering
        \begin{tikzpicture}[
            every node/.style=black
        ]
            \footnotesize
    
            \fill [grx] (0.5,0) rectangle (1.5,0.5);
            \node [right] at (1.5,0.25) {Continuous};
    
            \draw [grx,very thick]
                (0.5,-0.03125) -- ++(1,0)
                (0.5,-0.0625) -- ++(1,0)
                (0.5,-0.125) -- ++(1,0)
                (0.5,-0.25) -- ++(1,0)
                (0.5,-0.5) -- ++(1,0)
                (0.5,-1) -- ++(1,0) node[right]{$E_2=E_1/4$}
                (0.5,-2) -- ++(1,0) node[right]{$E_1=-\SI{13.6}{\electronvolt}$}
            ;
    
            \draw [stealth-stealth] (-0.7,0) -- (3.5,0);
            \draw [stealth-stealth] (0,0.6) -- (0,-2.5);
            \node [above left] {$E_n$};
    
            \draw [semithick,decorate,decoration=brace] (-1,0) -- node[left=1mm]{$E>0$} (-1,0.5);
            \draw [semithick,decorate,decoration=brace] (-1,-2.5) -- node[left=1mm]{$E<0$} (-1,0);
        \end{tikzpicture}
        \caption{\ce{{}^1H} energy spacing vs. $n$.}
        \label{fig:1HEn}
    \end{figure}
    \begin{itemize}
        \item The energies get closer and closer together as $n$ increases until they become continuous for positive values of energy.
        \item The equations and figure imply that $-E_1=\Ry$ is the minimum energy necessary to remove the electron from the hydrogen atom.
        \item If we don't have this much energy, a lesser amount will still affect the electron, just moving it to an \textbf{excited state}.
        \begin{itemize}
            \item In particular, $E_m-E_1$ is the amount of energy necessary to move the electron to an excited state $E_m$ of higher energy than $E_1$.
        \end{itemize}
        \item What is also interesting is that if the electron is in an excited state of energy $E_m$, then it will \emph{not} stay there forever.
        \begin{itemize}
            \item Experimentally, even in this time independent potential, the electron can jump back down to a lower state by emitting electromagnetic radiation of energy $E_\gamma=E_m-E_1$.
            \item This is evidence that the vacuum in which we assume the hydrogen atom lies is not really \emph{vacuum}! Rather, the vacuum contains something called the EM field, and there are fluctuations in this EM field that can push the electron down energy states.
            \item This is discussed more in Quantum Mechanics II, but is ignored in our present formalism of the hydrogen atom.
        \end{itemize}
    \end{itemize}
    \item Now for every fixed $n$, the equation $n=N+\ell+1$ implies that $\ell=0,1,\dots,n-1$.
    \begin{itemize}
        \item But for every $\ell$, there are $2\ell+1$ solutions with the same energy.
        \item Thus, for every $n$, there are
        \begin{equation*}
            \sum_{\ell=0}^{n-1}(2\ell+1) = 2\sum_{\ell=0}^{n-1}\ell+\sum_{\ell=0}^{n-1}1
            = 2\cdot\frac{n(n-1)}{2}+n
            = n^2
        \end{equation*}
        different states with the same energy.
    \end{itemize}
    \item Aside: A fun application of this stuff to cosmology.
    \begin{itemize}
        \item The universe started as a hot plasma that cooled down as the universe expanded.
        \item The early universe contained a lot of crap, including photons.
        \item When early protons and electrons tried to combine at hot temperatures, the huge amount of EM radiation would kick the electrons out.
        \item Thus, stable atoms could not form.
        \item The specific temperatures at which this would occur were
        \begin{equation*}
            k_\text{B}T > \SI{13.6}{\electronvolt}
        \end{equation*}
        \item At temperatures $k_\text{B}T<\SI{13.6}{\electronvolt}$, protons and electrons bind together, and the universe becomes transparent to radiation.
        \item Evidence that this happened: Cosmic microwave background.
        \begin{itemize}
            \item When the universe became transparent, it was at microwave temperatures.
            \item This was when the universe was about \num{13000} years old.
        \end{itemize}
    \end{itemize}
    \item Now back to math.
    \item Since we now have an explicit definition for $k_{n\ell}$, we may rewrite the solutions as
    \begin{equation*}
        U_{n\ell}(r) = f_{n\ell}(r)r^{\ell+1}\e[-r/a_\text{B}n]
    \end{equation*}
    \begin{itemize}
        \item Notationally, do remember that $n$ gives energy, $\ell$ gives angular momentum, and $N=n-\ell-1$ gives the polynomial degree of $f_{n\ell}(r)$.
    \end{itemize}
    \item Thus, if $N=0$, then $n=\ell+1$ and the radial probability density of finding the particle at a given $r$ is
    \begin{equation*}
        r^2|R_{n\ell}(r)|^2 = |U_{n\ell}(r)|^2 = r^{2n}\e[-2r/a_\text{B}n]
    \end{equation*}
    \item What is the maximum, i.e., the most probable distance from the nucleus?
    \begin{itemize}
        \item Differentiate the probability density with respect to $r$ and determine where it equals zero.
        \begin{align*}
            0 &= 2nr^{2n-1}\e[-2r/a_\text{B}n]-\frac{2r^{2n}}{a_\text{B}n}\e[-2r/a_\text{B}n]\\
            \frac{2r^{2n}}{a_\text{B}n} &= 2nr^{2n-1}\\
            r_\text{max} &= a_\text{B}n^2
        \end{align*}
    \end{itemize}
    \item What happens if you have an ion of charge $Ze$?
    \begin{itemize}
        \item Then
        \begin{equation*}
            V(r) = -\frac{Ze^2}{4\pi\epsilon_0r}
        \end{equation*}
        \item Thus,
        \begin{equation*}
            E_n = -\frac{\Ry Z^2}{n^2}
        \end{equation*}
        and the Bohr radius halves.
    \end{itemize}
\end{itemize}



\section{Office Hours (Yunjia)}
\begin{itemize}
    \item \marginnote{2/20:}PSet 6, Q1a: Can I leave the solutions in terms of $U_{n\ell}$, or do I need to further manipulate them?
    \begin{itemize}
        \item Since $U_{n\ell}$ is defined in the question, it's perfectly fine to leave the answer in terms of it.
    \end{itemize}
    \item PSet 6, Q1a: Do I need to do anything for the "observe\dots" part of the question? It almost sounds like I need to consider some nonzero values of $\ell$ here; if not, what else could "each value of $\ell$" refer to?
    \begin{itemize}
        \item $\ell=0$ is the only value of $\ell$ we need to discuss in this part of this problem.
    \end{itemize}
    \item PSet 6, Q1c: Is all we have to do start with the radial equation, turn it into the spherical Bessel equation, and do the variable substitution?
    \begin{itemize}
        \item Yes.
    \end{itemize}
    \item PSet 6, Q1d: Can you differentiate with respect to $k_{n1}r$, or do we need to do a $u$-substitution type thing and introduce a $\dv*{u}{r}$ term?
    \begin{itemize}
        \item We can differentiate with respect to $k_{n1}r$.
        \item This yields the function we see on Wikipedia.
    \end{itemize}
    \item PSet 6, Q1d: Do I need to do anything else with the boundary condition other than just commenting on it?
    \begin{itemize}
        \item No.
    \end{itemize}
    \item PSet 6, Q1e: Do I need to do anything else here?
    \begin{itemize}
        \item Say something about the probability density getting larger at the origin vs. going to zero at the origin.
    \end{itemize}
    \item PSet 6, Q2a: Do you want the $\bar{n}=0$ solutions or what else?
    \begin{itemize}
        \item Give the first few examples ($\bar{n}=0,1,2$) and then say that the pattern continues.
        \item There is a proof that the pattern continues for all $\bar{n}$, but we are neither expected to produce it nor research/learn it.
    \end{itemize}
    \item PSet 6, Q2b: Just copy what he wrote on 2/14 for the $\bar{n}=1$ case and do a bit of algebra to prove linear combinations?
    \begin{itemize}
        \item Yes.
    \end{itemize}
    \item PSet 6, Q2c: Just describe quantitatively the shape of the functions in 3D?
    \begin{itemize}
        \item Yes.
    \end{itemize}
    \item PSet 6, Q3b: What is this variable $z$ and why does the $n\ell$ subscript on $f$ sometimes disappear?
    \begin{itemize}
        \item $z:=k_{n\ell}r$ is a variable defined to be unitless. It rigorously justifies the fact that we can do all of our differential equation manipulations abstractly in the context of pure mathematics without having to worry about units.
        \item The $n\ell$ subscript's occasional disappearance is just an abuse of notation (and another indication that we can treat $f$ purely as a mathematical function).
    \end{itemize}
    \item PSet 6, Q3c: How do I "demonstrate that $q-w-1=n$ must be a positive integer (or zero)" and how does this help imply the final energy equation?
    \begin{itemize}
        \item The relevant note is included in the problem statement purely to make it clear that the group of numbers $n+w+1$ can be treated as a single quantity and does not have to be binomially expanded.
        \item Include some note about this in your answer, and you'll be all good to go!
    \end{itemize}
\end{itemize}




\end{document}