\documentclass[../notes.tex]{subfiles}

\pagestyle{main}
\renewcommand{\chaptermark}[1]{\markboth{\chaptername\ \thechapter\ (#1)}{}}
\setcounter{chapter}{3}

\begin{document}




\chapter{Observables and Hermitian Operators}
\section{Harmonic Oscillator: Raising and Lowering Operators}
\begin{itemize}
    \item \marginnote{1/22:}\textbf{Raising operator}: The operator defined as follows. \emph{Denoted by} $\bm{\hat{a}_+}$, $\bm{a_+}$. \emph{Given by}
    \begin{equation*}
        \hat{a}_+ = \frac{1}{\sqrt{2\hbar m\omega}}[-i\hat{\vec{p}}+m\omega\hat{\vec{x}}\,]
    \end{equation*}
    \item \textbf{Lowering operator}: The operator defined as follows. \emph{Denoted by} $\bm{\hat{a}_-}$, $\bm{a_-}$. \emph{Given by}
    \begin{equation*}
        \hat{a}_- = \frac{1}{\sqrt{2\hbar m\omega}}[i\hat{\vec{p}}+m\omega\hat{\vec{x}}\,]
    \end{equation*}
    \item \textbf{Number operator}: The operator defined as follows. \emph{Denoted by} $\bm{a_+a_-}$. \emph{Given by}
    \begin{equation*}
        a_+a_- = \hat{a}_+\circ\hat{a}_-
        = \frac{1}{2\hbar m\omega}\left[ \hat{\vec{p}}{\,}^2+m^2\omega^2x^2-im\omega[\hat{\vec{p}},\hat{\vec{x}}] \right]
    \end{equation*}
    \item Properties of these operators.
    \begin{itemize}
        \item We can express $\hat{\vec{p}},\hat{\vec{x}}$ in terms of $a_+,a_-$ via
        \begin{align*}
            \hat{\vec{p}} &= i\sqrt{\frac{\hbar m\omega}{2}}(\hat{a}_+-\hat{a}_-)&
            \hat{\vec{x}} &= \sqrt{\frac{\hbar}{2m\omega}}(\hat{a}_++\hat{a}_-)
        \end{align*}
        \begin{itemize}
            \item It follows that
            \begin{equation*}
                [\hat{\vec{p}},\hat{\vec{x}}] = \frac{i\hbar}{2}[a_+-a_-,a_++a_-]
                = \frac{i\hbar}{2}([a_+,a_-]-[a_-,a_+])
                = i\hbar[a_+,a_-]
            \end{equation*}
            \item Consequently, since $[\hat{\vec{p}},\hat{\vec{x}}]=-i\hbar$, we have that
            \begin{equation*}
                [a_+,a_-] = -1
            \end{equation*}
            \item We also have that
            \begin{equation*}
                [a_-,a_+] = 1
            \end{equation*}
        \end{itemize}
        \item Since $[\hat{\vec{p}},\hat{x}]=-i\hbar$ and $\omega^2=k/m$, we have that
        \begin{align*}
            a_+a_- &= \frac{1}{2\hbar m\omega}\left[ \hat{\vec{p}}{\,}^2+m^2\omega^2x^2-m\hbar\omega \right]\\
            &= \frac{1}{\hbar\omega}\Bigg[ \underbrace{\frac{\hat{\vec{p}}{\,}^2}{2m}+\frac{kx^2}{2}}_{\hat{H}}-\frac{\hbar\omega}{2} \Bigg]\\
            \hat{H} &= \hbar\omega\left( a_+a_-+\frac{1}{2} \right)
        \end{align*}
        \begin{itemize}
            \item Because of the properties of $[a_+,a_-]$ proven above, we similarly have that
            \begin{equation*}
                \hat{H} = \hbar\omega\left( a_-a_+-\frac{1}{2} \right)
            \end{equation*}
            \item We can also derive this equation in a manner exactly analogous to the first one.
        \end{itemize}
    \end{itemize}
    \item How does the number operator act on the eigenstate $\ket{\psi_n}$ of the harmonic oscillator?
    \begin{itemize}
        \item Since $E_n=\hbar\omega(n+1/2)$, we have that
        \begin{align*}
            \hbar\omega\left( a_+a_-+\frac{1}{2} \right)\ket{\psi_n} &= \hat{H}\ket{\psi_n}\\
            \hbar\omega\left( a_+a_-+\frac{1}{2} \right)\ket{\psi_n} &= \hbar\omega\left( n+\frac{1}{2} \right)\ket{\psi_n}\\
            a_+a_-\ket{\psi_n} &= n\ket{\psi_n}
        \end{align*}
    \end{itemize}
    \item How do the raising and lowering operators act on the eigenstate $\ket{\psi_n}$ of the harmonic oscillator?
    \begin{itemize}
        \item Using a number of the above substitutions, we have that
        \begin{align*}
            \hat{H}(a_+\ket{\psi_n}) &= \left[ \hbar\omega\left( a_+a_-+\frac{1}{2} \right) \right](a_+\ket{\psi_n})\\
            &= \hbar\omega\left( a_+a_-a_++\frac{1}{2}a_+ \right)\ket{\psi_n}\\
            &= \hbar\omega a_+\left( a_-a_++\frac{1}{2} \right)\ket{\psi_n}\\
            &= \hbar\omega a_+\left( a_+a_-+1+\frac{1}{2} \right)\ket{\psi_n}\\
            &= \hbar\omega a_+\left( n+1+\frac{1}{2} \right)\ket{\psi_n}\\
            &= E_{n+1}(a_+\ket{\psi_n})
        \end{align*}
        \item This means that $\hat{H}$ acts on $a_+\ket{\psi_n}$ the same way it acts on $\ket{\psi_{n+1}}$. In other words, it must be that
        \begin{equation*}
            a_+\ket{\psi_n} \propto \ket{\psi_{n+1}}
        \end{equation*}
        \item Similarly,
        \begin{equation*}
            \hat{H}(a_-\ket{\psi_n}) = E_{n-1}(a_-\ket{\psi_n})
        \end{equation*}
        so
        \begin{equation*}
            a_-\ket{\psi_n} \propto \ket{\psi_{n-1}}
        \end{equation*}
        \item These actions are why $a_+,a_-$ are called the \emph{raising} and \emph{lowering} operators!
        \item We now seek to determine the constants of proportionality.
        \item First off, note that $a_+$ and $a_-$ are adjoints, i.e.,
        \begin{equation*}
            a_+^\dagger = a_-
        \end{equation*}
        \begin{itemize}
            \item See Section 2.3 of \textcite{bib:Griffiths} for a proof of this fact.
        \end{itemize}
        \item Then for $a_+$, we know that if
        \begin{align*}
            a_+\ket{\psi_n} = c_+\ket{\psi_n}
        \end{align*}
        then
        \begin{align*}
            c_+^2 &= c_+^2\braket{\psi_{n+1}}\\
            &= \braket{c_+\psi_{n+1}}\\
            &= \braket{a_+\psi_n}\\
            &= \ev{a_+^\dagger a_+}{\psi_n}\\
            &= \ev{a_-a_+}{\psi_n}\\
            &= \ev{a_+a_-+1}{\psi_n}\\
            &= (n+1)\braket{\psi_n}\\
            &= n+1
        \end{align*}
        so that, taking square roots,
        \begin{equation*}
            c_+ = \sqrt{n+1}
        \end{equation*}
        \item By the same method --- namely
        \begin{equation*}
            c_-^2 = \braket{a_-\psi_n}
            = \ev{a_+a_-}{\psi_n}
            = n
        \end{equation*}
        we can also learn that
        \begin{equation*}
            c_- = \sqrt{n}
        \end{equation*}
        \item Therefore,
        \begin{align*}
            a_+\ket{\psi_n} &= \sqrt{n+1}\ket{\psi_{n+1}}&
            a_-\ket{\psi_n} &= \sqrt{n}\ket{\psi_{n-1}}
        \end{align*}
        \item Note that what we have done here to derive this fact is far more slick than working directly with the unintuitive and complicated formal definitions of $a_+,a_-$.
    \end{itemize}
    \item Now is a good time to mention a bit more about Dirac notation.
    \begin{itemize}
        \item A "ket" represents a vector in a Hilbert space, so $\ket{\psi_n}$ demonstrates that we are talking about the wave function as a vector in the abstract linear algebra sense, not as a function $\psi_n:\R^4\to\C$.
        \item A "bra" represents a linear functional on a Hilbert space. In quantum mechanics, the linear functional $\bra{\eta}$ is given by
        \begin{equation*}
            \bra{\eta} := \int\dd^3\vec{r}\ \eta^*
        \end{equation*}
        \item Observe that this "functional" does indeed map any $\ket{\psi_n}$ given to it as an argument to a number $\braket{\eta}{\psi_n}$!
    \end{itemize}
    \item $\ket{\psi_n}$ can be defined in terms of $a_+$, $\ket{\psi_0}$, and constants.
    \begin{itemize}
        \item Observe that since $a_+\ket{\psi_0}=\ket{\psi_1}$ and $a_+\ket{\psi_1}=\sqrt{2}\ket{\psi_2}$, we have that
        \begin{equation*}
            \ket{\psi_2} = \frac{a_+}{\sqrt{2}}\ket{\psi_1}
            = \frac{a_+^2}{\sqrt{2}}\ket{\psi_0}
        \end{equation*}
        \item Similarly,
        \begin{equation*}
            \ket{\psi_3} = \frac{a_+}{\sqrt{3}}\ket{\psi_2}
            = \frac{a_+^3}{\sqrt{3\cdot 2}}\ket{\psi_0}
        \end{equation*}
        \item Generalizing, we have that
        \begin{equation*}
            \ket{\psi_n} = \frac{a_+^n}{\sqrt{n!}}\ket{\psi_0}
        \end{equation*}
    \end{itemize}
    \item Thus, we have that
    \begin{equation*}
        \psi_n(x) = \left( \frac{1}{\sqrt{2\hbar m\omega}} \right)^n\frac{1}{\sqrt{n!}}\left( -\hbar\dv{x}+xm\omega \right)^n\psi_0(x)
    \end{equation*}
    where we may recall that
    \begin{equation*}
        \psi_0(x) = \left( \frac{m\omega}{\hbar\pi} \right)^{1/4}\e[-m\omega x^2/2\hbar]
    \end{equation*}
    \item Final observations about the raising and lowering operators.
    \begin{itemize}
        \item Since $a_-\ket{\psi_0}=0$ (as we may readily verify by direct computation), we have that
        \begin{equation*}
            \hbar\dv{\psi_0}{x}+m\omega x\psi_0 = 0
        \end{equation*}
        \item We also know that
        \begin{equation*}
            \dd(\ln(\psi_0)) = -\frac{m\omega}{\hbar}\frac{\dd{x}^2}{2}
        \end{equation*}
        so
        \begin{equation*}
            \psi_0 \propto \e[-m\omega x^2/2\hbar]
        \end{equation*}
        \begin{itemize}
            \item What is the point of this line?? What new information does it give us?
        \end{itemize}
    \end{itemize}
    \item Raising and lowering operators allow us to compute the kinetic and potential energy of the harmonic oscillator.
    \begin{itemize}
        \item Kinetic energy.
        \begin{align*}
            \ev**{\frac{\hat{\vec{p}}{\,}^2}{2m}}{\psi_n} &= -\frac{\hbar\omega}{4}\ev{(a_+-a_-)^2}{\psi_n}\\
            &= -\frac{\hbar\omega}{4}\ev{a_+^2+a_-^2-a_+a_--a_-a_+}{\psi_n}\\
            % &= -\frac{\hbar\omega}{4}\left[ \ev{a_+^2}{\psi_n}+\ev{a_-^2}{\psi_n}-2\ev{a_+a_-}{\psi_n}-\ev{1}{\psi_n} \right]\\
            &= -\frac{\hbar\omega}{4}\big[ \underbrace{\ev{a_+^2}{\psi_n}}_{\propto\,\braket{\psi_n}{\psi_{n+2}}}+\underbrace{\ev{a_-^2}{\psi_n}}_{\propto\,\braket{\psi_n}{\psi_{n-2}}}-\underbrace{2\ev{a_+a_-}{\psi_n}}_{2n\braket{\psi_n}}-\underbrace{\ev{1}{\psi_n}}_{\braket{\psi_n}} \big]\\
            &= \frac{\hbar\omega}{4}(2n+1)\\
            &= \frac{\hbar\omega}{2}\left( n+\frac{1}{2} \right)\\
            &= \frac{E_n}{2}
        \end{align*}
        \item Potential energy.
        \begin{align*}
            \ev{\hat{H}}{\psi_n} &= E_n\\
            \ev**{\frac{\hat{\vec{p}}{\,}^2}{2m}}{\psi_n}+\ev**{\frac{k\hat{\vec{x}}{\,}^2}{2}}{\psi_n} &= \frac{E_n}{2}+\frac{E_n}{2}\\
            \ev**{\frac{k\hat{\vec{x}}{\,}^2}{2}}{\psi_n} &= \frac{E_n}{2}
        \end{align*}
        \item Implication: In an energy eigenstate, the harmonic oscillator has equal values of kinetic and potential energies!
    \end{itemize}
    \pagebreak
    \item Computing more observables.
    \begin{itemize}
        \item We can show that
        \begin{align*}
            \ev{\hat{\vec{x}}}{\psi_n} &= \ev{\hat{\vec{p}}\,}{\psi_n} = 0&
            \ev{\hat{\vec{x}}{\,}^2}{\psi_n} &= \frac{\hbar\omega}{k}\left( n+\frac{1}{2} \right)&
            \ev{\hat{\vec{p}}{\,}^2}{\psi_n} &= \hbar\omega m\left( n+\frac{1}{2} \right)
        \end{align*}
    \end{itemize}
    \item It follows from the above computations and the facts that
    \begin{align*}
        \Delta x^2 &= \ev{\hat{\vec{x}}{\,}^2}{\psi_n}-(\ev{\hat{\vec{x}}}{\psi_n})^2&
        \Delta p^2 &= \ev{\hat{\vec{p}}{\,}^2}{\psi_n}-(\ev{\hat{\vec{p}}\,}{\psi_n})^2
    \end{align*}
    that
    \begin{align*}
        \Delta x^2\cdot\Delta p^2 &= \hbar^2\left( n+\frac{1}{2} \right)^2\\
        \Delta x\cdot\Delta p &= \frac{\hbar}{2}(2n+1)
    \end{align*}
    \begin{itemize}
        \item Implication: The ground state $\psi_0(x)$ is represented by a Gaussian since in this case, $\Delta x\cdot\Delta p=\hbar/2$.
    \end{itemize}
    \item Review from last class.
    \begin{itemize}
        \item Mostly stuff I already wrote down.
        \item One new equation formalizing the even/odd solutions:
        \begin{equation*}
            f_n(x) = (-1)^nf_n(-x)
        \end{equation*}
        \item The first four Hermite polynomials:
        \begin{align*}
            H_0(\xi) &= 1&
            H_1(\xi) &= 2\xi&
            H_2(\xi) &= 4\xi^2-2&
            H_3 &= 8\xi^3-12\xi
        \end{align*}
        \item Summary of the characteristics of $E_n$: The energy is quantized and grows linearly with $n$ in quanta of $\hbar\omega$, and has a minimum value $\hbar\omega/2$.
        \item As with other time-independent potentials, the general solution to the Schr\"{o}dinger equation will be
        \begin{equation*}
            \psi(x,t) = \sum_nc_n\psi_n(x)\e[-iE_nt/\hbar]
        \end{equation*}
        where
        \begin{equation*}
            \ev{\hat{H}}{\psi} = \sum_n|c_n|^2E_n
        \end{equation*}
    \end{itemize}
\end{itemize}



\section{Time Dependence and Coherent States}
\begin{itemize}
    \item \marginnote{1/24:}Review of the harmonic oscillator.
    \begin{itemize}
        \item Our Hamiltonian is
        \begin{equation*}
            \hat{H} = -\frac{\hbar^2}{2m}\dv[2]{x}+\frac{kx^2}{2} = \frac{\hat{\vec{p}}{\,}^2}{2m}+\frac{k\hat{\vec{x}}{\,}^2}{2}
        \end{equation*}
        \begin{itemize}
            \item We have an analogy with the classical $\omega^2=k/m$.
        \end{itemize}
        \item Under this Hamiltonian, $\hat{H}\ket{\psi_n}=E_n\ket{\psi_n}$ implies that
        \begin{equation*}
            E_n = \hbar\omega\left( \frac{1}{2}+n \right)
        \end{equation*}
        \item The raising and lowering operators are given by
        \begin{align*}
            a_+ &= \frac{1}{\sqrt{2\hbar m\omega}}[-i\hat{\vec{p}}+m\omega\hat{\vec{x}}\,]&
            a_- &= \frac{1}{\sqrt{2\hbar m\omega}}[i\hat{\vec{p}}+m\omega\hat{\vec{x}}\,]
        \end{align*}
        \begin{itemize}
            \item Together, these imply that
            \begin{equation*}
                \hat{H} = \hbar\omega\left( a_+a_-+\frac{1}{2} \right)
            \end{equation*}
            \item We also have that
            \begin{align*}
                a_+a_-\ket{\psi_n} &= n\ket{\psi_n}&
                    a_+\ket{\psi_n} &= \sqrt{n+1}\ket{\psi_{n+1}}\\
                [a_-,a_+] &= 1&
                    a_-\ket{\psi_n} &= \sqrt{n}\ket{\psi_{n-1}}
            \end{align*}
            \item We call $a_+a_-$ the number operator.
            \item We should go home and learn these formulas.
        \end{itemize}
        \item The full eigenstate is
        \begin{equation*}
            \psi(x,t) = \sum_{n=0}^\infty\underbrace{c_n\psi_n(x)\e[-iE_nt/\hbar]}_{\psi_n(x,t)}
        \end{equation*}
        \item Two properties of this eigenstate.
        \begin{enumerate}
            \item We have that
            \begin{equation*}
                \ket{\psi} = \sum_{n=0}^\infty c_n\e[-iE_nt/\hbar]\ket{\psi_n}
            \end{equation*}
            which implies that
            \begin{equation*}
                \sum_{n=0}^\infty|c_n|^2 = 1
            \end{equation*}
            since $\braket{\psi}=1$ and $\braket{\psi_n}{\psi_m}=\delta_{nm}$.
            \item We have that
            \begin{equation*}
                \ev{\hat{H}}{\psi} = \sum_{n=0}^\infty|c_n|^2E_n
            \end{equation*}
        \end{enumerate}
        \item We have that
        \begin{align*}
            \ev**{\frac{k\hat{\vec{x}}{\,}^2}{2}}{\psi_n} &= \frac{\hbar\omega}{2}\left( n+\frac{1}{2} \right) = \frac{E_n}{2}&
            \ev**{\frac{\hat{\vec{p}}{\,}^2}{2m}}{\psi_n} &= \frac{\hbar\omega}{2}\left( n+\frac{1}{2} \right) = \frac{E_n}{2}
        \end{align*}
        \begin{itemize}
            \item Note that this makes sense because the sum $E_n/2+E_n/2$ of potential and kinetic should be $E_n$, and it will be!
        \end{itemize}
        \item Additionally, recall that we have
        \begin{align*}
            \hat{\vec{p}}{\,}^2 &\propto (a_+-a_-)^2&
            \hat{\vec{x}}{\,}^2 &\propto (a_++a_-)^2
        \end{align*}
        \begin{itemize}
            \item Thus, we have that
            \begin{align*}
                \ev{\hat{\vec{p}}\,}{\psi_n} &= \ev{(a_+-a_-)}{\psi_n} = 0&
                \ev{\hat{\vec{x}}}{\psi_n} &= \ev{(a_++a_-)}{\psi_n} = 0
            \end{align*}
        \end{itemize}
        \item The harmonic oscillator is a very important problem in physics, and we should know it by heart! (In order to pass the class.)
    \end{itemize}
    \item Recall as well that there is a correspondence between the Dirac notation and the functional notation, given by
    \begin{equation*}
        \psi_n(x) \mapsto \ket{\psi_n}
    \end{equation*}
    \begin{itemize}
        \item As an additional example,
        \begin{equation*}
            \frac{1}{\sqrt{2\hbar m\omega}}\left[ -\hbar\dv{x}+m\omega x \right]\psi_n(x) = \sqrt{n+1}\psi_{n+1}(x)
            \quad\mapsto\quad
            a_+\ket{\psi_n} = \sqrt{n+1}\ket{\psi_{n+1}}
        \end{equation*}
        \item One more example:
        \begin{equation*}
            \hbar\dv{\psi_0}{x}+m\omega x\psi_0(x) = 0
            \quad\mapsto\quad
            a_-\ket{\psi_0} = 0
        \end{equation*}
        \begin{itemize}
            \item Note that solving this ODE yields the solution
            \begin{equation*}
                \psi_0 = C\exp(-\frac{m\omega x^2}{2\hbar})
            \end{equation*}
            \item It appears that this is how we intuitively derive the ansatz we used last Friday!
        \end{itemize}
    \end{itemize}
    \item Now we start on some new content.
    \item Observe that
    \begin{align*}
        \frac{2m\omega\hat{\vec{x}}}{\sqrt{2\hbar m\omega}} &= a_++a_-\\
        \hat{\vec{x}} &= \sqrt{\frac{\hbar}{2m\omega}}(a_++a_-)
    \end{align*}
    \item In classical mechanics, the solution to the harmonic oscillator is
    \begin{equation*}
        x(t) = A\sin\omega t+B\cos\omega t
    \end{equation*}
    \item We now investigate the observables of $\ket{\psi}$.
    \item To start with, we show how $\ev{\hat{\vec{x}}}{\psi}$ varies with time. This will lead into a discussion of something called coherent states. Let's begin.
    \begin{itemize}
        \item We start with
        \begin{equation*}
            \ev{\hat{\vec{x}}}{\psi} = \sum_{m,n=0}^\infty c_m^*c_n\e[i(E_m-E_n)t/\hbar]\mel{\psi_m}{\hat{\vec{x}}}{\psi_n}
        \end{equation*}
        \item We can algebraically manipulate the above to
        \begin{align*}
            \ev{\hat{\vec{x}}}{\psi} ={}& \sum_{m,n=0}^\infty c_m^*c_n\e[i(\hbar\omega(m-n))t/\hbar]\sqrt{\frac{\hbar}{2m\omega}}\left( \sqrt{n+1}\delta_{m,n+1}+\sqrt{n}\delta_{m,n-1} \right)\\
            ={}& \sum_{n=0}^\infty c_{n+1}^*c_n\e[i\omega t]\sqrt{\frac{\hbar}{2m\omega}}\sqrt{n+1}+\sum_{n=1}^\infty c_{n-1}^*c_n\e[-i\omega t]\sqrt{\frac{\hbar}{2m\omega}}\sqrt{n}\\
            ={}& \sum_{n=0}^\infty c_{n+1}^*c_n\e[i\omega t]\sqrt{\frac{\hbar}{2m\omega}}\sqrt{n+1}+\sum_{n=0}^\infty c_n^*c_{n+1}\e[-i\omega t]\sqrt{\frac{\hbar}{2m\omega}}\sqrt{n+1}\\
            \begin{split}
                ={}& \sqrt{\frac{\hbar}{2m\omega}}\cos(\omega t)\left[ \sum_{n=0}^\infty\left( c_{n+1}^*c_n+c_n^*c_{n+1} \right)\sqrt{n+1} \right]\\
                & +\sqrt{\frac{\hbar}{2m\omega}}\sin(\omega t)\left[ \sum_{n=0}^\infty\left( c_{n+1}^*c_n-c_n^*c_{n+1} \right)\sqrt{n+1} \right]
            \end{split}
        \end{align*}
        \item Thus,
        \begin{equation*}
            \ev{\hat{\vec{x}}}{\psi} = A\cos\omega t+B\sin\omega t
        \end{equation*}
        where
        \begin{align*}
            A &= 2\re\left[ \sum_{n=0}^\infty c_{n+1}^*c_n\sqrt{n+1} \right]\sqrt{\frac{\hbar}{2m\omega}}&
                B &= 2\im\left[ \sum_{n=0}^\infty c_{n+1}^*c_n\sqrt{n+1} \right]\sqrt{\frac{\hbar}{2m\omega}}\\
            &= \re\left[ \sum_{n=0}^\infty c_{n+1}^*c_n\sqrt{n+1} \right]\sqrt{\frac{2\hbar}{m\omega}}&
                &= \im\left[ \sum_{n=0}^\infty c_{n+1}^*c_n\sqrt{n+1} \right]\sqrt{\frac{2\hbar}{m\omega}}
        \end{align*}
        \item Now for large values of $n$,
        \begin{equation*}
            \sqrt{n+1}\sqrt{\frac{2\hbar}{m\omega}} = \sqrt{\frac{2\hbar\omega(n+1)}{m\omega^2}}
            \approx \sqrt{\frac{2E_n}{m\omega^2}}
        \end{equation*}
        where
        \begin{align*}
            E_n &= \hbar\omega\left( n+\frac{1}{2} \right)&
            x &= A\sin\omega t&
            E &= \frac{m\omega^2A^2}{2}&
            A &= \sqrt{\frac{2E}{m\omega^2}}
        \end{align*}
        \begin{itemize}
            \item How can we just ignore the real and imaginary sum terms??
        \end{itemize}
        \item Now take the harmonic oscillator. Notice that $\sum_n$ is dominated by large values of $n\approx\bar{n}$, close to $\bar{n}$, where $\bar{n}\gg 1$. Thus,
        \begin{equation*}
            \ev{\hat{\vec{x}}}{\psi} = \sqrt{\frac{2E\bar{n}}{m\omega^2}}\sum_{n=0}^\infty\re\left[ \sum c_{n+1}^*c_n \right]\sin\omega t
        \end{equation*}
        and
        \begin{equation*}
            \ev{\hat{\vec{x}}{\,}^2}{\psi}-(\ev{\hat{\vec{x}}}{\psi})^2 \neq 0
        \end{equation*}
        \item This is \emph{not} classical motion.
        \item The states that come closest to realizing classical motion are called \textbf{coherent states}.
    \end{itemize}
    \item \textbf{Coherent state} (of the harmonic oscillator): A state in which the uncertainty in $\hat{\vec{x}}$ is minimized. \emph{Denoted by} $\bm{|\alpha\rangle}$.
    \item It turns out that the coherent states of the harmonic oscillator are the eigenstates of the lowering operator. Denoting the corresponding eigenvalue by $\alpha$, we have that
    \begin{equation*}
        a_-\ket{\alpha} = \alpha\ket{\alpha}
    \end{equation*}
    \item Aside: $\ket{\alpha}$ can surely be expressed as a linear combination of the $\psi_n$. What does the lowering operator do to $\psi_0$, in particular, should it have a nonzero coefficient?
    \begin{itemize}
        \item It acts as follows, simply zeroing it out.
        \begin{equation*}
            a_-\ket{\psi_0} = 0\ket{\psi_0}
        \end{equation*}
    \end{itemize}
    \item Now what is $\ket{\alpha}$?
    \item Well, for a state to be coherent, we must have
    \begin{align*}
        \frac{\hbar}{2} &= \sigma_x^2\\
        &= \ev{\hat{\vec{x}}{\,}^2}{\alpha}-(\ev{\hat{\vec{x}}}{\alpha})^2\\
        &= \frac{\hbar}{2m\omega}\left[ \ev{(a_++a_-)^2}{\alpha}-(\ev{(a_++a_-)}{\alpha})^2 \right]\\
        &= \frac{\hbar}{2m\omega}\left[ \ev{a_+^2+a_+a_-+a_-a_++a_-^2}{\alpha}-(\ev{(a_++a_-)}{\alpha})^2 \right]\\
        &= \dots
    \end{align*}
    \begin{itemize}
        \item We'll finish this up next time.
        \item Is it really $\hbar/2$ here??
    \end{itemize}
\end{itemize}



\section{Hermitian Operators; Position and Momentum Eigenstates}
\begin{itemize}
    \item \marginnote{1/26:}Recap of the harmonic oscillator.
    \begin{itemize}
        \item The Hamiltonian (in terms of $\hat{p},\hat{x}$; and in terms of $a_+,a_-$).
        \item The definitions of $a_+,a_-$.
        \item The effect of $a_+,a_-$ on $\ket{n}:=\ket{\psi_n}$.
        \item The effect of $\hat{H}$ on $\ket{n}$.
        \item Adjoints of the \textbf{ladder operators}:
        \begin{align*}
            (a_+)^\dagger &= a_-&
            (a_-)^\dagger &= a_+
        \end{align*}
        \item The commutator $[a_-,a_+]=1$.
        \item The formula for a generic state $\ket{\psi}$, i.e.,
        \begin{equation*}
            \ket{\psi} = \sum_{n=0}^\infty c_n\e[-iE_nt/\hbar]\ket{n}
        \end{equation*}
        \begin{itemize}
            \item This will of course appear as a question in the midterm and final!
            \item We must also remember that
            \begin{align*}
                1 &= \braket{\psi} = \sum_{n=0}^\infty|c_n|^2&
                1 &= \ev{\hat{H}}{\psi} = \sum_{n=0}^\infty|c_n|^2E_n
            \end{align*}
        \end{itemize}
        \item The probability of measuring the energy of $\ket{\psi}$ as $E_n$ is $|c_n|^2$.
        \begin{itemize}
            \item So when we perform a measurement, the energy of $\ket{\psi}$ collapses to that of one eigenstate.
        \end{itemize}
    \end{itemize}
    \item \textbf{Ladder operator}: An element in the class of operators that send $\ket{n}$ to scalar multiples of $\ket{n+i}$ for some $i\in\Z\setminus\{0\}$.
    \begin{itemize}
        \item The raising and lowering operators are ladder operators!
    \end{itemize}
    \item The midterm.
    \begin{itemize}
        \item 50\% of the midterm will be related to harmonic oscillator content, esp. the last few equations above following the definition of $\ket{\psi}$.
        \item The midterm will only cover what we covered through today.
        \item The midterm may be on February 5. It sounds like it will be on Friday, February 9, though.
        \item It will take place in this classroom.
        \item It will be open book.
        \begin{itemize}
            \item Can we bring virtual notes, or does everything have to be printed out??
        \end{itemize}
        \item The midterm questions will be the same level as the PSet questions; there may even be some repetition! Def take a look at the PSets.
        \item PSet 1 through PSet 4 will be covered on the midterm.
        \item Foundations of quantum mechanics plus one-dimensional problems.
        \item We will be allowed to turn in the midterm through 1:00 PM, though it shouldn't take us more than 50 minutes.
    \end{itemize}
    \item The first two problems of PSet 4 must be solved; the third one can be dropped \emph{or} can be solved for 5 bonus points.
    \item We now begin on new content.
    \item Recall the following expression from last class.
    \begin{equation*}
        \ev{\hat{\vec{x}}}{\psi} = \sqrt{\frac{2\hbar}{m\omega}}\sum_{n=0}^\infty\left[ \sqrt{n+1}\cos(\omega t)\re(c_{n+1}^*c_n)+\sqrt{n+1}\sin(\omega t)\im(c_{n+1}^*c_n) \right]
    \end{equation*}
    \begin{itemize}
        \item This is a really complicated expression, especially as we prepare to talk about coherent states.
        \item Thus, it was quite difficult to prove that
        \begin{equation*}
            \ev{\hat{\vec{x}}{\,}^2}{\psi} \neq (\ev{\hat{\vec{x}}}{\psi})^2
        \end{equation*}
        \item Can we introduce a notation that will allow us to work with this expression and similar ones more easily?
    \end{itemize}
    \item Wagner restates the definition of a coherent state and and the uncertainty principles.
    \item Recall that
    \begin{equation*}
        a_-\ket{\alpha} = \alpha\ket{\alpha}
    \end{equation*}
    and that
    \begin{equation*}
        \ket{\alpha} = \sum_nc_n\ket{n}
    \end{equation*}
    \item The Hermitian conjugate of $a_-$ is $a_+$ and hence, the Hermitian conjugate of $a_-\ket{\alpha}$ is
    \begin{equation*}
        \bra{\alpha}a_+ = \bra{\alpha}\alpha^*
    \end{equation*}
    \item Thus, since $\braket{\alpha}=1$
    \begin{equation*}
        \ev{a_+a_-}{\alpha} = \alpha\ev{a_+}{\alpha}
        = \alpha\ev{\alpha^*}{\alpha}
        = \alpha^*\alpha\braket{\alpha}
        = \alpha^*\alpha
    \end{equation*}
    \item We now seek to verify that an eigenstate of $a_-$ does, in fact, minimize the uncertainty in $\hat{x}$.
    \begin{itemize}
        \item For simplicity, we will consider $\ket{\alpha}$ at $t=0$ (this will remove the complex exponential from calculations).
        \item First off, we have that
        \begin{equation*}
            \ev{\hat{\vec{x}}}{\alpha} = \sqrt{\frac{\hbar}{2m\omega}}\ev{(a_++a_-)}{\alpha}
            = (\alpha^*+\alpha)\sqrt{\frac{\hbar}{2m\omega}}
        \end{equation*}
        and
        \begin{align*}
            \ev{\hat{\vec{x}}{\,}^2}{\alpha} &= \frac{\hbar}{2m\omega}\ev{(a_++a_-)(a_++a_-)}{\alpha}\\
            &= \frac{\hbar}{2m\omega}[\ev{a_+^2}{\alpha}+\ev{a_+a_-}{\alpha}+\ev{a_-a_+}{\alpha}+\ev{a_-^2}{\alpha}]\\
            &= \frac{\hbar}{2m\omega}[(\alpha^*)^2\underbrace{\braket{\alpha}}_1+\alpha^*\alpha+\ev{(\underbrace{a_-a_+-a_+a_-}_1+a_+a_-)}{\alpha}+\alpha^2]\\
            &= \frac{\hbar}{2m\omega}[(\alpha^*)^2+\alpha^2+2|\alpha|^2+1]
        \end{align*}
        \item Combining these, we have that
        \begin{equation*}
            \ev{\hat{\vec{x}}{\,}^2}{\alpha}-(\ev{\hat{\vec{x}}}{\alpha})^2 = \frac{\hbar}{2m\omega}[(\alpha^*)^2+\alpha^2+2|\alpha|^2+1-(\alpha^*)^2-\alpha^2-2|\alpha|^2]
            = \frac{\hbar}{2m\omega}
        \end{equation*}
        \item Second, we have that
        \begin{equation*}
            \ev{\hat{\vec{p}}}{\alpha} = \sqrt{\frac{\hbar m\omega}{2}}\ev{(a_+-a_-)}{\alpha}
            = \sqrt{\frac{\hbar m\omega}{2}}(\alpha^*-\alpha)
        \end{equation*}
        and
        \begin{align*}
            \ev{\hat{\vec{p}}{\,}^2}{\alpha} &= -\frac{\hbar m\omega}{2}\ev{(a_+-a_-)(a_+-a_-)}{\alpha}\\
            &= -\frac{\hbar m\omega}{2}\big[ (\alpha^*)^2+\alpha^2-|\alpha|^2-\ev{\underbrace{a_-a_+}_{a_+a_-+1}}{\alpha} \big]\\
            &= -\frac{\hbar m\omega}{2}\left[ (\alpha^*)^2+\alpha^2-2|\alpha|^2-1 \right]
        \end{align*}
        \item Combining these, we have that
        \begin{equation*}
            \ev{\hat{\vec{p}}{\,}^2}{\alpha}-(\ev{\hat{\vec{p}}\,}{\alpha})^2 = \frac{\hbar m\omega}{2}\left[ -(\alpha^*)^2-\alpha^2+2|\alpha|^2+1+(\alpha^*)^2+\alpha^2-2|\alpha|^2 \right]
            = \frac{\hbar m\omega}{2}
        \end{equation*}
        \item Therefore,
        \begin{align*}
            \sigma_p^2\sigma_x^2 &= \frac{\hbar m\omega}{2}\cdot\frac{\hbar}{2m\omega}\\
            &= \frac{\hbar^2}{4}\\
            \sigma_p\sigma_x &= \frac{\hbar}{2}
        \end{align*}
        as desired.
    \end{itemize}
    \item If we reassert full time dependence, we obtain
    \begin{equation*}
        \ket{\alpha}(t) = \sum_nc_n\e[-iE_nt/\hbar]\ket{n}
    \end{equation*}
    \begin{itemize}
        \item Then
        \begin{align*}
            a_-\ket{\alpha} &= \sum_{n=0}^\infty c_n\e[-iE_nt/\hbar]\sqrt{n}\ket{n-1}\\
            &= \sum_{n=0}^\infty c_{n+1}\e[-iE_{n+1}t/\hbar]\sqrt{n+1}\ket{n}
        \end{align*}
        \item And recall that
        \begin{equation*}
            a_-\ket{\alpha} = \alpha\ket{\alpha}
        \end{equation*}
        \item Thus, via term-by-term transitivity for each $\ket{n}$,
        \begin{align*}
            \alpha c_n &= c_{n+1}\e[-i(E_{n+1}-E_n)t/\hbar]\sqrt{n+1}\\
            \alpha c_n &= c_{n+1}\e[-i\omega t]\sqrt{n+1}
        \end{align*}
        \item We can continue on with this recurrence relation to find a formula for all coefficients $c_n$, from which we can define $\ket{\alpha}$ explicitly as a linear combination of the $\ket{n}$.
    \end{itemize}
    \item If $\alpha$ is real and $\psi_\alpha(x)$ denotes the time-independent factor in $\ket{\alpha}$, then
    \begin{align*}
        a_-\psi_\alpha(x) &= \alpha\psi_\alpha(x)\\
        \left[ \hbar\dv{x}+m\omega x \right]\psi_\alpha(x) &= \alpha\psi_\alpha(x)
    \end{align*}
    \begin{itemize}
        \item Then
        \begin{equation*}
            \frac{1}{\psi_\alpha}\dv{x}\psi_\alpha+\left( \frac{m\omega x}{\hbar}-\alpha \right) = 0
        \end{equation*}
        \item Thus, solving the differential equation, we obtain
        \begin{equation*}
            \psi_\alpha = \exp\left[ -\frac{m\omega}{2\hbar}(x-\Exp{x})^2 \right]
        \end{equation*}
        which is a Gaussian.
        \item Therefore,
        \begin{equation*}
            a_-\ket{0} = 0\ket{0}
        \end{equation*}
    \end{itemize}
    \item We will program the time evolution of a coherent state in Python or Mathematica??
    \begin{itemize}
        \item A real wave function is a crazy thing that does flip from side to side at $T/2$ and $T$.
        \begin{itemize}
            \item Essentially,
            \begin{equation*}
                |\psi(x,t)|^2 = |\psi(-x,t+T/2)|^2
            \end{equation*}
        \end{itemize}
        \item A coherent state is just a Gaussian that oscillates back and forth to both sides of the $y$-axis.
    \end{itemize}
\end{itemize}



\section{G Chapter 2: Time-Independent Schr\"{o}dinger Equation}
\emph{From \textcite{bib:Griffiths}.}
\subsection*{Section 2.3: The Harmonic Oscillator}
\begin{itemize}
    \item \marginnote{1/29:}Sets up the relevant TISE, as in class.
    \item Note that "it is customary to eliminate the spring constant in favor of the classical frequency" \parencite[58]{bib:Griffiths}.
    \item Goes through the ladder operator method in great detail and very coherently; I should probably return!!
    \begin{itemize}
        \item There is a proof in here of why $a_+^\dagger=a_-$.
    \end{itemize}
    \item Goes through the \textbf{power series method} from the Lecture 7 notes.
    \begin{itemize}
        \item This is the brute force method, though it is useful (as with the hydrogen atom later on).
    \end{itemize}
    \item \textbf{Canonical commutation relation}: The relation defined as follows. \emph{Given by}
    \begin{equation*}
        [\hat{x},\hat{\vec{p}}] = i\hbar
    \end{equation*}
\end{itemize}


\subsection*{Section 2.4: The Free Particle}
\begin{itemize}
    \item Relevant to 1/5 and 1/17 discussions; I should probably return!!
\end{itemize}


\subsection*{Section 2.5: The Delta-Function Potential}
\begin{itemize}
    \item Relevant to PSet 2; I should probably return!!
\end{itemize}


\subsection*{Section 2.6: The Finite Square Well}
\begin{itemize}
    \item Relevant to PSet 2; I should probably return!!
\end{itemize}



\section{G Chapter 3: Formalism}
\emph{From \textcite{bib:Griffiths}.}
\subsection*{Section 3.1: Hilbert Space}
\begin{itemize}
    \item Purpose: Recast some of the miracles we've encountered thus far in more powerful terms.
    \item Lots of stuff I should read just for fun (tons of answers to questions I've wondered at over the years), and some stuff actually related to in-class discussions of Hermitian operators, compatible operators, proving the uncertainty principle, Gaussian wave packets, the Ehrenfest theorem, Dirac notation, etc.
\end{itemize}



\section{G Appendix: Linear Algebra}
\emph{From \textcite{bib:Griffiths}.}
\begin{itemize}
    \item A terrific review of relevant concepts, all expressed in Dirac notation.
\end{itemize}



\section{T Chapter 7: The One-Dimensional Harmonic Oscillator}
\emph{From \textcite{bib:Townsend}.}
\subsection*{Section 7.7: Time Dependence}
\begin{itemize}
    \item There is some stuff here on $|\psi(x,t)|^2=|\psi(-x,t+T/2)|^2$.
\end{itemize}


\subsection*{Section 7.8: Coherent States}
\begin{itemize}
    \item \textbf{Coherent state}: A superposition of energy eigenstates of the harmonic oscillator that is also an eigenstate of the lowering operator. \emph{Denoted by} $\bm{|\alpha\rangle}$.
    \item $\bm{\alpha}$: The eigenvalue corresponding to the coherent state $\ket{\alpha}$.
    \begin{itemize}
        \item Since $a_-$ is not Hermitian, $\alpha$ need not be real.
    \end{itemize}
    \item Coherent states "come closest to representing classical electromagnetic waves with a well-defined phase" \parencite[263]{bib:Townsend}.
    \begin{itemize}
        \item For a harmonic oscillator, they come "closest to the classical limit of a particle oscillating back and forth in a harmonic oscillator potential" \parencite[263]{bib:Townsend}.
    \end{itemize}
    \item Coherent states were first derived by Schr\"{o}dinger when he was looking for solutions to the Schr\"{o}dinger equation that satisfy the \textbf{correspondence principle}.
    \item \textbf{Correspondence principle}: The behavior of systems described by the theory of quantum mechanics reproduces classical physics in the limit of large quantum numbers.
    \item \textcite{bib:Townsend} completes Wagner's derivation of $\ket{\alpha}$ as a linear combination of the $\ket{n}$.
    \item Time Evolution of a Coherent State.
    \item Repeat of the derivation of the minimum uncertainty from class.
    \item Shows that the ground coherent state is an oscillating Gaussian.
\end{itemize}




\end{document}