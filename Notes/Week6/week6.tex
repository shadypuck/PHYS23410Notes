\documentclass[../notes.tex]{subfiles}

\pagestyle{main}
\renewcommand{\chaptermark}[1]{\markboth{\chaptername\ \thechapter\ (#1)}{}}
\setcounter{chapter}{5}

\begin{document}




\chapter{The Hydrogen Atom}
\section{Central Potentials}
\begin{itemize}
    \item \marginnote{2/5:}Review.
    \begin{itemize}
        \item Definition of \textbf{central potential}.
        \begin{itemize}
            \item In this case, we have three good observables: $\hat{H},\hat{\vec{L}}{\,}^2,\hat{L}_z$.
        \end{itemize}
        \item Last Friday, we discovered that the eigenstates are characterized by three numbers $n,\ell,m$ that correspond to the three operators above.
        \begin{itemize}
            \item Altogether, we have that
            \begin{align*}
                \hat{L}_z\ket{n\ell m} &= \hbar m\ket{n\ell m}&
                \hat{\vec{L}}{\,}^2\ket{n\ell m} &= \hbar^2\ell(\ell+1)\ket{n\ell m}&
                \hat{H}\ket{n\ell m} &= E_n\ket{n\ell m}
            \end{align*}
        \end{itemize}
        \item We also defined ladder operators $L_+,L_-$ such that
        \begin{equation*}
            \hat{L}_\pm\ket{n\ell m} = \sqrt{\ell(\ell+1)-m(m\pm 1)}\ket{n\ell(m\pm 1)}
        \end{equation*}
    \end{itemize}
    \item \textbf{Central potential}: A three-dimensional potential energy distribution in which the potential depends only on the distance from the origin. \emph{Denoted by} $\bm{V(r)}$.
    \item The eigenstates are well normalized, i.e.,
    \begin{equation*}
        \braket{n\ell m}{n\ell m'} = \delta_{mm'}
    \end{equation*}
    \begin{itemize}
        \item It follows that
        \begin{equation*}
            \ev{\hat{L}_x}{n\ell m} = \ev{\frac{1}{2}(\hat{L}_++\hat{L}_-)}{n\ell m} = 0
        \end{equation*}
        \item Similarly,
        \begin{equation*}
            \ev{\hat{L}_y}{n\ell m} = 0
        \end{equation*}
        \item Additionally, we have that
        \begin{equation*}
            \ev{(\hat{L}_x^2+\hat{L}_y^2)}{n\ell m} = \ev{(\hat{\vec{L}}{\,}^2-\hat{L}_z^2)}{n\ell m} = \hbar^2[\ell(\ell+1)-m^2]
        \end{equation*}
        \begin{itemize}
            \item Since the above eigenvalue must be greater than or equal to zero, $|m|\leq\ell$.
        \end{itemize}
        \item Recall that $\hat{L}_x,\hat{L}_y$ are incompatible with $\hat{L}_z$.
        \begin{itemize}
            \item This is why we have an uncertainty associated with the quantity $\hbar^2[\ell(\ell+1)-m^2]$.
            \item This is also why we have
            \begin{equation*}
                \ev{(\hat{L}_x^2+\hat{L}_y^2)}{n\ell m} = 2\ev{\hat{L}_x^2}{n\ell m}
                = 2\ev{\hat{L}_y^2}{n\ell m}
            \end{equation*}
        \end{itemize}
    \end{itemize}
    \item Recall expressing the wave function in polar coordinates via $\psi(r,\theta,\phi)$.
    \begin{itemize}
        \item Solving by separation of variables, we have
        \begin{equation*}
            \ket{n\ell m} = \psi_{n\ell m}(r,\theta,\phi)
            = R_{n\ell}(r)\cdot Y_{\ell m}(\theta,\phi)
        \end{equation*}
        \item This has the interesting property that if we define
        \begin{equation*}
            U_{n\ell}(r) = rR_{n\ell}(r)
        \end{equation*}
        then
        \begin{equation*}
            -\frac{\hbar^2}{2M}\dv[2]{r}[U_{n\ell}(r)]+\underbrace{\left[ \frac{\hbar^2\ell(\ell+1)}{2Mr^2}+V(r) \right]}_{V_\text{eff}(r)}U_{n\ell}(r) = E_{n\ell}U_{n\ell}(r)
        \end{equation*}
        \item This means that $U$ is the solution to a one-dimensional problem in an effective potential.
    \end{itemize}
    \item A couple of interesting comments.
    \begin{itemize}
        \item $m$ doesn't appear because directionality doesn't matter. We don't care which direction we project into; we only care about the total angular momentum.
        \begin{itemize}
            \item Recall that there is a $2\ell+1$ degeneracy associated with the fact that $m$ doesn't appear.
            \item Indeed, we get energy levels within this potential.
        \end{itemize}
        \item Recall that $M$ denotes the mass to avoid confusion with the quantum number $m$.
        \item The effective potential we are considering is of the same shape as the red line in Figure \ref{fig:Veff}.
    \end{itemize}
    \item Recall that solving for $Y$, we obtain
    \begin{equation*}
        \underbrace{-\hbar^2\left[ \frac{1}{\sin\theta}\pdv{\theta}(\sin\theta\pdv{Y_{\ell m}}{\theta})+\frac{1}{\sin^2\theta}\pdv[2]{Y_{\ell m}}{\phi} \right]}_{\hat{\vec{L}}{\,}^2Y_{\ell m}} = \hbar^2\ell(\ell+1)Y_{\ell m}
    \end{equation*}
    \begin{itemize}
        \item The rather complicated expression on the left above just describes $\hat{\vec{L}}{\,}^2Y_{\ell m}$ in polar coordinates.
        \item We'll get as a solution
        \begin{equation*}
            Y_{\ell m}(\theta,\phi) = \e[im\phi]\Theta_{\ell m}(\theta)
        \end{equation*}
        \item We can therefore see that if $\hat{L}_z=-i\hbar(\pdv*{\phi})$ then
        \begin{equation*}
            \hat{L}_zY_{\ell m}(\theta,\phi) = \hbar mY_{\ell m}(\theta,\phi)
        \end{equation*}
        \item Remember that $m$ and $\ell$ are both integers.
        \item Simplifying the above, we get
        \begin{equation*}
            \sin\theta\dv{\theta}(\sin\theta\dv{\Theta_{\ell m}}{\theta})-m^2\Theta_{\ell m}+[\ell(\ell+1)\sin^2\theta]\Theta_{\ell m} = 0
        \end{equation*}
        \item Secretly, all the dependence on $\theta$ is a dependence on $\cos\theta$ since we can make substitutions like $\sin^2\theta=1-\cos^2\theta$.
        \item The solutions are then
        \begin{equation*}
            \Theta_{\ell m}(u) = AP_\ell^m(u)
        \end{equation*}
        where $u=\cos\theta$ and $P_\ell^m$ are the \textbf{associated Legendre functions}.
        \item Finally, if we want to obtain a well-normalized solution, i.e., we need to calculate $A$. Computationally, this means that we need
        \begin{equation*}
            \int_0^{2\pi}\int_0^\pi\int_0^\infty\dd{r}\dd\theta\dd\phi\ r^2\sin\theta|Y_{\ell m}(\theta,\phi)R_{n\ell}(r)|^2
        \end{equation*}
        \item This integral splits into two.
        \begin{align*}
            \int_0^{2\pi}\int_0^\pi\dd\theta\dd\phi\ \sin\theta|Y_{\ell m}(\theta,\phi)|^2 &= 1&
            \int_0^\infty\dd{r}\ \underbrace{|rR_{n\ell}(r)|^2}_{|U_{n\ell}(r)|^2} &= 1
        \end{align*}
        \item Note that this implies that
        \begin{align*}
            \int\dd\phi\dd\theta\ \sin\theta Y_{\ell m}(\theta,\phi)Y_{\ell'm'}(\theta,\phi) &= \delta_{\ell\ell'}\delta_{mm'}&
            \int\dd{r}\ r^2R_{n\ell}(r)R_{n'\ell'}(r) = \delta_{nn'}\delta_{\ell\ell'}
        \end{align*}
    \end{itemize}
    \item \textbf{Rodrigues formula}: The formula given as follows. \emph{Given by}
    \begin{equation*}
        \frac{1}{2^\ell\ell!}\dv[\ell]{u}(u^2-1)^\ell
    \end{equation*}
    \item \textbf{Legendre polynomials}: The system of complete orthogonal polynomials defined via the Rodrigues formula. \emph{Denoted by} $\bm{P_\ell(u)}$. \emph{Given by}
    \begin{equation*}
        P_\ell(u) = \frac{1}{2^\ell\ell!}\dv[\ell]{u}(u^2-1)^\ell
    \end{equation*}
    \item \textbf{Associated Legendre functions}: The canonical solutions of the general Legendre equation. \emph{Denoted by} $\bm{P_\ell^m(u)}$. \emph{Given by}
    \begin{equation*}
        P_\ell^m(u) = (1-u^2)^{|m|/2}\dv[|m|]{u}[P_\ell(u)]
    \end{equation*}
    \item A couple of closing comments.
    \begin{itemize}
        \item The normalization constant is such that \emph{en toto},
        \begin{equation*}
            Y_{\ell m}(\theta,\phi) = (-1)^m\sqrt{\frac{(2\ell+1)}{4\pi}\cdot\frac{(\ell-m)!}{(\ell+m)!}}P_{\ell m}(\cos\theta)\e[im\phi]
        \end{equation*}
        \begin{itemize}
            \item This is for $m\geq 0$
        \end{itemize}
        \item If $m<0$, then use
        \begin{equation*}
            Y_{\ell(-|m|)} = (-1)^{|m|}Y_{\ell|m|}^*(\theta,\phi)
        \end{equation*}
        where the complex conjugate of $Y$ just switches the exponential term at the end to $\e[-im\phi]$.
        \item The probability $P_{00}(\cos\theta)$ is a constant. So if we draw a circle in the $zx$-plane, it will not vary in intensity??
        \item We also have $P_{10}(\cos\theta)=\cos\theta$. Thus, this particle will move more quickly past the $x$-axis and slower toward the bottom of its circular orbit, yielding a $p$-orbital shape. Maximum probability is moving in the perpendicular direction.
        \item $P_{11}(\cos\theta)=\sin\theta$.
        \begin{itemize}
            \item If you have a particle with angular momentum 1 and modulus 1, it moves in the $xy$ plane in such a way that the total angular momentum points in the vertical direction and thus then it has maximum probability of being in the perpendicular plane.
            \item This gives us something sideways (think $p_z$ vs. $p_x$ orbitals).
        \end{itemize}
    \end{itemize}
\end{itemize}




\end{document}