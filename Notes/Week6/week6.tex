\documentclass[../notes.tex]{subfiles}

\pagestyle{main}
\renewcommand{\chaptermark}[1]{\markboth{\chaptername\ \thechapter\ (#1)}{}}
\setcounter{chapter}{5}

\begin{document}




\chapter{The Hydrogen Atom}
\section{Central Potentials}
\begin{itemize}
    \item \marginnote{2/5:}Review.
    \begin{itemize}
        \item Definition of \textbf{central potential}.
        \begin{itemize}
            \item In this case, we have three good observables: $\hat{H},\hat{\vec{L}}{\,}^2,\hat{L}_z$.
        \end{itemize}
        \item Last Friday, we discovered that the eigenstates are characterized by three numbers $n,\ell,m$ that correspond to the three operators above.
        \begin{itemize}
            \item Altogether, we have that
            \begin{align*}
                \hat{L}_z\ket{n\ell m} &= \hbar m\ket{n\ell m}&
                \hat{\vec{L}}{\,}^2\ket{n\ell m} &= \hbar^2\ell(\ell+1)\ket{n\ell m}&
                \hat{H}\ket{n\ell m} &= E_n\ket{n\ell m}
            \end{align*}
        \end{itemize}
        \item We also defined ladder operators $L_+,L_-$ such that
        \begin{equation*}
            \hat{L}_\pm\ket{n\ell m} = \sqrt{\ell(\ell+1)-m(m\pm 1)}\ket{n\ell(m\pm 1)}
        \end{equation*}
    \end{itemize}
    \item \textbf{Central potential}: A three-dimensional potential energy distribution in which the potential depends only on the distance from the origin. \emph{Denoted by} $\bm{V(r)}$.
    \item The eigenstates are well normalized, i.e.,
    \begin{equation*}
        \braket{n\ell m}{n\ell m'} = \delta_{mm'}
    \end{equation*}
    \begin{itemize}
        \item It follows that
        \begin{equation*}
            \ev{\hat{L}_x}{n\ell m} = \ev{\frac{1}{2}(\hat{L}_++\hat{L}_-)}{n\ell m} = 0
        \end{equation*}
        \item Similarly,
        \begin{equation*}
            \ev{\hat{L}_y}{n\ell m} = 0
        \end{equation*}
        \item Additionally, we have that
        \begin{equation*}
            \ev{(\hat{L}_x^2+\hat{L}_y^2)}{n\ell m} = \ev{(\hat{\vec{L}}{\,}^2-\hat{L}_z^2)}{n\ell m} = \hbar^2[\ell(\ell+1)-m^2]
        \end{equation*}
        \begin{itemize}
            \item Since the above eigenvalue must be greater than or equal to zero, $|m|\leq\ell$.
        \end{itemize}
        \item Recall that $\hat{L}_x,\hat{L}_y$ are incompatible with $\hat{L}_z$.
        \begin{itemize}
            \item This is why we have an uncertainty associated with the quantity $\hbar^2[\ell(\ell+1)-m^2]$.
            \item This is also why we have
            \begin{equation*}
                \ev{(\hat{L}_x^2+\hat{L}_y^2)}{n\ell m} = 2\ev{\hat{L}_x^2}{n\ell m}
                = 2\ev{\hat{L}_y^2}{n\ell m}
            \end{equation*}
        \end{itemize}
    \end{itemize}
    \item Recall expressing the wave function in polar coordinates via $\psi(r,\theta,\phi)$.
    \begin{itemize}
        \item Solving by separation of variables, we have
        \begin{equation*}
            \ket{n\ell m} = \psi_{n\ell m}(r,\theta,\phi)
            = R_{n\ell}(r)\cdot Y_{\ell m}(\theta,\phi)
        \end{equation*}
        \item This has the interesting property that if we define
        \begin{equation*}
            U_{n\ell}(r) = rR_{n\ell}(r)
        \end{equation*}
        then
        \begin{equation*}
            -\frac{\hbar^2}{2M}\dv[2]{r}[U_{n\ell}(r)]+\underbrace{\left[ \frac{\hbar^2\ell(\ell+1)}{2Mr^2}+V(r) \right]}_{V_\text{eff}(r)}U_{n\ell}(r) = E_{n\ell}U_{n\ell}(r)
        \end{equation*}
        \item This means that $U$ is the solution to a one-dimensional problem in an effective potential.
    \end{itemize}
    \item A couple of interesting comments.
    \begin{itemize}
        \item $m$ doesn't appear because directionality doesn't matter. We don't care which direction we project into; we only care about the total angular momentum.
        \begin{itemize}
            \item Recall that there is a $2\ell+1$ degeneracy associated with the fact that $m$ doesn't appear.
            \item Indeed, we get energy levels within this potential.
        \end{itemize}
        \item Recall that $M$ denotes the mass to avoid confusion with the quantum number $m$.
        \item The effective potential we are considering is of the same shape as the red line in Figure \ref{fig:Veff}.
    \end{itemize}
    \item Recall that solving for $Y$, we obtain
    \begin{equation*}
        \underbrace{-\hbar^2\left[ \frac{1}{\sin\theta}\pdv{\theta}(\sin\theta\pdv{Y_{\ell m}}{\theta})+\frac{1}{\sin^2\theta}\pdv[2]{Y_{\ell m}}{\phi} \right]}_{\hat{\vec{L}}{\,}^2Y_{\ell m}} = \hbar^2\ell(\ell+1)Y_{\ell m}
    \end{equation*}
    \begin{itemize}
        \item The rather complicated expression on the left above just describes $\hat{\vec{L}}{\,}^2Y_{\ell m}$ in polar coordinates.
        \item We'll get as a solution
        \begin{equation*}
            Y_{\ell m}(\theta,\phi) = \e[im\phi]\Theta_{\ell m}(\theta)
        \end{equation*}
        \item We can therefore see that if $\hat{L}_z=-i\hbar(\pdv*{\phi})$ then
        \begin{equation*}
            \hat{L}_zY_{\ell m}(\theta,\phi) = \hbar mY_{\ell m}(\theta,\phi)
        \end{equation*}
        \item Remember that $m$ and $\ell$ are both integers.
        \item Simplifying the above, we get
        \begin{equation*}
            \sin\theta\dv{\theta}(\sin\theta\dv{\Theta_{\ell m}}{\theta})-m^2\Theta_{\ell m}+[\ell(\ell+1)\sin^2\theta]\Theta_{\ell m} = 0
        \end{equation*}
        \item Secretly, all the dependence on $\theta$ is a dependence on $\cos\theta$ since we can make substitutions like $\sin^2\theta=1-\cos^2\theta$.
        \item The solutions are then
        \begin{equation*}
            \Theta_{\ell m}(u) = AP_\ell^m(u)
        \end{equation*}
        where $u=\cos\theta$ and $P_\ell^m$ are the \textbf{associated Legendre functions}.
        \item Finally, if we want to obtain a well-normalized solution, i.e., we need to calculate $A$. Computationally, this means that we need
        \begin{equation*}
            \int_0^{2\pi}\int_0^\pi\int_0^\infty\dd{r}\dd\theta\dd\phi\ r^2\sin\theta|Y_{\ell m}(\theta,\phi)R_{n\ell}(r)|^2
        \end{equation*}
        \item This integral splits into two.
        \begin{align*}
            \int_0^{2\pi}\int_0^\pi\dd\theta\dd\phi\ \sin\theta|Y_{\ell m}(\theta,\phi)|^2 &= 1&
            \int_0^\infty\dd{r}\ \underbrace{|rR_{n\ell}(r)|^2}_{|U_{n\ell}(r)|^2} &= 1
        \end{align*}
        \item Note that this implies that
        \begin{align*}
            \int\dd\phi\dd\theta\ \sin\theta Y_{\ell m}(\theta,\phi)Y_{\ell'm'}(\theta,\phi) &= \delta_{\ell\ell'}\delta_{mm'}&
            \int\dd{r}\ r^2R_{n\ell}(r)R_{n'\ell'}(r) = \delta_{nn'}\delta_{\ell\ell'}
        \end{align*}
    \end{itemize}
    \item \textbf{Rodrigues formula}: The formula given as follows. \emph{Given by}
    \begin{equation*}
        \frac{1}{2^\ell\ell!}\dv[\ell]{u}(u^2-1)^\ell
    \end{equation*}
    \item \textbf{Legendre polynomials}: The system of complete orthogonal polynomials defined via the Rodrigues formula. \emph{Denoted by} $\bm{P_\ell(u)}$. \emph{Given by}
    \begin{equation*}
        P_\ell(u) = \frac{1}{2^\ell\ell!}\dv[\ell]{u}(u^2-1)^\ell
    \end{equation*}
    \item \textbf{Associated Legendre functions}: The canonical solutions of the general Legendre equation. \emph{Denoted by} $\bm{P_\ell^m(u)}$. \emph{Given by}
    \begin{equation*}
        P_\ell^m(u) = (1-u^2)^{|m|/2}\dv[|m|]{u}[P_\ell(u)]
    \end{equation*}
    \item A couple of closing comments.
    \begin{itemize}
        \item The normalization constant is such that \emph{en toto},
        \begin{equation*}
            Y_{\ell m}(\theta,\phi) = (-1)^m\sqrt{\frac{(2\ell+1)}{4\pi}\cdot\frac{(\ell-m)!}{(\ell+m)!}}P_{\ell m}(\cos\theta)\e[im\phi]
        \end{equation*}
        \begin{itemize}
            \item This is for $m\geq 0$
        \end{itemize}
        \item If $m<0$, then use
        \begin{equation*}
            Y_{\ell(-|m|)} = (-1)^{|m|}Y_{\ell|m|}^*(\theta,\phi)
        \end{equation*}
        where the complex conjugate of $Y$ just switches the exponential term at the end to $\e[-im\phi]$.
        \item The probability $P_{00}(\cos\theta)$ is a constant. So if we draw a circle in the $zx$-plane, it will not vary in intensity??
        \item We also have $P_{10}(\cos\theta)=\cos\theta$. Thus, this particle will move more quickly past the $x$-axis and slower toward the bottom of its circular orbit, yielding a $p$-orbital shape. Maximum probability is moving in the perpendicular direction.
        \item $P_{11}(\cos\theta)=\sin\theta$.
        \begin{itemize}
            \item If you have a particle with angular momentum 1 and modulus 1, it moves in the $xy$ plane in such a way that the total angular momentum points in the vertical direction and thus then it has maximum probability of being in the perpendicular plane.
            \item This gives us something sideways (think $p_z$ vs. $p_x$ orbitals).
        \end{itemize}
    \end{itemize}
\end{itemize}



\section{Midterm Exam Review}
\begin{itemize}
    \item \marginnote{2/7:}Format of the midterm.
    \begin{itemize}
        \item 5 conceptual questions (multiple choice) that we should know by now.
        \item Two computational problems.
        \begin{itemize}
            \item One that appears in the problem set.
            \item One that appears in the problem set but we will have to do a couple extra things.
            \item Subject: One on harmonic oscillators and one on motion in potential wells.
        \end{itemize}
        \item If we fail the multiple choice, "something is wrong with you."
        \item The exam is not curved, but the class will have a curve.
        \item We can bring virtual notes.
    \end{itemize}
    \item Conceptual things to remember for the midterm.
    \begin{itemize}
        \item In classical mechanics, a particle is given by a path/trajectory $\vec{r}(t)$.
        \item In quantum mechanics, there is no path. The best we can do is define $\ev{\vec{r}}{\psi}(t)$, but we will always be hampered by the fact that $\sigma_{\vec{p}}\neq 0$.
        \begin{itemize}
            \item The uncertainty in momentum comes from the Heisenberg uncertainty relation.
            \item If the operator is independent of time (such as $\hat{x},\hat{p}_x,\hat{\vec{r}},\hat{\vec{p}},V(\vec{r})$), then
            \begin{equation*}
                \dv{t}(\ev{\hat{O}}{\psi}) = \frac{i}{\hbar}\ev{[\hat{H},\hat{O}]}{\psi}
            \end{equation*}
            \begin{itemize}
                \item This means that if $[\hat{H},\hat{O}]=0$, then the expected value of the operator is independent of time.
            \end{itemize}
            \item We most often deal with time-independent potentials $V(\vec{r},t)=V(\vec{r})$.
            \item Recall that since $[\hat{H},\hat{H}]=0$, $E=\ev{\hat{H}}{\psi}$ is a good quantum number.
            \begin{itemize}
                \item It follows that
                \begin{align*}
                    \hat{H}\ket{\psi_n} &= E_n\ket{\psi_n}&
                    \hat{H}^2\ket{\psi_n} &= E_n^2\ket{\psi_n}
                \end{align*}
                \item We also have that
                \begin{align*}
                    \sigma_{\hat{H}} &= 0&
                    \ev{\hat{H}^2}{\psi_n}-(\ev{\hat{H}}{\psi_n})^2 &= 0
                \end{align*}
            \end{itemize}
            \item It is very important to remember that
            \begin{align*}
                \ket{\psi} &= \sum_nc_n\e[-iE_nt/\hbar]\ket{\psi_n}\\
                \braket{\psi} &= \sum_n|c_n|^2 = 1\\
                \ev{\hat{H}}{\psi} &= \sum_n|c_n|^2E_n\\
                \braket{\psi_n}{\psi_m} &= \int\dd{\vec{r}}\ \psi_n^*\psi_m = \delta_{nm}
            \end{align*}
            \begin{itemize}
                \item It follows from the bottom three statements that $|c_n|^2$ is the probability of measuring $E_n$.
            \end{itemize}
            \item We can obtain the $m^\text{th}$ coefficient of $\psi$ using the inner product formula.
            \begin{equation*}
                \braket{\psi_m}{\psi} = \sum_nc_n\underbrace{\braket{\psi_m}{\psi_n}}_{\delta_{nm}} = c_m
            \end{equation*}
            \begin{itemize}
                \item Equivalently,
                \begin{equation*}
                    c_m = \int\dd{\vec{r}}\ \psi_m^*(\vec{r})\psi(\vec{r})
                \end{equation*}
            \end{itemize}
        \end{itemize}
    \end{itemize}
    \pagebreak
    \item Computational things to remember for the midterm.
    \item The harmonic oscillator.
    \begin{itemize}
        \item Since we are in one dimension, $\hat{p}=\hat{p}_x$
        \item The Hamiltonian is
        \begin{equation*}
            \hat{H} = \frac{\hat{p}^2}{2m}+\frac{k\hat{x}^2}{2}
        \end{equation*}
        \item We have that
        \begin{equation*}
            [\hat{p},\hat{x}] = -i\hbar
        \end{equation*}
        \begin{itemize}
            \item Note that this statement is not only true in the context of the harmonic oscillator. Indeed, $\hat{p}_x$ and $\hat{x}$ always compatibilize in this way.
        \end{itemize}
        \item Recall that compatibility is important because the \emph{generic} uncertainty principle (restated as follows) requires a zero commutator in order for it to be possible for both uncertainties to be zero!
        \begin{equation*}
            \sigma_{\hat{A}}^2\sigma_{\hat{B}}^2 \geq \frac{1}{4}|\ev{[\hat{A},\hat{B}]}{\psi}|^2
        \end{equation*}
        \item We defined ladder operators
        \begin{align*}
            a_+ &= \frac{1}{\sqrt{2\hbar m\omega}}(-i\hat{p}+m\omega\hat{x})&
            a_- &= \frac{1}{\sqrt{2\hbar m\omega}}(i\hat{p}+m\omega\hat{x})
        \end{align*}
        \item Having defined these operators, we may write the Hamiltonian in terms of them as follows.
        \begin{equation*}
            \hat{H} = \hbar\omega\left( a_+a_-+\frac{1}{2} \right)
        \end{equation*}
        \item Defining $\ket{n}:=\ket{\psi_n}$ and remembering that
        \begin{equation*}
            a_+a_-\ket{n} = n\ket{n}
        \end{equation*}
        this form of the Hamiltonian makes it obvious that
        \begin{equation*}
            E_n = \hbar\omega\left( n+\frac{1}{2} \right)
        \end{equation*}
        since
        \begin{align*}
            \hat{H}\ket{n} &= \hbar\omega\left( n+\frac{1}{2} \right)\ket{n}&
            \ev{\hat{H}}{n} &= \hbar\omega\left( n+\frac{1}{2} \right)
        \end{align*}
        \item The ladder operators also have distinctive actions on the energy eigenstates.
        \begin{align*}
            a_-\ket{n} &= \sqrt{n}\ket{n-1}&
            a_+\ket{n} &= \sqrt{n+1}\ket{n+1}
        \end{align*}
        \item Don't forget that overall,
        \begin{equation*}
            \braket{n}{m} = \delta_{nm}
        \end{equation*}
        \item The ladder operators enable us to calculate the observables of a generic state $\psi$ of the harmonic oscillator as follows.
        \begingroup
        \allowdisplaybreaks
        \begin{align*}
            \ev{\hat{x}}{\psi} &= \sqrt{\frac{\hbar}{2M\omega}}\ev{(a_++a_-)}{\psi}\\
            &= \sum_{m,n}c_m^*c_n\sqrt{\frac{\hbar}{2M\omega}}\mel{m}{(a_++a_-)}{n}\e[i(E_m-E_n)t/\hbar]\\
            &= \sum_{m,n}c_m^*c_n\sqrt{\frac{\hbar}{2M\omega}}\e[i(E_m-E_n)t/\hbar]\big( \sqrt{n+1}\underbrace{\braket{m}{n+1}}_{\delta_{m,n+1}}+\underbrace{\sqrt{n}\braket{m}{n-1}}_{\delta_{m,n-1}} \big)\\
            &= \sum_{n=0}^\infty c_{n+1}^*c_n\e[i\omega t]\sqrt{n+1}+\sum_{n=0}^\infty c_{n-1}^*c_n\e[-i\omega t]\sqrt{n}\\
            &= \sum_{n=0}^\infty\left( c_{n+1}^*c_n\e[i\omega t]+c_n^*c_{n+1}\e[-i\omega t] \right)\sqrt{n+1}
        \end{align*}
        \endgroup
        \begin{itemize}
            \item Note that in the next to last line above, the second sum \emph{can} go from zero to $\infty$ because for the $n=0$ term, although we have an undefined $c_{-1}$, we also have $\sqrt{0}=0$ so the problematic "undefined" term vanishes.
            \item We can expect to see a computation like this in the midterm.
        \end{itemize}
        \item Using similar methods, we can calculate that
        \begin{equation*}
            \ev**{\frac{k\hat{x}^2}{2}}{n} = \frac{E_n}{2}
            = \ev{\hat{p}^2}{n}
            = \frac{\hbar\omega}{2}\left( n+\frac{1}{2} \right)
        \end{equation*}
        \begin{itemize}
            \item In particular, we expand
            \begin{equation*}
                \ev{(a_++a_-)^2}{n} = \underbrace{\ev{a_+^2}{n}}_0+\underbrace{\ev{a_-^2}{n}}_0+\underbrace{\ev{a_+a_-}{n}}_n+\ev{\underbrace{a_-a_+}_{a_+a_-+1}}{n}
                = 2n+1
            \end{equation*}
        \end{itemize}
        \item Note that for the same reason discussed above,
        \begin{equation*}
            a_-a_+\ket{n} = (n+1)\ket{n}
        \end{equation*}
        \item Since $\sigma_x^2=\ev{\hat{x}^2}{n}-(\ev{\hat{x}}{n})^2\neq 0$ as we can verify by further calculations, there is \emph{always} some nonzero $\sigma_x$ for the harmonic oscillator.
    \end{itemize}
    \item Final note.
    \begin{itemize}
        \item If we want to compute $\ev{\hat{x}}{\psi}$ for a generic potential, we must use
        \begin{equation*}
            \ev{\hat{x}}{\psi}(t) = \sum_{m,n}c_m^*c_n\e[i(E_m-E_n)t/\hbar]\mel{m}{\hat{x}}{n}
        \end{equation*}
        \item In other words, it is only in the harmonic oscillator specifically that we can use the ladder operators.
        \item If we are in a specific energy eigenstate (of a general potential), though, then we do get conservation of position and momentum because $E_m=E_n$ so $E_m-E_n=0$ removes the time term. In particular,
        \begin{equation*}
            \ev{\hat{x}}{\psi_n(x,t)} = c_n^*c_n\e[i(E_n-E_n)t/\hbar]\ev{\hat{x}}{\psi_n(x)}= c_n^*c_n\ev{\hat{x}}{\psi_n(x)}
        \end{equation*}
        and
        \begin{equation*}
            \dv{t}(\ev{\hat{x}}{\psi_n}) = \frac{i}{\hbar}\ev{[\hat{H},\hat{x}]}{\psi_n}
            = \frac{i}{\hbar}\ev{\hat{H}\hat{x}-\hat{x}\hat{H}}{\psi_n}
            = \frac{i}{\hbar}\left( E_n\ev{\hat{x}}{\psi_n}-E_n\ev{\hat{x}}{\psi_n} \right)
            = 0
        \end{equation*}
        so
        \begin{equation*}
            \dv{t}(\ev{\hat{x}}{\psi_n}) = \dv{t}(\ev{\hat{p}}{\psi_n})
            = 0
        \end{equation*}
        \begin{itemize}
            \item Why does $\ev{\hat{H}\hat{x}}{\psi_n}=E_n\ev{\hat{x}}{\psi_n}$?? I thought $\hat{H}$ and $\hat{x}$ didn't commute.
        \end{itemize}
    \end{itemize}
\end{itemize}




\end{document}