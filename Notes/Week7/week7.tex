\documentclass[../notes.tex]{subfiles}

\pagestyle{main}
\renewcommand{\chaptermark}[1]{\markboth{\chaptername\ \thechapter\ (#1)}{}}
\setcounter{chapter}{6}

\begin{document}




\chapter{Spin, Fermions, and Bosons}
\section{Three-Dimensional Harmonic Oscillator}
\begin{itemize}
    \item \marginnote{2/12:}Last time.
    \begin{itemize}
        \item We discussed some of the problems we face in 3D.
        \item The Hamiltonian is now
        \begin{equation*}
            \hat{H} = -\frac{\hbar^2}{2m}\left[ \pdv[2]{x}+\pdv[2]{y}+\pdv[2]{z} \right]+V(x,y,z)
        \end{equation*}
        \begin{itemize}
            \item Derivatives in three coordinates.
            \item The potential is time-independent.
            \item If the potential does not depend on anything more specific (e.g., is not central, for instance), then only $\hat{H}$ is conserved.
        \end{itemize}
        \item We solve
        \begin{equation*}
            \hat{H}\psi(x,y,z) = E\psi(x,y,z)
        \end{equation*}
        for $\psi,E$.
        \item There are three compatible operators:
        \begin{equation*}
            \hat{H},\ \hat{\vec{L}}{\,}^2,\ \hat{L}_z
        \end{equation*}
        \begin{itemize}
            \item The $z$-angular momentum operator, in particular, has the form
            \begin{equation*}
                \hat{L}_z = -i\hbar\pdv{\phi}
            \end{equation*}
            which is analogous to the form $\hat{p}_z=-i\hbar(\pdv*{z})$.
        \end{itemize}
        \item The potential is central, i.e.,
        \begin{equation*}
            V(x,y,z) = V(r) = V(\sqrt{x^2+y^2+z^2})
        \end{equation*}
        \begin{itemize}
            \item If the potential is depends on $r$, we solve the ODE in polar coordinates $(r,\theta,\phi)$.
        \end{itemize}
    \end{itemize}
    \item There are also many cases when we only have
    \begin{equation*}
        V(x,y,z) = V(\sqrt{x^2+y^2})
    \end{equation*}
    \begin{itemize}
        \item In this case, $\hat{H},\hat{L}_z,\hat{p}_z$ will all be compatible.
    \end{itemize}
    \item If the potential depends via
    \begin{equation*}
        V(x,y,z) = V(\sqrt{x^2+y^2},z)
    \end{equation*}
    then we will conserve $\hat{H},\hat{L}_z$.
    \begin{itemize}
        \item We will play with this in the problem set.
    \end{itemize}
    \pagebreak
    \item Today, we begin with the \textbf{asymmetric harmonic oscillator}.
    \item \textbf{Asymmetric harmonic oscillator}: A particle subject to the following three-dimensional potential. \emph{Constraint}
    \begin{equation*}
        V(x,y,z) = \frac{M\omega_1^2x^2}{2}+\frac{M\omega_2^2y^2}{2}+\frac{M\omega_3^2z^2}{2}
    \end{equation*}
    \begin{itemize}
        \item This potential is special in the sense that it allows us to solve by separation of variables.
        \item In other words, since we can write the ODE in the form
        \begin{equation*}
            \left[ -\frac{\hbar^2}{2M}\pdv[2]{\psi}{x}+\frac{M\omega_1^2x^2}{2}\psi \right]+\left[ -\frac{\hbar^2}{2M}\pdv[2]{\psi}{y}+\frac{M\omega_2^2y^2}{2}\psi \right]+\left[ -\frac{\hbar^2}{2M}\pdv[2]{\psi}{z}+\frac{M\omega_3^2z^2}{2}\psi \right] = E\psi
        \end{equation*}
        we may write
        \begin{equation*}
            \psi(x,y,z) = X(x)Y(y)Z(z)
        \end{equation*}
        \item This allows us to algebraically manipulate the ODE into the form
        \begin{equation*}
            \frac{1}{X}\left[ -\frac{\hbar^2}{2M}\dv[2]{X}{x}+\frac{M\omega_1^2x^2}{2}X \right]+\frac{1}{Y}\left[ -\frac{\hbar^2}{2M}\dv[2]{Y}{y}+\frac{M\omega_2^2y^2}{2}Y \right]+\frac{1}{Z}\left[ -\frac{\hbar^2}{2M}\dv[2]{Z}{z}+\frac{M\omega_3^2z^2}{2}Z \right] = E
        \end{equation*}
        \begin{itemize}
            \item We switch from partial to total derivatives here because now each function is only a function of one variable (e.g., $X(x)$ depends only on $x$)!
        \end{itemize}
        \item Since the sum of these three independent terms is equal to a constant, each term must equal a constant!
        \item Splitting the above equation into three, we obtain
        \begin{align*}
            -\frac{\hbar^2}{2M}\dv[2]{X}{x}+\frac{M\omega_1^2x^2}{2}X &= E_{n_1}X\\
            -\frac{\hbar^2}{2M}\dv[2]{Y}{y}+\frac{M\omega_2^2y^2}{2}Y &= E_{n_2}Y\\
            -\frac{\hbar^2}{2M}\dv[2]{Z}{z}+\frac{M\omega_3^2z^2}{2}Z &= E_{n_3}Z
        \end{align*}
        \begin{itemize}
            \item It follows that
            \begin{equation*}
                E = E_{n_1}+E_{n_2}+E_{n_3}
            \end{equation*}
        \end{itemize}
        \item We already know the solution to each of these three ODEs! They are just quantum harmonic oscillators. Thus,
        \begin{equation*}
            E_{n_i} = \hbar\omega_i\left( n_i+\frac{1}{2} \right)
        \end{equation*}
        and
        \begin{equation*}
            E = E_{n_1n_2n_3}
            = \hbar\omega_1\left( n_1+\frac{1}{2} \right)+\hbar\omega_2\left( n_2+\frac{1}{2} \right)+\hbar\omega_3\left( n_3+\frac{1}{2} \right)
        \end{equation*}
        \item Additionally, it follows that the wave functions of each direction are of the form (for example)
        \begin{equation*}
            X_{n_1}(x) = \left( \frac{M\omega_1}{\hbar\pi} \right)^{1/4}\frac{H_{n_1}(\xi_1)}{\sqrt{2^{n_1}n_1!}}\exp[-\frac{\xi_1^2}{2}]
        \end{equation*}
        where $\xi_1=x\sqrt{M\omega_1/\hbar}$.
    \end{itemize}
    \item What happens to $X_{n_1}(x),Y_{n_2}(y)$ in the limiting case that $n_1\to n_2$, $x\to y$, and $\omega_1\to\omega_2$?
    \begin{itemize}
        \item We start approaching something interesting.
        \item We need to go a bit further, though.
    \end{itemize}
    \item Now consider the limiting case where
    \begin{equation*}
        \omega_1 = \omega_2
        = \omega_3
        = \omega
    \end{equation*}
    \begin{itemize}
        \item Herein, the Hamiltonian becomes
        \begin{align*}
            \hat{H} &= -\frac{\hbar^2}{2M}\vec{\nabla}^2+\frac{M\omega^2}{2}(x^2+y^2+z^2)\\
            &= -\frac{\hbar^2}{2M}\vec{\nabla}^2+\frac{M\omega^2r^2}{2}
        \end{align*}
        \item In this \emph{central potential}, recall that we have
        \begin{equation*}
            \hat{\vec{L}}{\,}^2Y_{\ell m}(\theta,\phi) = \hbar^2\ell(\ell+1)Y_{\ell m}(\theta,\phi)
        \end{equation*}
        and
        \begin{equation*}
            \hat{L}_zY_{\ell m}(\theta,\phi) = \hbar mY_{\ell m}(\theta,\phi)
        \end{equation*}
        and
        \begin{equation*}
            -\frac{\hbar^2}{2M}\dv[2]{r}[U_{n\ell}(r)]+\underbrace{\left[ V(r)+\frac{\hbar^2\ell(\ell+1)}{2Mr^2} \right]}_{V_\text{eff}(r)}U_{n\ell}(r) = E_{n\ell}U_{n\ell}(r)
        \end{equation*}
    \end{itemize}
    \item This leads directly into our discussion of the \textbf{spherically symmetric harmonic oscillator}.
    \item \textbf{Spherically symmetric harmonic oscillator}: A particle subject to the following one-dimensional effective potential. \emph{Constraint}
    \begin{equation*}
        V_\text{eff}(r) = \frac{M\omega^2r^2}{2}+\frac{\hbar^2\ell(\ell+1)}{2Mr^2}
    \end{equation*}
    \begin{figure}[h!]
        \centering
        \begin{tikzpicture}
            \small
            \draw [-stealth] (-1,0) -- (4,0) node[right]{$r$};
            \draw [-stealth] (0,-0.5) -- (0,2.5);
    
            \footnotesize
            \draw [blx,thick] plot[domain=0.318:3.95,samples=50,smooth] (\x,{1/(4*\x*\x)}) node[above left]{$\dfrac{\hbar^2\ell(\ell+1)}{2Mr^2}$};
            \draw [rex,thick] plot[domain=0:3.162,smooth] (\x,{\x*\x/4}) node[below=5mm]{$\dfrac{M\omega r^2}{2}$};
            
            \draw [grx,thick] plot[domain=0.318:3.146,samples=50,smooth] (\x,{1/(4*\x*\x)+\x*\x/4});
            \draw [grx]
                (0.648,0.7) -- (1.543,0.7)
                (0.551,0.9) -- (1.816,0.9)
                (0.490,1.1) -- (2.040,1.1)
            ;
            \node [grx] at (1,2) {$V_\text{eff}(r)$};
        \end{tikzpicture}
        \caption{Spherically symmetric harmonic oscillator potential.}
        \label{fig:sphrSymHOV}
    \end{figure}
    \begin{itemize}
        \item The problem we have to solve here is
        \begin{equation*}
            -\frac{\hbar^2}{2M}\dv[2]{r}[U_{n\ell}(r)]+\left[ \frac{M\omega r^2}{2}+\frac{\hbar^2\ell(\ell+1)}{2Mr^2} \right]U_{n\ell}(r) = E_{n\ell}U(r)
        \end{equation*}
        \item Recall that
        \begin{align*}
            \psi(r,\theta,\phi) &= R_{n\ell}(r)Y_{\ell m}(\theta,\phi)&
            R_{n\ell}(r)r &= U_{n\ell}(r)
        \end{align*}
        \item In the effective potential, we have the interplay of two peaking potentials as in Figure \ref{fig:sphrSymHOV}.
        \begin{itemize}
            \item The particle will have certain energy states within the well.
        \end{itemize}
        \item In the limiting case that $r$ is small ($r\to 0$), we can approximate the potential as giving us
        \begin{equation*}
            -\frac{\hbar^2}{2M}\dv[2]{U_{n\ell}}{r}+\frac{\hbar^2\ell(\ell+1)}{2Mr^2}U_{n\ell}+\cdots = 0
        \end{equation*}
        \begin{itemize}
            \item In this case, the solution is proportional to
            \begin{equation*}
                U_{n\ell} \propto Cr^{\ell+1}
            \end{equation*}
            \item This is because
            \begin{align*}
                \dv{r}(Cr^{\ell+1}) &= (\ell+1)Cr^\ell\\
                \dv[2]{r}(Cr^{\ell+1}) &= \ell(\ell+1)C\frac{r^{\ell+1}}{r^2}
            \end{align*}
        \end{itemize}
        \item In the limiting case that $r$ is large ($r\to\infty$), we can approximate the potential as giving us
        \begin{equation*}
            -\frac{\hbar^2}{2M}\dv[2]{U_{n\ell}}{r}+\frac{M\omega^2r^2}{2}U_{n\ell}+\cdots = 0
        \end{equation*}
        \begin{itemize}
            \item In this case, the solution is proportional to
            \begin{equation*}
                U_{n\ell} = C\e[-M\omega r^2/\hbar]
            \end{equation*}
        \end{itemize}
        \item Thus, we combine the two partial solutions to propose the overall ansatz
        \begin{equation*}
            U_{n\ell} = f_{n\ell}r^{\ell+1}\e[-M\omega r^2/\hbar]
        \end{equation*}
        \item Substituting back into the original ODE, we obtain the differential equation
        \begin{equation*}
            f_{n\ell}''+2\left( \frac{\ell+1}{r}-\frac{M\omega r}{\hbar} \right)f_{n\ell}'+\left[ \frac{2ME_{n\ell}}{\hbar^2}-\frac{(2\ell+3)M\omega}{\hbar} \right]f_{n\ell} = 0
        \end{equation*}
        \item As we have previously, propose that
        \begin{equation*}
            f_{n\ell}(r) = \sum_ja_jr^j
        \end{equation*}
        \begin{itemize}
            \item But there's a problem: $f_{n\ell}'(r=0)=a_1$, and this would allow the $(\ell+1)/r$ term to diverge and make the differential equation blow up.
            \item Thus, we choose $a_1=0$ and proceed.
        \end{itemize}
        \item Substituting this power series into the differential equation, we obtain
        \begin{equation*}
            \sum_jj(j-1)a_jr^{j-2}+2\left( \frac{\ell+1}{r}-\frac{M\omega r}{\hbar} \right)\sum_jja_jr^{j-1}+\left[ \frac{2ME_{n\ell}}{\hbar^2}-\frac{(2\ell+3)M\omega}{\hbar} \right]\sum_ja_jr^j = 0
        \end{equation*}
        \item Make a change of variables $j\to j+2$ so that we can start the sum from zero.
        \begin{multline*}
            \sum_{j=0}^\infty(j+2)(j+1)a_{j+2}r^j+2\left( \frac{\ell+1}{r}-\frac{M\omega r}{\hbar} \right)\sum_{j=0}^\infty(j+2)a_{j+2}r^{j+1}\\
            +\left[ \frac{2ME_{n\ell}}{\hbar^2}-\frac{(2\ell+3)M\omega}{\hbar} \right]\sum_{j=0}^\infty a_{j+2}r^{j+2} = 0
        \end{multline*}
        \item We will finish this derivation on Wednesday.
    \end{itemize}
\end{itemize}



\section{Office Hours (Yunjia)}
\begin{itemize}
    \item \marginnote{2/13:}PSet 2, Q2c.
    \begin{itemize}
        \item If we can get up to Equation 12 in the answer key, that's full credit.
        \item The thing with $\kappa_{II}^{-1}$ is the idea that if we have a value that's very large (like $\kappa_{II}$ will be as $V_0\to\infty$ since $\kappa_{II}\propto V_0^{1/2}$), then we can Taylor expand in its reciprocal.
        \begin{itemize}
            \item We cannot Taylor expand in the large values; we can only Taylor expand in small values.
            \item This technique is called \textbf{perturbation theory} and will be a major topic of QMechII; Yunjia's use of it here was admittedly a bit extra.
        \end{itemize}
    \end{itemize}
    \item A brief introduction to perturbation theory.
    \begin{itemize}
        \item Suppose we seek to solve an equation
        \begin{equation*}
            f(x,\epsilon) = 0
        \end{equation*}
        where $\epsilon$ is small.
        \item We can approximate the solution in the form
        \begin{equation*}
            f^{(0)}(x)+f^{(1)}(x)\epsilon+f^{(2)}(x)\epsilon^2 = 0
        \end{equation*}
        where the digit superscripts in parentheses just denote different functions, not derivatives or anything like that. For example, we could equally well have used the notation $f,g,h$; it's just that this is less general.
        \item To solve the original equation, we first solve
        \begin{equation*}
            f^{(0)}(x_0) = 0
        \end{equation*}
        for $x_0$.
        \item Then we solve
        \begin{equation*}
            f^{(0)}(x_0+\epsilon x_1)+\epsilon f^{(1)}(x_0) = 0
        \end{equation*}
        for $x_1$.
        \item Continuing in this fashion, our solution takes on the following form and is progressively refined as more terms are calculated.
        \begin{equation*}
            x = x_0+\epsilon x_1+\epsilon^2x_2+\cdots
        \end{equation*}
    \end{itemize}
\end{itemize}




\end{document}